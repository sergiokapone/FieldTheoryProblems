% !TeX program = lualatex
% !TeX encoding = utf8
% !TeX spellcheck = uk_UA
% !TeX root =../FTProblems.tex

%=========================================================
\Opensolutionfile{answer}[\currfilebase/\currfilebase-Answers]
\Writetofile{answer}{\protect\section*{\nameref*{\currfilebase}}}
\chapter{Спеціальна теорія відносності}\label{\currfilebase}
%=========================================================

\begin{Theory}


	Тензор Мінковського
	\begin{equation}
		\eta_{\mu\nu} = \left(
		\begin{array}{cccc}
				1 & 0  & 0  & 0  \\
				0 & -1 & 0  & 0  \\
				0 & 0  & -1 & 0  \\
				0 & 0  & 0  & -1
			\end{array}
		\right).
	\end{equation}

	За допомогою тензора Мінковського можна опускати і підіймати індекси у 4-тензорів. Зокрема, у 4-векторів:
	\[
		\eta_{\mu\nu}A^{\nu} = A_{\mu}, \quad \eta^{\mu\nu}A_{\nu} = A^{\mu}.
	\]


	Контраваріантний 4-вектор густини струму:
	\begin{equation}
		j^\mu = \rho\frac{dx^\mu}{dt} = (c\rho,  \vect{j}),
	\end{equation}
	де $\rho$~-- густина електричного заряду, $\vect{j}$~-- 3-вектор густини струму.


	4-Потенціал електромагнітного поля:

	\begin{subequations}
		\begin{align}
			A^\mu & = (\phi, \vect{A})  \quad \text{--- контраваріантний}, \\
			A_\mu & = (\phi, -\vect{A}) \quad \text{--- коваріантний}.
		\end{align}
	\end{subequations}

	Калібрувальні перетворення:

	\begin{equation}
		A_\mu' = A_\mu - \frac{\partial f}{\partial x^\mu}
	\end{equation}
	Тензор електромагнітного поля:

	\begin{equation}\label{rel_tensor_elmagfield}
		F_{\mu\nu} = \frac{\partial A_\nu}{\partial  x^\mu} - \frac{\partial A_\mu}{\partial  x^\nu},
	\end{equation}

	\noindent у формі матриці тензор $F_{\mu\nu}$ має вигляд:

	\begin{equation}
		F_{\mu\nu} = \left(
		\begin{array}{cccc}
				0    & E_x  & E_y  & E_z  \\
				-E_x & 0    & -B_z & B_y  \\
				-E_y & B_z  & 0    & -B_x \\
				-E_z & -B_y & B_x  & 0
			\end{array}
		\right),
	\end{equation}
	де $\mu,\nu = 0,1,2,3$ ($\mu$~-- нумерує рядки, $\nu$~-- стовпці.)

	Нехай відносно інерціальної системи відліку $K$  з координатами $\{ t, \vect{r} \}$, $\vect{r} = (x,y,z)$ рухається із сталою швидкістю $V$ інша інерціальна система $K'$ з декартовими координатами  $\{ t', \vect{r}' \}$, $\vect{r}' = (x',y',z')$, причому у момент $t = 0$  початки просторових координат збігалися. Перетворення координат $K \to K'$  мають вигляд:


	\begin{equation}\label{eq:Lorentz_transform}
		t' = \frac{t - (\vect{v}\cdot\vect{r})/c^2}{\sqrt{1 - \frac{v^2}{c^2}}}, \quad \vect{r}'_{\perp} = \vect{r}_{\perp}, \quad
		\vect{r}'_{\parallel} = \frac{\vect{r}_{\parallel} - \vect{v}t} {\sqrt{1 - \frac{v^2}{c^2}}},
	\end{equation}
	де позначки $\parallel$  та $\perp$ відповідають поздовжнім (паралельним) та поперечним до $\vect{v}$ складовим тривимірних векторів.

	Відповідні до~\eqref{eq:Lorentz_transform} перетворення поперечних та поздовжніх компонент напруженості електричного поля та індукції магнітного поля мають вигляд
	\begin{equation}\label{eq:Lorentz_transform_F}
		\Efield'_{\parallel}=\Efield_{\parallel}, \quad \Bfield'_{\parallel}=\Bfield_{\parallel}, \quad
		\Efield'_{\perp}= \frac{\Efield_{\perp} + \frac1c\left[ \vect{v}\times\vect\Bfield \right] }{\sqrt{1 - \frac{v^2}{c^2}}}, \quad
		\Bfield'_{\perp}= \frac{\Bfield_{\perp} - \frac1c\left[ \vect{v}\times\vect\Efield \right] }{\sqrt{1 - \frac{v^2}{c^2}}}.
	\end{equation}

	Відповідні до~\eqref{eq:Lorentz_transform} перетворення густини заряду та густини струму мають вигляд
	\begin{equation}\label{eq:Lorentz_transform_j}
		\rho' = \frac{\rho - (\vect{v}\cdot\vect{j})/c^2}{\sqrt{1 - \frac{v^2}{c^2}}}, \quad \vect{j}'_{\perp} = \vect{j}_{\perp}, \quad
		\vect{j}'_{\parallel} = \frac{\vect{j}_{\parallel} - \rho \vect{v}} {\sqrt{1 - \frac{v^2}{c^2}}}.
	\end{equation}

	Із компонент тензора можна утворити інваріанти:
	\begin{equation*}
		F_{\mu\nu}F^{\mu\nu} = \inv, \quad e^{\mu\nu\alpha\beta}F_{\mu\nu}F_{\alpha\beta} = \inv,
	\end{equation*}
	де $e^{\mu\nu\alpha\beta}$~--- повністю антисиметричний тензор. У тривимірному представленні інваріанти перетворень Лоренца виглядають таким чином:
	\begin{equation*}
		E^2 - B^2 = \inv, \quad \Efield\cdot\Bfield = \inv.
	\end{equation*}

	Рівняння руху частинки в електромагнітному полі:
	\begin{equation}
		mc\frac{du^\mu}{ds} = \frac{e}{c}F^{\mu\nu}u_{\nu},
	\end{equation}
	де $u_{\nu}$~-- 4-швидкість частинки.

	Рівняння Максвелла в тензорній формі у вакуумі%

	\begin{gather}
		\frac{\partial F^{\mu\nu}}{\partial x^\nu} =                                                                                -\frac{4\pi}{c}j^\mu \label{eqn:rel_Maxwell_eqn_with_sources}\\
		\frac{\partial F_{\beta\gamma}}{\partial x^\alpha} + \frac{\partial F_{\gamma\alpha}}{\partial x^\beta} +\frac{\partial
				F_{\alpha\beta}}{\partial x^\gamma}  = 0. \label{eqn:rel_Maxwell_eqn_with_sourceless}
	\end{gather}

	Тензор енергії-імпульсу електромагнітного поля:

	\begin{equation}\label{rel_TEI}
		T^{\mu\nu} = \frac{1}{4\pi}\left( -F^{\mu \alpha}F^\nu_{\;\;\alpha} + \frac14\eta^{\mu\nu}F_{\alpha\beta}F^{\alpha\beta}\right).
	\end{equation}
\end{Theory}

\section{Перетворення Лоренца та закони збереження}

%=========================================================
\begin{problem}
У деякій системі $K$ задано компоненти \emph{простороподібного} коваріантного вектора $A_{\mu}$. Чи існує система відліку $K'$, де компоненти цього вектора мають вигляд
\begin{enumerate*}[label=\alph*)]
	\item $A'_{\mu} = \{a,0,0,0\}$?
	\item $A'_{\mu} = \{0,a,0,0\}$?
\end{enumerate*}
Якщо це так, виразіть~$A_{\mu}$.
\begin{solution}
	\begin{enumerate*}[label=\alph*)]
		\item Не існує, оскільки $A'_{\mu}$ часоподібний.
		\item $A_{\mu} = \left\lbrace -\frac{Va}{\sqrt{1-V^2}},\frac{a}{\sqrt{1-V^2}},0,0 \right\rbrace$.
	\end{enumerate*}
\end{solution}
\end{problem}

%=========================================================
\begin{problem}
У деякій системі $K$ задано компоненти \emph{часоподібного} коваріантного вектора $A_{\mu}$. Чи існує система відліку $K'$, де компоненти цього вектора мають вигляд
\begin{enumerate*}[label=\alph*)]
	\item $A'_{\mu} = \{a,0,0,0\}$?
	\item $A'_{\mu} = \{0,a,0,0\}$?
\end{enumerate*}
Якщо це так, виразіть~$A_{\mu}$.
\begin{solution}
	\begin{enumerate*}[label=\alph*)]
		\item $A_{\mu} = \left\lbrace \frac{a}{\sqrt{1-V^2}},-\frac{aV}{\sqrt{1-V^2}},0,0 \right\rbrace$.
		\item Не існує, оскільки $A'_{\mu}$ простороподібний.
	\end{enumerate*}
\end{solution}
\end{problem}

%=========================================================
\begin{problem}
Дві релятивістські частинки з масами спокою  $m_1$ та $m_2$ рухаються вздовж деякої прямої з імпульсами $p_1$ та $p_2$, відповідно. Знайти швидкість центру мас цієї системи. Використайте систему одиниць в якій $c = 1$.
\begin{solution}
	$V_C = \frac{p_1 + p_2}{\sqrt{p_1^2 + m_1^2} + \sqrt{p_2^2 + m_2^2}}$.
\end{solution}
\end{problem}

%%=========================================================
%\begin{problem}
%Покажіть, що вільний електрон у вакуумі не може поглинути або випромінювати фотон.
%\begin{solution}
%	Спосіб 1 (як приклад застосування операцій з 4-векторами).
%
%	Запишемо закон збереження 4-імпульсу в системі відліку електрона:
%	\[
%		{p}_{\gamma}^{\mu} + {p}_e^{\mu} = {p'}_e^{\mu},
%	\]
%	де ${p}_{\gamma}^{\mu}$~--- 4-імпульс фотона, ${p}_e^{\mu}= m_e$~--- 4-імпульс електрона до поглинання (чи випромінювання), ${p'}_e^{\mu}$~--- 4-імпульс електрона після поглинання чи випромінювання (електрон віддачі).
%
%	Піднесемо до квадрату ліву і праву частину рівняння, врахувавши, що квадрат 4-вектора імпульсу дорівнює квадрату маси частинки. Для електрона ${p}_e^{\mu}{p}_{e\mu} = m_e^2$, для фотона ${p}_{\gamma\,\mu}{p}_\gamma^\mu = 0$. Отже:
%	\[
%		0 + 2  {p}_{\gamma\,\mu}\cdot{p}_{e\,\mu}	 + m_e^2 = m_e^2,
%	\]
%	звідки
%	\[
%		{p}_{\gamma\,\mu}\cdot{p}_{e\,\mu} = 0.
%	\]
%	або
%	\[
%		E\cdot m = 0.
%	\]
%	Остання рівність може виконатись лише при $E = 0$, тобто за відсутності фотона. Отже вільний електрон у вакуумі не може поглинати або випромінювати фотон.
%
%	Спосіб 2 (міркування на основі закону збереження енергії).
%
%	Перейдемо у власну систему електрона. Його енергія $m_e$. Після випромінювання фотона енергія електрона не може зменшитися (оскільки він ще й набуває імпульс віддачі). Сумарна енергія електрона й фотона виходить  більше $m_e$, що неможливо в силу збереження енергії.
%\end{solution}
%\end{problem}
%
%%=========================================================
%\begin{problem}
%Покажіть, що електрон-позитронна пара у вакуумі не може анігілювати в один фотон.
%\end{problem}

%=========================================================
\begin{problem}
Вільна частинка, що рухається з швидкістю $v < c$, розпадається на два фотони (які можуть вилітати під довільним кутом до напрямку руху частинки). Знайти мінімальний кут між напрямками розльоту фотонів.
\end{problem}

%=========================================================
\begin{problem}
При зіткненні частинки з масою спокою $m$ та імпульсом $p$ із нерухомою частинкою з масою $m_1$ відбувається реакція, після якої залишаються частинки з сумою мас спокою $\ge M$. Знайдіть мінімальне значення кінетичної енергії частинки, що налітає. \emph{Вказівка}: спочатку треба показати, що кінетична енергія системи в с.ц.м. є монотонною функцією $p$.
\end{problem}


%=========================================================
\begin{problem}
Електрон з імпульсом $р$ та фотон з частотою $\omega$ рухаються вздовж прямої (в лабораторній системі); відбувається пружне зіткнення. Заданий кут $\theta = \pi/2$ між напрямками імпульсів фотона та електрона після зіткнення. Знайти відношення частот фотона до і після зіткнення. Дослідити, як зміниться (збільшиться чи зменшиться) частота фотона? Розглянути випадки, коли
\begin{enumerate*}[label=\alph*)]
	\item електрон знаходився в спокої до зіткнення
	\item був ультра-релятивістським.
\end{enumerate*}
Коли частота фотона не зміниться після зіткнення?
%\begin{solution}
%Система одиниць $\hbar = c = 1$. Закони збереження і умова $\theta = \pi/2$ дають умову іспування розв'язку:
%\[
%(E - \omega)^2 \ge 4\omega p
%\]
%\end{solution}
\end{problem}

%=========================================================
\begin{problem}
У загальній ситуації $\pi^0$–мезон розпадається на два $\gamma$-кванта; у власній системі мезона розподіл по кутах ізотропний. Пучок $\pi^0$-мезонів рухається зі швидкістю $\vect{v}$ відносно лабораторії. Знайти (в лабораторній системі) відношення числа $\gamma$-квантів розпаду, що випромінюються у передню напівсферу відносно напрямку руху, до числа $\gamma$-квантів розпаду, що випромінюються у задню напівсферу.
\begin{solution}
	$\frac{N_\rightarrow}{N_\leftarrow} = \frac{1+ \beta}{1-\beta}$, де $\beta = \frac{v}{c}$.
\end{solution}
\end{problem}

%=========================================================
\begin{problem}
    4-хвильовий вектор електромагнітної хвилі має вигляд $k^{\mu} = \{\omega/c,\vect{k}\}$, де $\vect{k}$- 3-хвильовий вектор. Використовуючи закон перетворення 4-векторів при переході від однієї системи відліку до іншої, знайдіть закон перетворення частоти $\omega$ хвилі (ефект Допплера) і хвильового вектора $\vect{k}$.
\end{problem}



%=========================================================
\begin{problem}
Електричне та магнітне поля відносно системи $K$ (декартові координати) визначаються векторами $\Efield = a(1,0,0)$ та $\Bfield = a(1,0,2)$. Відносно системи $K'$ ці ж поля $\Efield' = (E_x',a,0)$ та $\Bfield' = (a,B_y',a)$ --- невідомі. Система $K''$ рухається відносно системи $K'$ з швидкістю $v$ ($c=1$) вздовж осі $OX$. Знайдіть поля  $\Efield'$ та $\Bfield'$ відносно $K'$ та $\Efield''$ та $\Bfield''$ відносно $K''$.
\begin{solution}
	$\Efield' = a(-1,1,0)$, $\Bfield' = a(1,2,1)$, $\Efield'' = a(-1,\Gamma(1- v),2\Gamma v)$, $\Bfield'' = a(1,2\Gamma,\Gamma(1- v))$, де $\Gamma = \frac{1}{\sqrt{1 - v^2}}$.
\end{solution}
\end{problem}


%=========================================================
%\begin{problem}
%У лабораторній системі задане однорідне електричне $\Efield$ і магнітне $\Bfield$ поля. Визначити швидкість системи відліку, в якій електричне і магнітне поля будуть паралельні. Чому вони дорівнюють?
%\end{problem}

%=========================================================
\begin{problem}
В системі $K$ $\Efield\perp\Bfield$. Яка швидкість системи $K'$,  де залишиться лише одне з полів?
\end{problem}


\section{Електродинаміка в релятивістських позначеннях}

%=========================================================
\begin{problem}
Доведіть, що  згортка довільного симетричного тензора з антисиметричним дорівнює $0$.
%\begin{solution}
%	Розглянемо деякий симетричний тензор $S_{\mu\nu}$ і антисиметричний тензор $A_{\mu\nu}$. Тоді
%	\[
%		S_{\mu\nu}A_{\mu\nu} = - S_{\nu\mu}A_{\nu\mu}
%	\]
%	Перепозначивши індекси підсумовування, ми отримуємо, що цей вираз дорівнюватиме
%$-S_{\mu\nu}A_{\mu\nu}$, а отже дорівнюватиме $0$.
%\end{solution}
\end{problem}

%=========================================================
\begin{problem}
Доведіть калібрувальну інваріантність тензора електромагнітного поля $F^{\mu\nu}$ при перетвореннях
4-вектора потенціалу $A_{\mu} \rightarrow A_{mu}+\partial_{\mu} \chi$.
\end{problem}

%=========================================================
\begin{problem}
Показати, що з рівнянь Максвелла з джерелами~\eqref{eqn:rel_Maxwell_eqn_with_sources}
\[
	\frac{\partial F^{\mu\nu}}{\partial x^\nu} = -\frac{4\pi}{c}j^\mu,
\]
випливає закон збереження для заряду $\partial_\nu j^{\nu} = 0$.
\end{problem}

%=========================================================
\begin{problem}
Показати, що з виразу тензора електромагнітного поля через потенціали~\eqref{rel_tensor_elmagfield} $F_{\mu\nu} = \frac{\partial A_\nu}{\partial  x^\mu} - \frac{\partial A_\mu}{\partial  x^\nu}$ випливають рівняння~\eqref{eqn:rel_Maxwell_eqn_with_sourceless}:
\[
    \frac{\partial F_{\beta\gamma}}{\partial x^\alpha} + \frac{\partial F_{\gamma\alpha}}{\partial x^\beta} +\frac{\partial
				F_{\alpha\beta}}{\partial x^\gamma}  = 0.
\]
\end{problem}


%=========================================================
\begin{problem}
Отримайте вираз для енергії і імпульсу електромагнітного поля через напруженість $\Efield$ та індукцію $\Bfield$ виходячи з тензора енергії-імпульсу~\eqref{rel_TEI}
\[
	T^{\mu\nu} = \frac{1}{4\pi}\left( -F^{\mu \alpha}F^\nu_{\;\;\alpha} + \frac14\eta^{\mu\nu}F_{\alpha\beta}F^{\alpha\beta}\right).
\]
\end{problem}

\section{Рух заряджених частинок в електромагнітному полі}

%=========================================================
\begin{problem}
Релятивістська частинка з масою спокою $m$ рухається в однорідних сталих  магнітному  $\Bfield = (0,0,B)$ та електричному полі $\Efield = (E,0,0)$, $\Efield\perp\Bfield$  і $E = B$. В момент $t=0$ частинка знаходилась у початку декартових координат та мала імпульс $\vect{p} = (p_{0_x}, p_{0_y}, p_{0_z})$ . Знайти залежність 4-швидкості $u^{\alpha}$, 4-прискорення $a^{\alpha}$, а також усіх чотирьох координат частинки і згортки $u_{\alpha}a^{\alpha}$  (знайти явним обчисленням) від власного часу $\tau$.
\end{problem}

%========================================`=================
\begin{problem}
Електричне та магнітне поля відносно системи $K$ (декартові координати) визначаються векторами $\Efield = a(0,1,0)$ та $\Bfield = a(1,1,0)$. Спостерігач (система $K'$) рухається з швидкістю $v=3c/5$ вздовж бісектриси кута $XOY$. Знайдіть електромагнітне поле в системі спостерігача $K'$ і, використовуючи результат, знайдіть прямим обчисленням величину $\Efield^{\prime^2} - \Bfield^{\prime^2}$  в цій системі.
\end{problem}

%=========================================================
\begin{problem}%Меледин 5.13
У схрещених полях $\Efield\perp\Bfield$ рухається заряджена частинка. З якою швидкістю і в якому напрямку вона повинна рухатися, щоб її траєкторія залишалася прямолінійною, а швидкість постійною?
\begin{solution}
	$\vect{v} = c \frac{\left[ \Efield\times\Bfield\right] }{B^2}$, при $E<B$.
\end{solution}
\end{problem}

%=========================================================
\begin{problem}
Визначте кут $\theta$ розсіювання нерелятивістської частинки масою $m$ з електричним зарядом $q$, що рухається в електричному полі нерухомого розсіювального центру, що має заряд $Q$. Швидкість частинки до розсіювання $v_0$ і значення прицільного параметра $b$ вважати відомими.
\begin{solution}
	$\tg\frac{\theta}{2} = \frac qm \frac{Q}{bv_0^2}$.
\end{solution}
\end{problem}

\begin{problem}
Знайти переріз розсіяння на малі кути нерелятивістських електронів, швидкість на нескінченності яких $v_0$, що пролітають з великим прицільним параметром $\rho$ повз сферу радіуса $R$, якщо:
\begin{enumerate*}[label=\alph*)]
	\item сфера провідна і заземлена;
	\item сфера провідна і ізольована.
\end{enumerate*}

\begin{solution}
	\begin{enumerate*}[label=\alph*)]
		\item провідна заземлена сфера: $\frac{d\sigma}{d\Omega} \approx \frac{\pi Re^2}{4mυ_0^2} \cdot \frac{1}{\theta ^3}$,
		\item провідна ізольована сфера: $\frac{d\sigma}{d\Omega} \approx \frac{\pi R^2}{8} \cdot \sqrt {\frac{3e^2}{R\pi mυ_0^2}}\cdot\theta^{-5/2}$.
	\end{enumerate*}
\end{solution}
\end{problem}

%=========================================================
\begin{problem}\label{MotioninEM}
Нерелятивістська частинка з питомим зарядом $q/m$, що рухається в присутності постійних і однорідних електромагнітних полів, заданих у вигляді $\Efield = E_0\vect{e}_y$ та $\Bfield = B_0\vect{e}_z$. В початковий момент частинка знаходиться в стані спокою у початку координат. Знайти закони руху частинки. Зобразіть траєкторію частинки в площині $xOy$.
\begin{solution}
	$x(t)  = \frac{E_0}{B_0}c\left( t - \frac{1}{\Omega}\sin\Omega t \right) $,
	$y(t)  = \frac{E_0}{B_0} \frac{c}{\Omega}\left( 1 - \cos\Omega t \right) $, де $\Omega = \frac{qB_0}{mc}$~--- циклотронна частота.
\end{solution}
\end{problem}

%=========================================================
\begin{problem}
Розглянути задачу~\ref{MotioninEM} випадку релятивістських рухів частинки. Знайти її траєкторію, як функцію власного часу для різних знаків інваріанту $\Efield_0^2-\Bfield_0^2$.
\end{problem}


%=========================================================
\begin{problem}
Розв'язати задачу~\ref{MotioninEM} для випадку, якщо магнітне поле задане у вигляді $\Bfield = B_0\vect{e}_z$ електричне поле змінюється за законом $\Efield = E_0\cos\Omega t\,\vect{e}_y$,  де $\Omega = \frac{qB_0}{mc}$~--- циклотронна частота.
\begin{solution}
	$x(t)  = \frac{qE_0}{2m\Omega^2}\left( \sin\Omega t - \Omega t\cos\Omega t \right) $,
	$y(t)  = \frac{qE_0}{2m\Omega}t\sin\Omega t$.
\end{solution}
\end{problem}


%%=========================================================
%\begin{problem}
%Розглянемо пучок іонів питомим зарядом $q/m$, що рухається влітають в область постійних і однорідних паралельних електромагнітних полів, заданих у вигляді $\Efield = E_0\vect{e}_y$ та $\Bfield = B_0\vect{e}_y$. з швидкістю $\vect{v} = v_0 \vect{e}_x$. На відстані $l$ від точки початку координат знаходиться плоский екран, орієнтований перпендикулярно осі $Ox$. Знайти рівняння сліду іонів на екрані. Показати, що при $z \ll l$ слід матиме вигляд параболи.
%\begin{solution}
%	$z = l\tg\sqrt{\frac{qB_0^2}{2mc^2E_0}y}$.
%\end{solution}
%\end{problem}

%=========================================================
\begin{problem}
Знайдіть диференціальний переріз розсіювання заряджених частинок на кулонівському потенціалі. Перевірте, що повний переріз розсіювання нескінченний (розбігається за малих кутів).
\end{problem}

\section{Випромінювання релятивістських заряджених частинок}

%=========================================================
\begin{problem}[Синхротронне випромінювання]% МГ Иванов с. 202
Знайти енергію випромінювання релятивістського електрона в однорідному магнітному полі за один оберт. Знайти повну потужність (в мегаватах) синхротронного випромінювання в прискорювачі з енергією $100$~ГеВ. Довжина кола прискорювача $30$~км, число прискорюваних частинок в кільці $5\cdot 10^{12}$. Оцінити характерну довжину хвилі випромінювання.
\end{problem}

%=========================================================
%\begin{problem}% МГ Иванов с. 202
%Пучок релятивістських електронів пролітає крізь плоский конденсатор паралельно обкладинкам, до якого прикладена змінна напруга з частотою $\omega_0$. Знайти частоту випромінювання електронів в залежності від кута $\theta$ між напрямком випромінювання і напрямком руху пучка.
%\end{problem}


\Closesolutionfile{answer}

