\protect \section *{\nameref *{SpecialRelativity}}
\begin{Solution}{4.{1}}
	\begin{enumerate*}[label=\alph*)]
		\item Не існує, оскільки $A'_{\mu}$ часоподібний.
		\item $A_{\mu} = \left\lbrace -\frac{Va}{\sqrt{1-V^2}},\frac{a}{\sqrt{1-V^2}},0,0 \right\rbrace$.
	\end{enumerate*}
\end{Solution}
\begin{Solution}{4.{2}}
	\begin{enumerate*}[label=\alph*)]
		\item $A_{\mu} = \left\lbrace \frac{a}{\sqrt{1-V^2}},-\frac{aV}{\sqrt{1-V^2}},0,0 \right\rbrace$.
		\item Не існує, оскільки $A'_{\mu}$ простороподібний.
	\end{enumerate*}
\end{Solution}
\begin{Solution}{4.{3}}
	$V_C = \frac{p_1 + p_2}{\sqrt{p_1^2 + m_1^2} + \sqrt{p_2^2 + m_2^2}}$.
\end{Solution}
\begin{Solution}{4.{7}}
	$\frac{N_\rightarrow}{N_\leftarrow} = \frac{1+ \beta}{1-\beta}$, де $\beta = \frac{v}{c}$.
\end{Solution}
\begin{Solution}{4.{9}}
	$\Efield' = a(-1,1,0)$, $\Bfield' = a(1,2,1)$, $\Efield'' = a(-1,\Gamma(1- v),2\Gamma v)$, $\Bfield'' = a(1,2\Gamma,\Gamma(1- v))$, де $\Gamma = \frac{1}{\sqrt{1 - v^2}}$.
\end{Solution}
\begin{Solution}{4.{12}}
	Розглянемо деякий симетричний тензор $S_{\mu\nu}$ і антисиметричний тензор $A_{\mu\nu}$. Тоді
	\[
		S_{\mu\nu}A_{\mu\nu} = - S_{\nu\mu}A_{\nu\mu}
	\]
	Перепозначивши індекси підсумовування, ми отримуємо, що цей вираз дорівнюватиме $-S_{\mu\nu}A_{\mu\nu}$, а отже дорівнюватиме $0$.
\end{Solution}
\begin{Solution}{4.{19}}
	$\vect{v} = c \frac{\left[ \Efield\times\Bfield\right] }{B^2}$, при $E<B$.
\end{Solution}
\begin{Solution}{4.{20}}
	$\tg\frac{\theta}{2} = \frac qm \frac{Q}{bv_0^2}$.
\end{Solution}
\begin{Solution}{4.{21}}
	\begin{enumerate*}[label=\alph*)]
		\item провідна заземлена сфера: $\frac{d\sigma}{d\Omega} \approx \frac{\pi Re^2}{4mυ_0^2} \cdot \frac{1}{\theta ^3}$,
		\item провідна ізольована сфера: $\frac{d\sigma}{d\Omega} \approx \frac{\pi R^2}{8} \cdot \sqrt {\frac{3e^2}{R\pi mυ_0^2}}\cdot\theta^{-5/2}$.
	\end{enumerate*}
\end{Solution}
\begin{Solution}{4.{22}}
	$x(t)  = \frac{E_0}{B_0}c\left( t - \frac{1}{\Omega}\sin\Omega t \right) $,
	$y(t)  = \frac{E_0}{B_0} \frac{c}{\Omega}\left( 1 - \cos\Omega t \right) $, де $\Omega = \frac{qB_0}{mc}$~--- циклотронна частота.
\end{Solution}
\begin{Solution}{4.{24}}
	$x(t)  = \frac{qE_0}{2m\Omega^2}\left( \sin\Omega t - \Omega t\cos\Omega t \right) $,
	$y(t)  = \frac{qE_0}{2m\Omega}t\sin\Omega t$.
\end{Solution}
