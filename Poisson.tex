% !TeX program = lualatex
% !TeX encoding = utf8
% !TeX spellcheck = uk_UA


\documentclass[]{ProblemBook}

\begin{document}

Нехай задана нескінченна область простору $\Omega_\text{out}$, яка не містить зарядів і оточує внутрішню область $\Omega_\text{in}$, яка містить розподіл заряду $\rho(\vect{r})$. Поверхня розділу між областями $\partial\Omega$ є гладкою . Заряди створюють електричне поле, яке характеризується потенціалом $\pot(\vect{r})$. Металеві провідники відсутні. 

%\begin{center}
%		\begin{tikzpicture}
%			\pgfmathsetseed{5}
%			%        \draw (-3,-3) rectangle (3,3);
%			\draw[fill=gray!20, draw=gray, dashed] plot [smooth cycle, samples=8,domain={1:8}] (\x*360/8+2*rnd:4cm+2cm*rnd);
%			\pgfmathsetseed{10}
%			\draw[fill=red!20] plot [smooth cycle, samples=8,domain={1:8}] (\x*360/8+6*rnd:1cm+2cm*rnd) ;
% 
%			\node at (-3,3) {$\Omega_\text{out}$};
%%			\node at (3,-3) {$\rho{\vect{r}}$};
%			\node at (-1.5,1) {$\Omega_\text{in}$};
%			\node at (1.5,0) {$\rho(\vect{r})$};
%			\draw[-latex, thick] (2,0.8) -- ++(55:0.5) node[above] {$\vect{n}$};
%			\draw[-latex, thick] (-1.5,-1.55) -- ++(-128:0.5) node[below] {$\vect{n}$};
%			\draw[-latex] (-1.5,2.3) node[above right] {$\partial\Omega$} to[bend right] ++(-128:0.5) ;
%		\end{tikzpicture}
%		\captionof{figure}{До пояснення основної задачі електростатики}
%		\label{pic:main_elstat_problem}
%	\end{center}

У випадку, якщо заряди зосереджені у скінченній області, на потенціал накладаються умови скінченності:
\begin{equation}\label{eq:Efinity}
	\lim\limits_{r \to \infty } \left( {\left| {\nabla \pot({\vect{r}})} \right|{r^2}} \right) = 0,\quad r = \left| {\vect{r}} \right|.
\end{equation}

Оскільки потенціал є неперервною функцією, а за відсутності поверхневих зарядів неперервним є також і градієнт потенціалу, то на поверхні $\partial\Omega$, мають виконуватись умови 
\begin{equation}\label{eq:BC}
    \begin{aligned}
        \pot_1(\vect{r}) &=  \pot_2(\vect{r}), \quad \vect{r} \in \partial\Omega, \\[1em]
        \left.\frac{\partial\pot_1(\vect{r})}{\partial n_1}\right|_{\vect{r} \in \partial\Omega} &= 
    \left.\frac{\partial\pot_2(\vect{r})}{\partial n_2}\right|_{\vect{r} \in \partial\Omega}. 
    \end{aligned}  
\end{equation}

Зв'язок між потенціалом  і густиною заряду в даній точці $\vect{r}$ простору дається рівнянням Пуассона:

\begin{equation}\label{eq:Poisson}
\vect{\nabla}^2\pot(\vect{r}) = -4\pi\rho(\vect{r}).
\end{equation}


Зв'язок між потенціалом в точці $\vect{r}$ і зарядами, розташованими в точці $\vect{r}'$ ($\vect{r} \neq \vect{r}'$) дається рівнянням Лапласа:

\begin{equation}\label{eq:Laplase}
    \vect{\nabla}^2\pot(\vect{r}) = 0.
\end{equation}

У випадку, якщо заряди зосереджені в скінченній області простору $\Omega$, то розв'язком рівняння~\eqref{eq:Poisson} (або ж Лапласа) буде вираз:
\begin{equation}\label{eq:IntSolution}
\pot(\vect{r}) = \int\frac{\rho(\vect{r}') dV'}{|\vect{r} - \vect{r}'|}.
\end{equation}
Дійсно, не важко переконатись прямим обчисленням (враховуючи співвідношен\-ня $\vect{\nabla}^2\left( \frac{1}{|\vect{r} - \vect{r}'|}\right) = -4\pi\delta(|\vect{r} - \vect{r}'|) $:
\[
    \vect{\nabla}^2 \pot = \int\rho(\vect{r}') \vect{\nabla}^2\left( \frac{1}{|\vect{r} - \vect{r}'|}\right)  dV' =  -4\pi\int\rho(\vect{r}') \delta(|\vect{r} - \vect{r}'|)  dV' = -4\pi\rho(\vect{r}).
\]


При обчислені потенціалу на практиці  рідко доводиться користуватись виразом~\eqref{eq:IntSolution}, бо безпосереднє застосування цієї формули є громіздким.  Натомість, краще  безпосередньо розв'язувати рівняння Пуассона~\eqref{eq:Poisson} та Лапласа~\eqref{eq:Laplase} відповідно, у областях де є заряд, і де його немає. На границі цих областей для забезпечення неперервності потенціалу, треба скористатись умовами~\eqref{eq:BC}.

На відміну від  виразу~\eqref{eq:IntSolution}, рівняння Пуассона та Лапласа є більш універсальними, оскільки не передбачають наявності  нормування потенціалу на нуль на нескінченності та умову відсутності заряду на нескінченності. Наприклад, якщо потрібно знайти потенціал однорідно зарядженого плоского шару товщиною $d$, то формула~\eqref{eq:IntSolution} вже не застосовна, оскільки тепер заряди вже присутні і на нескінченності, однак безпосередній розв'язок рівняння Пуассона та Лапласа з використанням~\eqref{eq:BC} не викликає труднощів. 

Ба, навіть у випадку якщо заряд зосереджений в скінченній області простору, застосування формули~\eqref{eq:IntSolution} також складне і тут теж набагато простіше скористатись розв'язуючи рівняння Пуассона (та Лапласа). Для прикладу, спробуйте знайти потенціал однорідно зарядженої кулі у довільній точці простору обома способами.

До методів  безпосереднього розв'язку задачі електростатики за допомогою рівнянь Пуассона (та Лапласа) можна віднести методи представлення розв'язку у вигляді розкладання в ряд по системі ортогональних функцій з невідомими коефіцієнтами (базисні ортогональні функції вибираються в залежності від симетрії задачі). Вже далі невідомі коефіцієнти знаходяться з використанням умов на границях розділу~\eqref{eq:BC}.

Однак, все ж таки формула~\eqref{eq:IntSolution} має застосування. Наприклад, якщо ми хочемо знайти знайти потенціал скінченого зарядженого тіла, то формулу можна представити у вигляді ряду, який називається розкладанням за мультиполями:

\begin{equation}\label{eq:multipole}
	\pot(\vect{r}) = \int\frac{\rho(\vect{r}') dV'}{|\vect{r} - \vect{r}'|} = \frac{q}{r} + \frac{\vect{p}\cdot\vect{r}}{r^3} + \frac16\sum\limits_{i,j} Q_{ij}\frac{3x_ix_j - r^2\delta_{ij}}{r^5} + \ldots,
\end{equation}
де мультипольні моменти
\begin{equation*}\label{}
\begin{aligned}
    q & = \int \rho(\vect{r'}) dV', \quad &&\text{повний заряд тіла }, \\
    \vect{p} &=  \int \rho(\vect{r'}) \vect{r'} dV' , \quad &&\text{дипольний момент тіла}, \\
    Q_{ij} &= \int \rho(\vect{r'}) (3x'_ix'_j - r^{\prime 2}\delta_{ij}) dV', \quad  &&\text{квадрупольний момент тіла}.
\end{aligned}
\end{equation*}


Якщо в нас є тепер такий ряд, то обчисливши для даного тіла відповідні моменти, ми знатимемо потенціал. Крім того, якщо нам буде потрібно обчислити потенціал з певною точністю , то ми без докорів сумління можемо відкинуті доданки, які містять вищі $2^l$-польні моменти, навіть, якщо вони відмінні від нуля.

%Ще одним представленням мультипольного розкладу є розклад по сферичним гармонікам $Y_{lm}(\theta, \phi)$ (те ж саме, але не в декартових, а в сферичних координатах): 
%\begin{equation}\label{eq:multipole2}
%    \pot(\vect{r}) = \sum\limits_{l = 0}^{\infty}\sum\limits_{m=-l}^l \frac{4\pi}{2l + 1} q_lm \frac{Y_{lm}(\theta, \phi)}{r^{l+1}}.
%\end{equation}
%де коефіцієнти $q_lm$, як і в попередньому випадку називаються мультипольними моментами, які визначаються як:
%\begin{equation}\label{eq:2^l-pole}
%    q_{lm} = \int Y_{lm}^*(\theta', \phi') \rho(\vect{r}')dV'.
%\end{equation}
%З цієї формули видно, що в залежності від $l$, ці моменти називаються $2^l$-польними моментами. Так для $l = 0$~--- матимемо $2^0$-польний момент, або прямою підставковою видно, що це заряд тіла. При $l = 1$, матимемо дипольний ($2^1$-польний) момент, при $l = 2$~--- квадрупольний ($2^2$-польний) момент. Однак, так визначені $2^l$-польні моменти дещо відрізняються від таких, що були отримані в декартовій системі.

Однак на практиці в електростатиці постають задачі, в яких зазвичай цікаво не розташування об'ємних зарядів (припустимо що вони відсутні), а задано розташування та форма провідників в просторі і відомі а) потенціали всіх провідників, б) або їх повні заряди і  необхідно визначити напруженість електричного поля у всіх точках простору і розподіл зарядів по поверхнях провідників. В такому випадку розв'язується рівняння Лапласа з відповідними граничними умовами, причому використовуючи теорему єдиності, показується, що розв'язок цього рівняння буде єдиним. Тобто, тепер зад ача вже не зводиться до взяття інтегралу~\eqref{eq:IntSolution}. 

\end{document} 
