% !TeX program = lualatex
% !TeX encoding = utf8
% !TeX spellcheck = uk_UA


\documentclass[]{ProblemBook}

\def\Dolgoshey{D\raisebox{-0.5ex}{o}\raisebox{0.5ex}{l}}

\begin{document}

%=========================================================
\begin{problem}
    В порожнині незаземленого і незарядженого металевого провідника довільної форми знаходиться електричний диполь. Порожнина також має довільну форму. Знайдіть електричне поле за межами провідника.
\end{problem}

\begin{problem}
Електричне поле утворене розподілом зарядів $\sigma(\theta, \phi) = \sigma_0\cos\theta$ на поверхні сфери радіуса $R$, всередині і зовні зарядів немає. Знайти потенціал зовні та усередині кулі, в сферичних координатах.
\begin{solution}

Для обчислення скористаємось інтегралом $\pot = \int\frac{\sigma'dS'}{|\vect{r} - \vect{r}'|}$, для цього скористаємось мультипольним розкладом підінтегрального виразу:
\[
    \frac{1}{|\vect{r} - \vect{r}'|} = \sum\limits_{l=0}^{\infty}\sum\limits_{m=-l}^l \frac{4\pi}{2l + 1} \frac{r_<^l}{r_>^{l+1}} Y_{lm}(\theta, \phi) Y_{lm}^*(\theta', \phi').
\]

Також пам'ятатимемо, що 
\[
    \int Y^*_{lm}(\theta', \phi')Y_{l'm'}(\theta', \phi')d\Omega' = \delta_{ll'}\delta_{mm'},
\]
а також $dS' = R^2\sin\theta'd\theta'd\phi' = R^2d\Omega'$.

Отже, записуємо:

\begin{multline}
    \pot = \sigma_0R^2  \sum\limits_{l=0}^{\infty}\sum\limits_{m=-l}^l \frac{4\pi}{2l + 1} \frac{r_<^l}{r_>^{l+1}} Y_{lm}(\theta, \phi) \int Y_{lm}^*(\theta', \phi') \overbrace{\cos\theta'}^{\sqrt{\frac{4\pi}{3}}Y_{10}} d\Omega' = \\
   = \sigma_0R^2  \sum\limits_{l=0}^{\infty}\sum\limits_{m=-l}^l \frac{4\pi}{2l + 1} \frac{r_<^l}{r_>^{l+1}} Y_{lm}(\theta, \phi) \cdot \sqrt{\frac{4\pi}{3}} \underbrace{\int Y_{lm}^*(\theta', \phi')  Y_{10} d\Omega'}_{\delta_{l1}\delta_{m0}} = \\
      = \sigma_0R^2  \sum\limits_{l=0}^{\infty}\sum\limits_{m=-l}^l \delta_{l1}\delta_{m0} \frac{4\pi}{2l + 1} \frac{r_<^l}{r_>^{l+1}} Y_{lm}(\theta, \phi) \cdot \sqrt{\frac{4\pi}{3}} = \\
       =  \sigma_0R^2 \frac{r_<}{r_>^2}  \frac{4\pi}{3} \sqrt{\frac{4\pi}{3}} Y_{10}(\theta, \phi) = \sigma_0R^2 \frac{r_<}{r_>^2}  \frac{4\pi}{3} \cos\theta.
\end{multline}
Звідки, для зовнішньої частини сфери $r_< = R, r_> = r$, для внутрішньої $r_< = r, r_> = R$, остаточно:
\[
    \pot(\theta) = 
\begin{cases}
     \frac{4\pi\sigma_0 R^3}{3r^3}\cos\theta, \quad r \ge R, \\
    4\pi\sigma_0 r \cos\theta, \quad r < R.
\end{cases}
\]

\end{solution}
\end{problem}
\end{document}
