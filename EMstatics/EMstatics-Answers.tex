\protect \section *{\nameref *{EMstatics}}
\begin{Solution}{2.{7}}
	\begin{enumerate*}[label=\alph*)]
		\item $\rho = q\delta(\vect{r})$;
		\item $\rho = \frac{q}{2\pi R^2}\delta(r - R)\delta(\cos\theta)$;
		\item $\rho = \frac{\lambda(\phi)}{R}\delta(r - R)\delta(\cos\theta)$;
		\item $\rho = \frac{q}{4\pi R^2}\delta (r - R)$;
		\item $\rho = \sigma (\theta)\delta (r - R)$.
	\end{enumerate*}
\end{Solution}
\begin{Solution}{2.{10}}
    Тілесний кут, під яким видно поверхню $S$ з точки $r$:
    \[
        \Omega(\vect{r}) = \int\limits_S d\vect{S}'\frac{\vect{r}' - \vect{r}}{|\vect{r}' - \vect{r}|^3}.
    \]
    Обчислимо зміну тілесного кута при зсуві точки спостереження на малий вектор $\delta\vect{r}$   ($\vect{r} \to \vect{r} + \delta\vect{r}$, контур залишається нерухомим). Це можна зробити, залишаючи, навпаки, точку $\vect{r}$  нерухомою, але зсуваючи поверхню $S$  на $-\delta\vect{r}$, тобто у протилежному напрямку. При цьому зміна $\Omega$ відбувається за рахунок тонкої смужки, яка відповідає переміщенню контура $L$ , який є границею $S$ ($\delta d{\vect{S}} =  - \delta\vect{r} \times d{\vect{l}'}$):
    \[
        \delta\Omega(\vect{r}) = \delta\vect{r}\times \vect{\nabla}\Omega = -\oint \delta\vect{r}\times d\vect{l}' \cdot \frac{\vect{r}' - \vect{r}}{|\vect{r}' - \vect{r}|^3} = - \oint  d\vect{l}' \times \cdot \frac{\vect{r}' - \vect{r}}{|\vect{r}' - \vect{r}|^3} \delta\vect{r}.
    \]
    Звідси виходячи з властивості тілесного кута, під яким видно замкнений контур $L$:
	\[
		\vect{\nabla}\Omega = - \oint  d\vect{l}' \times \cdot \frac{\vect{r}' - \vect{r}}{|\vect{r}' - \vect{r}|^3} = \oint  d\vect{l}' \times  \frac{\vect{r} - \vect{r}'}{|\vect{r} - \vect{r}'|^3}.
	\]
    З іншого боку, індукція в точці $\vect{r}$, створювана струмом в контурі $L$, є
	\[
		\Bfield = \frac{I}{c} \oint  d\vect{l}' \times  \frac{\vect{r} - \vect{r}'}{|\vect{r} - \vect{r}'|^3} = \frac{I}{c} \vect{\nabla}\Omega.
	\]
\end{Solution}
\begin{Solution}{2.{14}}
	$\vect{B}(r) = \frac{4\pi}{cr^2}\left[ \vect{e}_0 \times \vect{r} \right]\int\limits_0^r j(r)r dr$.
\end{Solution}
\begin{Solution}{2.{24}}
    Розв'яжемо задачу за допомогою рівняння Лапласа в декартовій системі координат  та граничних умов на площині $z = 0$. Умова~\eqref{EtauBc} виконується за неперервності потенціалу
	\begin{equation*}\label{continuity}
		\left.\pot(x,y,z)\right|_{z \to 0^+} = \left.\pot(x,y,z)\right|_{z \to 0^-},
	\end{equation*}
	а умова~\eqref{EnBc} дає
	\begin{equation*}\label{boundary}
		-\left. \frac{\partial \pot(x,y,z)}{dz}\right|_{z \to 0^+} + \left. \frac{\partial \pot(x,y,z)}{dz}\right|_{z \to 0^-} =4\pi\sigma(x,y).
	\end{equation*}

	Обмежимось випадком а).

	Оскільки поверхнева густина заряду в граничних умовах має вигляд добутку $\sigma(x, y) = \sigma_0 \sin\left( \frac{x}{L}\right) \cos\left( \frac{2y}{L}\right) $ функцій від $x$ та $y$, можна застосувати метод розділення змінних, покладаючи:
    \[
        \pot(x,y,z) = X(x)Y(y)Z(z).
    \]
Більше того, тут має сенс <<вгадати>> $X(x) = \sin\left(\frac{x}{L}\right) $, $Y(y) =
\cos\left(\frac{2y}{L}\right) $, щоб відповідні функції можна було скомпенсувати в граничних умовах.
Розв'язок має бути вибрано так, щоб виконувалося рівняння Лапласа та граничні умови~\eqref{EtauBc},
\eqref{EnBc}.%, це однозначно фіксує розв'язок в силу теореми єдиності.
Підстановка в рівняння Лапласа дає рівняння на $Z(z)$:

\begin{equation*}
    \frac{d^2Z}{dz^2} - \left( \frac{5}{L^2}\right) Z = 0.
\end{equation*}
Площина  $z = 0$ у середньому є електронейтральною, відповідно, далеко від неї потенціал має прямувати до нуля. Відповідно до цього, при $z >0$  відкидаємо експоненційно зростаючий розв’язок останнього рівняння, залишаючи $ Z(z) = C_1  e^{-\frac{\sqrt5}{L}z} $. Аналогічно при $z < 0$  маємо $ Z(z) = C_2  e^{\frac{\sqrt5}{L}z} $. З умови неперервності потенціалу при $Z = 0$, дістаємо $C_1 = C_2$ . Тепер потенціал визначено з точністю до сталого множника


	\begin{equation*}
		\pot (x,y,z) = C \sin\left( \frac{x}{L} \right)   \cos\left( \frac{2y}{L} \right)  \cdot e^{-\gamma |z|}.
	\end{equation*}

З граничної умови \eqref{EnBc} визначаємо $C$.

Відповідь а):

	\begin{equation*}
		\pot = \frac{2\pi\sigma_0 L}{\sqrt{5}}e^{-\frac{\sqrt{5}}{L}|z|}\sin\left( \frac{x}{L}\right) \cos\left( \frac{2y}{L}\right) .
	\end{equation*}
\end{Solution}
\begin{Solution}{2.{25}}
	$\pot  = \frac{{2{\sigma _0}ab}}{{\sqrt {{a^2} + {b^2}} }}{e^{ - \alpha |z|}}\sin \left( {\frac{{\pi x}}{a}} \right)\sin \left( {\frac{{\pi y}}{b}} \right)$, де $\alpha  = \frac{\pi }{{ab}}\sqrt {{a^2} + {b^2}}$.

    Зауважимо, що, енергію системи зручно за обчислювати за допомогою формули~\eqref{eq:We}
	$W_e =  \frac{{\sigma _0^2{a^2}{b^2}}}{{\sqrt {{a^2} + {b^2}} }}$.
\end{Solution}
\begin{Solution}{2.{27}}
	$
		\pot(r) =
		\begin{cases}
			4\pi a\ln\frac{R_2}{R_1},                                                  & r \le R_1         \\
			4\pi a \left[ \left( 1- \frac{R_1}{r}\right)  + \ln\frac{R_2}{r} \right] , & R_1 \le r \le R_2 \\
			4\pi a \frac{R_2 - R_1}{r},                                                & r \ge R_2
		\end{cases}
	$
\end{Solution}
\begin{Solution}{2.{28}}
	Розпишемо потенціал як:
    \[
	\pot = \frac{q}{r}e^{-\frac{r}{a}} = \frac{q}{r} +  \frac{q}{r}\left(e^{-\frac{r}{a}} - 1\right) .
    \]
    Подіємо оператором Лапласа на цей потенціал. Перший доданок відповідає полю точкового заряду з  $\delta$-видним розподілом густини. Другий доданок обчислюємо в сферичних координатах:
    \[
        \Delta \left[\frac{q}{r}\left(e^{-\frac{r}{a}} - 1\right) \right] =
    \frac{q}{r^2}\frac{\partial}{\partial r}\left[ r^2\left( \frac{\partial }{\partial r}\frac{e^{-r/a} - 1}{r} \right) \right] = \frac{qe^{-r/a}}{a^2r}.
    \]
	Співставляючи з рівнянням Пуассона, знайдемо розподіл заряду:
	\[
		\rho = q\delta(\vect{r}) - \frac{q}{4\pi a^2} \frac{e^{-\frac{r}{a}}}{r},
	\]
	звідки видно, що екранований кулонівський потенціал створюється точковим зарядом, навколо якого розподілена <<хмара>> електричного заряду. Такий характер розподілу зустрічається, наприклад, в електролітах, або плазмі.
\end{Solution}
\begin{Solution}{2.{29}}
	Оскільки ми маємо азимутальну симетрію в розподілі заряду, будемо шукати потенціал у вигляді:
	\begin{equation}
		\pot(r, \theta) = \begin{cases}
			\sum\limits_{l=0}^{\infty}A_l \left( \frac{r}{R}\right)^l P_l(\cos\theta), \quad r<R \\[1.5em]
			\sum\limits_{l=0}^{\infty}B_l \left( \frac{R}{l}\right)^{l+1} P_l(\cos\theta), \quad r \ge R.
		\end{cases}
	\end{equation}

	Випадок (а)

	З урахуванням граничних умов на сфері:
	\begin{equation}
		\pot = \begin{cases}
			\pot_0\frac{r}{R} \cos\theta, \quad r<R \\
			\pot_0\left( \frac{R}{r}\right)^2 \cos\theta, \quad r \ge R.
		\end{cases}
	\end{equation}
	Поверхневу густину заряду на поверхні знайдемо з граничної умови:
	\begin{equation}
		\sigma = \frac{1}{4\pi} \left( \left. \frac{\partial \pot_{\mathrm{in}}}{\partial r}\right|_{r \to R} - \left. \frac{\partial \pot_{\mathrm{out}}}{\partial r}\right|_{r \to R} \right) =  \frac{3\pot_0}{4\pi R} \cos\theta.
	\end{equation}
\end{Solution}
\begin{Solution}{2.{30}}
	\begin{enumerate}[label=\alph*)]
		\item%
		      \begin{equation}
			      \pot = \begin{cases}
				      \frac43\pi\sigma_0r \cos\theta, \quad r<R \\[1em]
				      \frac{\frac43\pi\sigma_0R^3}{r^2} \cos\theta, \quad r \ge R.
			      \end{cases}
		      \end{equation}
		\item%
		      \begin{equation}
			      \pot = \begin{cases}
				      \frac43\pi\sigma_0r \cos\theta + 4\pi\sigma_1 R, \quad r<R \\[1em]
				      \frac{\frac43\pi\sigma_0R^3}{r^2} \cos\theta + \frac{4\pi\sigma_1 R^2}{r}, \quad r \ge R.
			      \end{cases}
		      \end{equation}
	\end{enumerate}
\end{Solution}
\begin{Solution}{2.{32}}
	$\pot_\text{ind}(\vect{r}) =  - \frac{Q}{r}\sum\limits_{n = 1}^N \sum\limits_{m =  - n}^n {C_{nm}{\left( \frac{R}{r} \right)^n}} Y_{nm}(\theta ,\varphi )$
\end{Solution}
\begin{Solution}{2.{36}}
	$\pot (r) =
		\begin{cases}
			\frac{{2\pi }}{{2n + 3}}{\rho _0}{R^2}\left[ {\frac{{2n + 3}}{{2n + 1}}{{\left( {\frac{r}{R}} \right)}^n} - {{\left( {\frac{r}{R}} \right)}^{n + 2}}} \right]{P_n}(\cos (\theta )) , \quad r<R \\
			\frac{{4\pi {\rho _0}{R^2}}}{{(2n + 3)(2n + 1)}}{\left( {\frac{R}{r}} \right)^{n + 1}}{P_n}(\cos (\theta )) , \quad r>R
		\end{cases}$
\end{Solution}
\begin{Solution}{2.{40}}
	$
		\pot(r, \phi, z) =
		\begin{cases}
			\frac{\pi }{2}{\sigma _0}R\left( {3\sin (\varphi )\frac{r}{R} - \frac{1}{3}\sin (3\varphi ){{\left( {\frac{r}{R}} \right)}^3}} \right), \quad r < R , \\
			\frac{\pi }{2}{\sigma _0}R\left( {3\sin (\varphi )\frac{R}{r} - \frac{1}{3}\sin (3\varphi ){{\left( {\frac{R}{r}} \right)}^3}} \right) , \quad r > R .
		\end{cases}
	$
\end{Solution}
\begin{Solution}{2.{46}}
	Спосіб 1 (використання закону Біо-Савара-Лапласа).
	\[
		\Bfield(\vect{r}) = \frac1c \int dV' \vect{j} \times \frac{(\vect{r} - \vect{r}')}{|\vect{r} - \vect{r}'|^3}
	\]

	Підставимо вираз густини струму з умови задачі, і будемо шукати поле в точці $\vect{r} = 0$:
	\begin{multline}\label{B(0)_prb:BSL}
		\Bfield(0) = -\frac1c \int dV' \left( c\vect{\nabla}\times \left( \rho(r) \vect{a}\right) \right) \times \frac{\vect{r}'}{|\vect{r}'|^3} = \\
		= -\int \left[ \vect{\nabla}\rho(r)\times\vect{a} \right] \times \frac{\vect{r}'}{|\vect{r}'|^3} dV' = \\
		= -\int\limits_0^{\infty} dr'\frac{d\rho}{dr'} \int d\Omega' [{\vect{n}}' \times [{\vect{a}} \times {\vect{n}}']],
	\end{multline}
	де $\vect{n}' = \frac{\vect{r}'}{r'}$. Спрямуємо вісь $z$ вздовж вектора  $\vect{a} = a\vect{e}_z$. Використовуючи <<BAC - CAB>>
	\[
		\int {d\Omega'} [{\vect{n}}' \times [{\vect{a}} \times {\vect{n}}']] = \int {d\Omega'} [\vect{a} - \vect{n}'(\vect{a} \cdot {\vect{n}}')],
	\]
	можна бачити (з міркувань симетрії), що
%    $\int d\Omega n_x n_z = \int d\Omega n_н n_z $, а отже,
    інтеграл  $\int {d\Omega'} [{\vect{n}}' \times [{\vect{a}} \times {\vect{n}}']] $ має напрямок вздовж $z$, тому, його легко обчислити:
	\begin{multline}
		\int d\Omega' [{\vect{n}}' \times [{\vect{a}} \times {\vect{n}}']] = {\vect{a}}\int {d\Omega'} \left( {[{\vect{n}}' \times [{{\vect{e}}_z} \times {\vect{n}}']] \cdot {{\vect{e}}_z}} \right) = \\
		= \vect{a}\int\limits_0^\pi  {d\theta } \sin \theta \int\limits_0^{2\pi } {d\varphi } \left[ {1 - {\cos^2}\theta } \right] = \vect{a}\frac{8\pi}{3}.
	\end{multline}
	Інтеграл $\int\limits_0^{\infty} dr'\frac{d\rho}{dr'} = -\rho(0)$.
	Підставляючи все в~\eqref{B(0)_prb:BSL}, маємо:
	\[
		\Bfield(0) = \vect{a}\frac{8\pi}{3}\rho(0).
	\]

	Спосіб 2 (використання рівняння Максвелла).

	Використаємо безпосередньо рівняння Максвелла:
	\[
		\vect{\nabla}\times\Bfield = \frac{4\pi}{c}\vect{j}
	\]

	і підставимо вираз густини струму з умови задачі:
	\[
		\vect{\nabla}\times\Bfield = 4\pi\vect{\nabla}\times \left( \rho(r) \vect{a}\right).
	\]

	Якщо ротори лівої і правої частин рівняння однакові, то вони можуть відрізнятись на градієнт довільної скалярної функції $\psi$, отже
	\begin{equation}\label{B(0)_prb:ME}
		\Bfield(r) = 4\pi \left( \rho(r) \vect{a}\right) + \vect{\nabla}\psi.
	\end{equation}
	Причому, оскільки $\rho(r)$   і $\Bfield(r)$  спадають до нуля на нескінченності,  то $\psi$  також спадає до нуля на нескінченності.

	З іншого рівняння Максвела $\vect{\nabla} \cdot \Bfield = 0$ маємо,
	\[
		4\pi \vect{\nabla} \cdot \left( {\rho (r){\vect{a}}} \right) + \Delta \psi  = 0,
	\]
	звідки отримуємо рівняння Пуассона для $\psi$:
	\[
		\Delta \psi = - 4\pi \vect{\nabla} \cdot \left( {\rho (r){\vect{a}}} \right) = - 4\pi\frac{d\rho}{dr} \left( \vect{n} \cdot \vect{a} \right).
	\]

	Оскільки $\psi$ спадає на нескінченності, то рівняння має єдиний розв'язок:
	\[
		\psi (\vect{r}) = \int dV' \frac{d\rho}{dr'}\frac{(\vect{n}'\cdot \vect{a})}{|\vect{r} - \vect{r}'|}.
	\]
	Тепер знайдемо градієнт $\psi$:
	\[
		\vect{\nabla}\psi = - \int {dV'} \frac{d\rho}{dr'}\frac{(\vect{n}' \cdot \vect{a})}{|\vect{r} - \vect{r}'|^3}\left( {\vect{r} - \vect{r}'} \right),
	\]
	і в нулі
	\[
		\nabla \psi (0) = \int dV' \frac{d\rho}{dr'}\frac{(\vect{n}' \cdot \vect{a})}{|\vect{r}'|^3} \vect{r}'= \int {d\Omega} \int\limits_0^\infty  dr \frac{d\rho}{dr}\frac{(\vect{n} \cdot \vect{a})}{r^2}{\vect{n}}.
	\]

	Далі аналогічно до попереднього способу проводимо обчислення інтегралів, а тому матимемо:
	\[
		\nabla \psi (0) = -\frac{4\pi}{3}\vect{a}\rho(o).
	\]
	Підставимо тепер цей вираз в формулу~\eqref{B(0)_prb:ME}, остаточно отримуємо
	\[
		\Bfield(0) = \vect{a}\frac{8\pi}{3}\rho(0),
	\]
	що збігається з відповіддю, отриманою в попередньому способі.
\end{Solution}
\begin{Solution}{2.{51}}
	Проведемо вісь $OZ$ вздовж поля $\Bfield = (0, 0, B)$.
	\begin{enumerate*}[label=\alph*)]
		\item $A_x = -\frac12 By$, $A_y = \frac12 Bx$, $A_z = 0$;
		\item $A_{\phi} = \frac12 Br$, $A_r = A_z = 0$;
		\item $A_{\phi} = \frac12 Br\sin\theta$, $A_r = A_{\theta} = 0$,
	\end{enumerate*}
або у векторній формі 	$\vect{A} = \frac12 \left[ \Bfield\times \vect{r} \right] $.
	З огляду калібрувальної інваріантності вектор-потенціал не визначений однозначно. Тут наведені найбільш зручні представлення вектор-потенціалу однорідного поля в різних системах координат.
\end{Solution}
\begin{Solution}{2.{52}}
	Потенціал кулі є квадрупольним, $\pot(z) = \frac{\pi^2\rho_0 R^5}{48z^3}$.
\end{Solution}
\begin{Solution}{2.{55}}
	$\Bfield =
		\begin{cases}
			\frac{2q}{3c}\frac1R\vect{\omega}, \quad r < R \\
			\frac{3(\vect{p}_m \cdot \vect{r})\vect{r}}{r^5} - \frac{\vect{p}_m}{r^3}, \quad r > R
		\end{cases}
	$
	де $\vect{p}_m = \frac{qR^2}{3c}\vect{\omega}$~--- магнітний момент сфери.
\end{Solution}
\begin{Solution}{2.{56}}
	$B_z = \frac{324\pi a}{5c}j_0$.
\end{Solution}
\begin{Solution}{2.{58}}
    $W = \sum\limits_{l=0}^{\infty}\sum\limits_{m=-l}^{l}a_{lm}Q_{lm}^*$.
\end{Solution}
\begin{Solution}{2.{59}}
    \begin{multline*}
    	W = -(\vect{p}_1\cdot \Efield_2) = -\frac{3(\vect{p}_1\cdot\vect{r})(\vect{p}_2\cdot\vect{r})}{r^5}  + \frac{(\vect{p}_1\cdot\vect{p}_2)}{r^3} =\\ = -\frac{p_1p_2}{d^3}\left[ 3\cos\theta_1\cos\theta_2 - \cos(\theta_1 - \theta_2)\right] ,
    \end{multline*}
\end{Solution}
\begin{Solution}{2.{60}}
	$F = \frac{6R^6}{l^4} E_0^2$.
\end{Solution}
