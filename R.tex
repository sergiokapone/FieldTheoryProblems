% !TeX program = lualatex
% !TeX encoding = utf8
% !TeX spellcheck = uk_UA


\documentclass[]{ProblemBook}



\begin{document}

%=========================================================
\begin{problem}
    Обчисліть активний опір на одиницю довжини циліндричного провідника радіуса $a$ провідністю $\lambda$. Отримайте значення цієї величини за малих та за великих частот
\begin{solution}


Густина струму 
\[
	j_z = \frac{k}{2\pi a} \frac{J_0(kr)}{J_1(ka)} I_0e^{-i\omega t}.
\]

Опір будемо шукати як $R = \frac{\left\langle Q \right\rangle }{\frac12I_0^2}$.


Із закону Джоуля-Ленца розрахуємо теплоту, яка виділяється в провіднику як%
\footnote{Використано співвідношення \href{https://www.wolframalpha.com/input/?i=integrate+from+0+to+a+besselJ0\%28k*x\%29*besselJ0\%28Conjugate\%28k\%29*x\%29+x+dx}{$\int\limits_0^a  J_0(kr)J_0(k^*r) r dr = \frac{kJ_1(ka)J_0(k^*a) - k^*J_1(k^*a)J_0(ka)}{k^2-k^{*2}} $} (в додатках нема)}

\begin{multline*}
    \left\langle Q \right\rangle = \frac1{2\lambda}\int\limits_0^a \left| j_z^2\right| 2\pi r dr = \frac{2\pi}{2(2\pi a)^2} \frac{I_0^2}{\lambda} \frac{|k|^2}{J_1(ka)J_1(k^*a)}  \int\limits_0^a  J_0(kr)J_0(k^*r) r dr = \\
    = \frac{2\pi a}{2(2\pi a)^2} \frac{I_0^2}{\lambda} \frac{|k|^2}{J_1(ka)J_1(k^*a)} \frac{kJ_1(ka)J_0(k^*a) - k^*J_1(k^*a)J_0(ka)}{k^2-k^{*2}} =\\
    = \frac{2\pi a }{2(2\pi a)^2} \frac{I_0^2}{\lambda} \frac{|k|^2\delta^2}{4i}\left( \frac{kJ_0(k^*a)}{J_1(k^*a)} - \frac{k^*J_0(ka)}{J_1(ka)}\right) = \\ = \frac{2\pi a}{2(2\pi a)^2} \frac{I_0^2}{\lambda} \frac12  \left( \frac{k^*J_0(k^*a)}{J_1(k^*a)} + \frac{kJ_0(ka)}{J_1(ka)}\right) =  \frac{1}{4\pi a} \frac{I_0^2}{\lambda}  \mathrm{Re}\left(  \frac{kJ_0(ka)}{J_1(ka)} \right).
\end{multline*}

Отже 
\[
    R = \frac{1}{2\pi a\lambda}  \mathrm{Re}\left[  \frac{kJ_0(ka)}{J_1(ka)} \right].
\]

Розглянемо випадок малих частот $|ka| \ll 1$:
\[
    R \approx \frac{1}{2\pi a\lambda}  \mathrm{Re} \left[k\left( 1 - \frac{k^2a^2}{4}\right)\frac{2}{ka}  \right] = \frac{1}{\pi a^2 \lambda}.
\]
Тобто, при малих частотах опір такий же, як і для для постійного струму.

Розглянемо випадок великих частот $|ka| \gg 1$:
\[
    R \approx \frac{1}{2\pi a\lambda} \mathrm{Re}\left[  ke^{k(r-a)} \right] \approx \frac{1}{2\pi a\lambda} \mathrm{Re}\left[  k(1-k(r-a)) \right] =  \frac{1}{2\pi a\delta\lambda}.
\]
З останньої формули випливає, що ефективна площа перерізу провідника в області великих частот дорівнює $2\pi a\delta$, тобто має дуже маленьку величину, зосереджену поблизу поверхні провідника.
\end{solution}
\end{problem}

\end{document}
