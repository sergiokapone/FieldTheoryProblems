% !TeX program = lualatex
% !TeX encoding = utf8
% !TeX spellcheck = uk_UA
% !TeX root =../EMProblems.tex

\introtrue
\chapter*{Передмова}

Вивчення класичної електродинаміки забезпечується дисциплінами <<Тео\-рія поля>> та <<Електродинаміка суцільних середовищ>>, які входять до циклу базової та професійної підготовки бакалаврів за спеціальностями 105 <<Прикладна фізика та наноматеріали>> і 113 <<Прикладна математика>> й вивчаються студентами Фізико-технічного інституту КПІ ім. Ігоря Сікорського на третьому курсі. На вивчення дисциплін відведено два семестри, практичні заняття проводяться один раз на тиждень. Велика частина програмного матеріалу пов'язана з умінням розв'язування конкретних задач. Вироблення умінь, навичок і методів розв'язку величезної кількості задач, звичайно, не може бути реалізоване тільки за рахунок годин, відведених на практичні заняття, і передбачає значну самостійну роботу студента.

Назви розділів та підрозділів посібника відповідають робочим програмам дисциплін. До кожного з них подано короткий теоретичний матеріал, який містить основні формули, необхідні для розв'язування задач. В кінці збірника міститься довідковий матеріал та перелік використаної літератури. 


Задачі даного збірника щорічно пропонуються студентам при складанні письмового екзамену, який передбачено навчальним планом. Зрозуміло, що багато задач було взято з джерел, які перелічені в списку літератури, оскільки методика викладання дисциплін протягом  століть вже є переважно завершеною і багато вже винайдених іншими вченими задач є ключовими для розуміння предмета. Однак в збірнику є й велика частина авторських задач, ідеї яких сформувались на основі власного досвіду викладання предмету.

Як правило, в тексті використовуються такі позначення координат: $x$,$y$,$z$ для декартових координат;  $r$, $\theta$, $\phi$ для сферичних та  $r$, $\phi$, $z$ для циліндричних (в останніх позначеннях використано $r=\sqrt{x^2+y^2}$,  щоб уникнути плутанини з густиною заряду).
 
Дана версія посібника є електронним виданням, тому для зручності користування ним передбачена система навігації у вигляді гіперпосилань та бічної панелі змісту.

%\vspace*{4em}
%
%\begin{flushright}\Annabelle
%	С.~М.~Пономаренко
%\end{flushright}
\introfalse





