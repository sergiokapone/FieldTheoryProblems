% !TeX program = lualatex
% !TeX encoding = utf8
% !TeX spellcheck = uk_UA
% !BIB program = biber

\documentclass[%
biblatex,
%marginversioninfo
]{ProblemBook}

%========================================================================================================
%
%									      Вихідні дані докумета
%
%========================================================================================================

\title{Класична електродинаміка}
\def\subtitle{Збірник задач}


%======================================== Бібліографія ===================================================
%
\addbibresource{d:/Projects/LaTeX/MyPackages/bib/MyBase.bib}
%\addbibresource{Kvant.bib}
%
%========================================================================================================

%========================================================================================================
%
% !TeX root =./FTProblems.tex

%========================================================================================================
%
%									      Палітурка
%
%========================================================================================================

\newcommand{\CoverPage}{
	\begin{alwayssingle}
		\begin{center}
			\begin{flushright}\bfseries\sffamily
				\MakeUppercase{Міністерство освіти і науки України}\\
				КПІ ім. Ігоря Сікорського\\
			\end{flushright}
			\begin{tcolorbox}[titlepagestyle,
					toprule=0.10cm,
					bottomrule=0.10cm,
					overlay={%
						\node (picture) at ([xshift=4cm]frame.west) {\includegraphics[width=3.5cm]{logo_IPT}};
					}
			]%
			\begin{flushright}
				\large\bfseries\color{white}Фізико-технічний інститут
			\end{flushright}
			\end{tcolorbox}
			\vspace*{-2em}
		 	\begin{Large}\color{themecolordark!90!black}
			\begin{alignat*}{3}
				\vect{\nabla}&\times&&\Efield &&= -\dfrac{1}{c}\dfrac{\partial\Bfield}{\partial t} \\
				\vect{\nabla}&\,\cdot&&\Bfield &&= 0 \\
				\vect{\nabla}&\,\cdot&&\Dfield &&= 4\pi\rho \\
				\vect{\nabla}&\times&&\Hfield &&= \dfrac{4\pi}{c} \vect{j}+\dfrac{1}{c}\dfrac{\partial\Dfield}{\partial t}
			\end{alignat*}
			\end{Large}
			\vspace*{2em}
			\begin{tcolorbox}[
				titlepagestyle,
				toprule=0.15cm,
				bottomrule=0.15cm,
				top=1.3cm,
				bottom=0.7cm,
				overlay={%
				\node[%
							fill=white,
							rounded corners = 15pt,
							draw=themecolorlight,
							line width=0.15cm,
							inner sep=0pt,
							text width=17cm,
							minimum height=2cm,
							align=center,
							%anchor=east,
							font=\sffamily\bfseries\large
						] (title) at (frame.north) {В.~І. Жданов, %
													С.~М. Пономаренко,
													В.~Б. Долгошей%
					};
				}
			]
			\centering
			\Huge\sffamily\bfseries\textcolor{white}{\realtitle}\\
			\huge\sffamily\bfseries\textcolor{white}{\subtitle}
			\end{tcolorbox}
			\vfill
		 	\begin{Large}\color{themecolordark!90!black}
			\begin{gather*}
				\frac{\partial F^{\mu\nu}}{\partial x^\nu} =                                                                                -\frac{4\pi}{c}j^\mu \\
				\frac{\partial F_{\beta\gamma}}{\partial x^\alpha} + \frac{\partial F_{\gamma\alpha}}{\partial x^\beta} +\frac{\partial
				F_{\alpha\beta}}{\partial x^\gamma}  = 0.
			\end{gather*}
			\end{Large}
			\vfill
			\begin{tcolorbox}[titlepagestyle,
					toprule=0.10cm,
					bottomrule=0.10cm]
				\begin{center}\color{white}\bfseries\normalsize
					\MakeUppercase{Київ~\the\year} \\
%					КПІ ім. Ігоря Сікорського \\
%					\the\year
				\end{center}
			\end{tcolorbox}
		\end{center}
		\clearpage
	\end{alwayssingle}
\setcounter{page}{1}
}


%========================================================================================================
%
%									      Титульна сторінка
%
%========================================================================================================

\renewcommand\maketitle{
	\begin{alwayssingle}
		\begin{center}
				\MakeUppercase{Міністерство освіти і науки України}

				\bigskip
				\MakeUppercase{Національний технічний університет України}\\
				<<КИЇВСЬКИЙ ПОЛІТЕХНІЧНИЙ ІНСТИТУТ \\ імені ІГОРЯ СІКОРСЬКОГО>>
				\vspace*{100pt}

				{\large В.~І.~Жданов,
						С.~М.~Пономаренко,
						В.~Б.~Долгошей
						}
				\vspace*{50pt}

				{\Huge\sffamily\bfseries\realtitle}\\[1em]
				{\huge\sffamily\bfseries\subtitle}

			\vspace*{50pt}
			\begin{center}\itshape
			Рекомендовано Методичною радою КПІ ім. Ігоря Сікорського як навчальний посібник для
			здобувачів ступеня бакалавра за спеціальностями E6 <<Прикладна фізика та наноматеріали>>,
			F1 <<Прикладна математика>>
			\end{center}
            \vfill
                \textcolor{malina}{Версія від~\today, час компіляції \currenttime.}
			\vfill
			\begin{center}
				\MakeUppercase{Київ} \\
				КПІ ім. Ігоря Сікорського \\
				\the\year
			\end{center}
		\end{center}
		\clearpage
	\end{alwayssingle}
}


%========================================================================================================
%
%									      Друга сторінка
%
%========================================================================================================
\newcommand\makeinfopage{
	\begin{alwayssingle}
		\noindent%
		\begin{minipage}[t]{0.5\textwidth}
				\begin{flushleft}
					УДК  537\\
					Ж 42
				\end{flushleft}
		\end{minipage}


		\bigskip\noindent%
        \begin{minipage}[t]{0.2\linewidth}
            	\begin{flushleft}
                    Рецензенти:
                \end{flushleft}
        \end{minipage}\hfill
        \begin{minipage}[t]{0.78\linewidth}
                \href{http://www.nas.gov.ua/UA/PersonalSite/Pages/default.aspx?PersonID=0000006576}{В.~О.~Кочелап}, д.ф.-м.н., професор,
                член-кореспондент НАН України, завідувач відділу теоретичної фізики Інституту фізики напівпровідників ім. В.~Є. Лашкарьова
%                \\[1ex]
%                \href{http://apd.ipt.kpi.ua/blog/author/19}{Я.~Д.~Кривенко-Еметов}, к.ф.-м.н., доцент кафедри прикладної фізики, Фізико-технічного
%інституту КПІ ім. Ігоря Сікорського
        \end{minipage}

		\bigskip\noindent%
        \begin{minipage}[t]{0.2\linewidth}
            	\begin{flushleft}
                    Відповідальний редактор:
                \end{flushleft}
        \end{minipage}\hfill
        \begin{minipage}[t]{0.78\linewidth}
                \href{http://is.ipt.kpi.ua/is/smirnovsa/}{С.~А.~Смирнов}, к.ф.-м.н., доцент, голова
                методичної ради НН ФТІ КПІ ім. Ігоря Сікорського
        \end{minipage}

		\begin{center}\itshape\small
				Гриф надано Методичною радою КПІ ім. Ігоря Сікорського (протокол №~10/2020~від 18.06.2020~р.) за поданням Вченої ради Фізико-технічного інституту (протокол №~5/2020 від 25.05.2020 р.)
		\end{center}
		\begin{center}
			\ifelectronic Електронне мережне навчальне видання \fi
			%\par {Версія від~\href{http://www.istpravda.com.ua/dates}{\today}} \par\else \par  \fi
		\end{center}
		\begin{center}
			\href{https://apd.ipt.kpi.ua/kafedra/stuff/zhdanov}{\itshape Жданов Валерій Іванович},
			д.ф.-м.н., професор \\
			\href{https://apd.ipt.kpi.ua/kafedra/stuff/ponomarenko}{\itshape Пономаренко Сергій
			Миколайович}, к.ф.-м.н., доцент \\
			\href{https://apd.ipt.kpi.ua/kafedra/stuff/dolgoshey}{\itshape Долгошей Володимир
			Борисович}, к.ф.-м.н., доцент
		\end{center}
%			\vspace*{1em}%
		\begin{center}\bfseries
			\LARGE\sffamily\realtitle \\
			\Large\sffamily\subtitle
		\end{center}
        \noindent%
        \begin{minipage}[t]{\textwidth}\small
                \realtitle: \subtitle\ [Електронний ресурс] : навч. посіб. для студ. спеціальностей
                E6 <<Прикладна фізика та наноматеріали>> та F1 <<Прикладна математика>> /  В.~І.
                Жданов, С.~М. Пономаренко, В.~Б. Долгошей ; КПІ ім. Ігоря Сікорського.~--- Електронні
                текстові дані
            (1 файл: 570~кБ). – Київ : КПІ ім. Ігоря Сікорського, \the\year. --- \the\numexpr\getpagerefnumber{LastPages}-1\relax~с.
        \end{minipage}

%		\noindent%
%		\begin{flushleft}
%			\begin{tabular}{lp{0.9\textwidth}}
%				     & В.~І. Жданов, С.~М. Пономаренко, В.~Б. Долгошей                                                                                                                                                                        \\
%				Ж 42 & \hspace*{3ex} \realtitle : \subtitle{} [Електронний ресурс] : навчальний посібник / %
%				В.~І. Жданов, %
%				С.~М. Пономаренко%,
%				В.~Б. Долгошей %
%				--- К.:~КПІ ім. Ігоря Сікорського, \the\year. --~\the\numexpr\getpagerefnumber{LastPage}-1\relax~с. -- Бібліогр.: с.~\pageref{BibPage}\relax. \ifelectronic\relax\else-- 80~прим.\fi
%			\end{tabular}
%		\end{flushleft}
		\vfill

		Наведено \TotalValue{totalproblems} задач з курсу класичної електродинаміки. До більшості задач подано відповіді. Задачі різноманітні як за змістом, так і за складністю. Поряд із задачами, які ілюструють основні поняття і закони електродинаміки, до збірника включена певна кількість більш складних задач, які допомагають суттєвішому вивченню предмета.

		Для студентів фізико-технічного інституту КПІ ім. Ігоря Сікорського, які навчаються за спеціальностями 105~<<Прикладна фізика та наноматеріали>> та 113 <<Прикладна математика>>.

		\vfill

%		\begin{flushleft}\small
%			Ілюстративний матеріал підручника підготовлений за допомогою пакету \href{http://pgf.sourceforge.net}{TikZ/Pgf}. Верстка тексту проведена в видавничій системі \LaTeXe{} (компілятор Lua\LaTeX) на базі системи комп'ютерної верстки \TeX{} (Збірка  \href{https://www.tug.org/texlive/}{\TeX Live~\the\year}) з використанням оболонки \href{https://www.texstudio.org}{\TeX Studio}.
%		\end{flushleft}
	\hfill
	\begin{minipage}[t]{0.65\linewidth}\small
        \textcopyright{} В.~І. Жданов, С.~М. Пономаренко, В.~Б. Долгошей, \the\year\,р. \\
        \textcopyright{}  КПІ ім. Ігоря Сікорського (ФТІ), \the\year~р.
    \end{minipage}
		\newpage%
	\end{alwayssingle}
}

%%%%%%%%%%%%%%%%%%%%%%%%%%%%%%%%%%%%%%%%%%%%%%%%%%%%%%%%%%%%%%%%%%%%%%%%%%%%
%%                                                                        %%
%%                              Last Page                                 %%
%%                                                                        %%
%%%%%%%%%%%%%%%%%%%%%%%%%%%%%%%%%%%%%%%%%%%%%%%%%%%%%%%%%%%%%%%%%%%%%%%%%%%%
\newcommand{\makelastpage}{%
\clearpage%
\thispagestyle{empty}%
\vspace*{0.4\textheight}
\begin{center}
		\textbf{Жданов} Валерій Іванович \\
	    \textbf{Пономаренко} Сергій Миколайович \\
	    \textbf{Долгошей} Володимир Борисович
\end{center}

\begin{center}\bfseries
    \Large\sffamily\realtitle \\
    \large\sffamily\subtitle
\end{center}

\begin{center}\itshape
    Комп'ютерне верстання в системі \LaTeXe{} С.\ М. Пономаренко
\end{center}

\vspace*{1em}
\begin{center}\small
Національний технічний університет України \\
<<Київський політехнічний інститут імені Ігоря Сікорського>> \\
Свідоцтво про державну реєстрацію: серія ДК № 5354 від 25.05.2017 р.\\
просп. Перемоги, 37, м. Київ, 03056
\end{center}

%\vfill
%
%\begin{center}
%    Видавництво <<Політехніка>> \kpishort \\
%    вул. Політехнічна, 14, корп. 15 \\
%    м. Київ, 03056 \\
%    Тел. (044) 204-81-78
%\end{center}
}


%					      Титульна сторінка
%
%========================================================================================================

\begin{document}
\pagestyle{empty}
%\ifelectronic\CoverPage\setcounter{page}{0}\fi
\CoverPage
\pagenumbering{arabic}
\maketitle
\makeinfopage
%========================================================================================================
%
\clearpage\pagestyle{plain}
\tableofcontents%				               Зміст
%
%========================================================================================================

%========================================================================================================
%
\include{intro}%                          Вступ та передмова
%
%========================================================================================================

%========================================================================================================
%
%									      Вставка файлів розділів
%
%========================================================================================================

\newcommand{\ChaptersOne}{
	EMstatics,
	Radiation,
	SpecialRelativity
}
\newcommand{\ChaptersTwo}{
	EMstaticsInMedia,
	QuasiStationarMedia,
	WavesInLinearMedia
}

\pagestyle{main}
\part{Мікроскопічна теорія}
\epigraph{\Annabelle  Неможливо позбутися відчуття, що ці математичні формули існують незалежно від нас і володіють власним розумом, що вони мудріші за нас, мудріше навіть тих, хто їх відкрив, і що ми дістаємо з них більше, ніж спочатку було закладено\ldots}{Heinrich Hertz}

% !TeX program = lualatex
% !TeX encoding = utf8
% !TeX spellcheck = uk_UA
% !TeX root =../FTProblems.tex

\newpage
\chapter{Система \mbox{мікроскопічних} рівнянь Максвелла}

%\epigraph{\Annabelle  Неможливо позбутися відчуття, що ці математичні формули існують незалежно від нас і володіють власним розумом, що вони мудріші за нас, мудріше навіть тих, хто їх відкрив, і що ми дістаємо з них більше, ніж спочатку було закладено\ldots}{Heinrich Hertz}
%\epigraph{\Annabelle Уж не Боги ли начертали эти знаки?}{Л.~Больцман}

%\renewcommand{\theequation}{\thepart.\arabic{equation}}
\section{Рівняння Максвелла}

Система рівнянь Максвелла сформульована в 60-х роках XIX століття на основі узагальнення емпіричних законів і розвитку ідей про електромагнітні явища. Система рівнянь є основою теорії електромагнітного поля і дозволяє розв'язувати задачі, пов'язані з відшуканням електричних і магнітних полів, що створюються заданим розподілом електричних зарядів і струмів.

Інтегральна форма мікроскопічних рівнянь:
\begin{align}
	\oiint\limits_{\partial\Omega} \Efield\cdot d\vect{S} & = 4\pi\iiint\limits_{\Omega}\rho dV   \label{Int
	I},                                                                                                         \\
	\oiint\limits_{\partial\Omega} \Bfield\cdot d\vect{S} & = 0   \label{Int
	II},                                                                                                                                   \\
	\oint\limits_{\partial S} \Efield\cdot d\vect{r}  & = - \frac1c \iint\limits_S \frac{\partial\Bfield}{\partial t}\cdot d\vect{S}  \label{Int
	III},                                                          \\
	\oint\limits_{\partial S} \Bfield\cdot d\vect{r}  & =\dfrac{4\pi}{c} \iint\limits_S \vect{j}\cdot d\vect{S} +\frac{1}{c} \iint\limits_S
	\frac{\partial\Efield}{\partial t}\cdot d\vect{S}  \label{Int IV},
\end{align}
де $\Omega$ --- довільний нерухомий об'єм, $\partial\Omega$ --- його межа; $S$ --- довільна нерухома орієнтована поверхня, $\partial S$ --- замкнений
контур, що її обмежує.

Диференціальна форма  мікроскопічних рівнянь:
\begin{flalign}
	\divg\Efield &= 4\pi\rho \label{Diff I},\\[0.8em]
	\divg\Bfield &= 0 \label{Diff II},\\
	\rot\Efield &= -\dfrac{1}{c}\dfrac{\partial\Bfield}{\partial t} \label{Diff III},\\
	\rot\Bfield &= \dfrac{4\pi}{c} \vect{j}+\dfrac{1}{c}\dfrac{\partial\Efield}{\partial t} \label{Diff IV}.
\end{flalign}


\noindent%
$ \Efield $~-- вектор напруженості електричного поля,\\
$ \Bfield $~-- вектор індукції магнітного поля.


\section{Закони збереження}

\begin{itemize}
	\item Електричного заряду
	      \begin{equation}
		      \frac{\partial \rho}{\partial t} + \divg\vect{j} = 0.
	      \end{equation}
	\item Збереження енергії (є наслідком рівнянь Максвелла)
	      \begin{equation}
		      \frac{\partial w}{\partial t} + \divg\vect{S} = -  \vect{j}\cdot\Efield,
	      \end{equation}
\end{itemize}
де $ w $~-- густина енергії електромагнітного поля:
\begin{equation}
	w = \frac{1}{8\pi} (\Efield^2 + \Bfield^2),
\end{equation}
\noindent%
$ \vect{S} $~-- вектор Пойнтінга (вектор густини потоку енергії):
\begin{equation}
	\vect{S} = \frac{c}{4\pi} \Efield\times\Bfield.
\end{equation}
Величина
\begin{equation}
	p =\vect{j}\cdot\Efield ,
\end{equation}
є потужністю, що виділяється в одиниці об'єму речовини.

\section{Граничні умови для границі розділу двох середовищ}

\begin{align}
	\left[ \vect{n} \times (\Efield_{1} - \Efield_{2}) \right] & = 0, \label{EnBc}                        \\
	\left( \Efield_{1} - \Efield_{2}\right) \cdot \vect{n}     & = 4\pi\sigma, \label{EtauBc}             \\
	\left( \Bfield_{1} - \Bfield_{2}\right) \cdot \vect{n}     & = 0,    \label{BnBc}                     \\
	\left[ \vect{n} \times (\Bfield_{1} - \Bfield_{2}) \right] & = \frac{4\pi}{c}\vect{i} \label{BtauBc},
\end{align}

\noindent%
де $\sigma$~-- поверхнева густина вільних зарядів на границі розділу,\\
$\vect{i}$~-- поверхневий струм провідності, який протікає по границі розділу, \\
$\vect{n}$~--- вектор нормалі з середовища $2$  в середовище $1$.

\section{Потенціали електромагнітного поля}

\begin{align}
	\Efield & = -\frac{1}{c}\frac{\partial \vect{A}}{\partial t} - \vect{\nabla}\phi, \label{E=phi+A} \\
	\Bfield & = \rot\vect{A},
\end{align}
де $\phi$~-- скалярний потенціал, $ \vect{A} $~-- вектор-потенціал.

Потенціали електромагнітного поля не визначені однозначно. Якщо замість потенціалів $\vect{A}$ та $\phi$ вибрати інші $\vect{A}'$ та $\phi'$, які пов'язані з вихідними перетвореннями:
\begin{align}
	\vect{A}' & = \vect{A} +\vect{\nabla}f,                     \\
	\phi'     & = \phi - \frac1c \frac{\partial f}{\partial t},
\end{align}
де  $f(x,y,z,t)$~-- довільна функція, то поля $\Efield$ та $\Bfield$ залишаються незмінними.

Умова
\begin{equation}\label{Lorentz_gauge}
	\vect\nabla\cdot\vect{A} = - \frac{1}{c}\frac{\partial\phi}{\partial t},
\end{equation}
називається калібруванням Лоренца і зазвичай використовується для динамічних задач.
Використовуючи калібрування Лоренца, рівняння для скалярного і векторного потенціалів можна записати у симетричному вигляді:
\begin{align}\label{Maxwell_equations_via_potantials}
	\Delta \vect{\phi} - \frac{1}{c^2} \frac{\partial^2 \phi}{\partial t^2} = -4\pi\rho, \\
	\Delta \vect{A} - \frac{1}{c^2} \frac{\partial^2 \vect{A}}{\partial t^2} = - \frac{4\pi}{c}\vect{j}.
\end{align}

У випадку ізольованої системи зарядів та струмів розв'язками цих рівнянь є \emph{запізнюючі потенціали}:
\begin{equation}\label{Retarded_potentials}
	\phi(\vect{r} , t) = \iiint\limits_{\Omega}\frac{\rho\left( \vect{r}', t - \frac{|\vect{r} - \vect{r}'|}{c}\right) }{|\vect{r} - \vect{r}'|}dV',
	\quad
	\vect{A}(\vect{r} , t) = \frac1c\iiint\limits_{\Omega}\frac{\vect{j}\left( \vect{r}', t - \frac{|\vect{r} - \vect{r}'|}{c}\right) }{|\vect{r} -
	\vect{r}'|}dV'.
\end{equation}

%\renewcommand{\theequation}{\thechapter.\arabic{equation}}

\multiinclude{\ChaptersOne}[]

\part{Електродинаміка суцільних середовищ}
\epigraph{\Annabelle  Вважаю, що більш приземлені та матеріальні науки аж ніяк не можуть бути зневажені у порівнянні з піднесеним вивченням розуму і духу \ldots}{James Clerk Maxwell}
% !TeX program = lualatex
% !TeX encoding = utf8
% !TeX spellcheck = uk_UA
% !TeX root =../FTProblems.tex
\newpage
\chapter{Система \mbox{макроскопічних} рівнянь Максвелла}

%\epigraph{\Annabelle  Вважаю, що більш приземлені та матеріальні науки аж ніяк не можуть бути зневажені у порівнянні з піднесеним вивченням розуму і духу \ldots}{James Clerk Maxwell}
%\setcounter{equation}{0}
%\renewcommand{\theequation}{\thepart.\arabic{equation}}

\section{Макроскопічні рівняння Максвелла}

Макроскопічні рівняння Максвелла, або рівняння макроскопічної електродинаміки є результатом усереднення мікроскопічних рівнянь Максвелла.

Інтегральна форма макроскопічних рівнянь:
\begin{align}
	\oiint\limits_S \Dfield\cdot d\vect{S} & = 4\pi\iiint\limits_{V}\rho dV,   \label{essInt I}                                                                                                         \\
	\oiint\limits_S \Bfield\cdot d\vect{S} & = 0,   \label{essInt II}                                                                                                                                   \\
	\oint\limits_L \Efield\cdot d\vect{r}  & = - \frac1c \iint\limits_S \frac{\partial\Bfield}{\partial t}\cdot d\vect{S},  \label{Int III}                                                             \\
	\oint\limits_L \Hfield\cdot d\vect{r}  & =\dfrac{4\pi}{c} \iint\limits_S \vect{j}\cdot d\vect{S} +\frac{1}{c} \iint\limits_S  \frac{\partial\Dfield}{\partial t}\cdot d\vect{S},  \label{essInt IV}
\end{align}
де $ \Efield $~-- вектор напруженості електричного поля;\\
\hspace*{3ex}$ \Bfield $~-- вектор індукції магнітного поля; \\
\hspace*{3ex}$ \rho $~--- густина вільних електричних зарядів (не включає зв'язані заряди); \\
\hspace*{3ex}$\vect{j}$~--- густина вільного струму (не включає струми намагнічення чи струми, які виникають за рахунок змінної поляризації); \\
$ \Dfield $~-- вектор індукції електричного поля, вводиться як:
\begin{equation}
	\Dfield = \Efield + 4\pi\vect{P},
\end{equation}
де $\vect{P}$~---  вектор поляризації, або густина дипольного моменту;\\
\hspace*{2ex} $ \Hfield $~-- вектор напруженості магнітного поля, вводиться як:

\begin{equation}
	\Bfield = \Hfield + 4\pi\vect{M},
\end{equation}
де  $\vect{M}$~--- вектор намагнічення, або густина магнітного моменту.
\clearpage
Диференціальна форма  макроскопічних рівнянь:
\begin{flalign}
	\divg\Dfield &= 4\pi\rho, \label{essDiff I}\\[0.8em]
	\divg\Bfield &= 0, \label{essDiff II}\\
	\rot\Efield &= -\dfrac{1}{c}\dfrac{\partial\Bfield}{\partial t}, \label{essDiff III}\\
	\rot\Hfield &= \dfrac{4\pi}{c} \vect{j}+\dfrac{1}{c}\dfrac{\partial\Dfield}{\partial t}. \label{essDiff IV}
\end{flalign}



Зв'язок між об'ємною густиною зв'язаних $\rho_\text{зв'яз}$ зарядів та вектором поляризації $ \vect{P} $ всередині діелектрика
\begin{equation}
	\rho_\text{зв'яз} = -\vect{\nabla}\cdot\vect{P},
\end{equation}


Поверхнева густина зв'язаних зарядів на межі розділу діелектриків:
\begin{equation}\label{}
	\sigma_\text{зв'яз} = (\vect{P}_1 - \vect{P}_2)\cdot\vect{n},
\end{equation}
де $ \vect{n}$~-- нормаль до поверхні розділу діелектриків (напрямлена від $2$-го до $1$-го середовища).

Ці співвідношення не є універсальним, але вони виконуються за досить широких умов,
зокрема, коли макроскопічні електромагнітні поля є значно меншими, ніж поля усередині атомів та молекул.

Зв'язок між струмами намагнічення $\vect{j}_m$:
\begin{equation}
	\vect{j}_\text{m} = c\vect{\nabla}\times\vect{M}.
\end{equation}
Ця формула також не є універсальною, але вона справедлива для широкого кола задач за помірних магнітних полів.

В результаті намагнічення на поверхні магнетика утворюються ефективні струми з поверхневою густиною:
\begin{equation}\label{}
	\vect{i}_\text{m} = -c\left[ \vect{n}\times\vect{M}\right],
\end{equation}
де $ \vect{n}$~-- зовнішня нормаль до поверхні магнетика.

\section{Співвідношення для лінійних ізотропних середовищ}

Величини $\Dfield$ і $\Hfield$ залежать, інколи досить складним чином, від
макроскопічних напруженості електричного поля $\Efield$ та індукції магнітного поля $\Bfield$. На цю залежність можуть впливати, прямо чи опосередковано, певні властивості чи характеристики середовища (густина, молекулярний склад, термодинамічні параметри), яке може бути однорідним чи неоднорідним, ізотропним чи анізотропним. Однак існує широке коло задач, де залежність $\Dfield$ і $\Hfield$  від $\Efield$ і $\Bfield$ можна вважати лінійною, і для її опису потрібно відносно невелике число параметрів середовища.


Так, вектор поляризації пропорційний напруженості електричного поля:
\begin{equation}\label{}
	\vect{P} = \alpha\Efield,
\end{equation}
де коефіцієнт $\alpha$ називають поляризовністю діелектрика.

Аналогічно, для магнітного поля вектор намагнічення пропорційний вектору напруженості магнітного поля:
\begin{equation}\label{}
	\vect{M} = \chi\Hfield,
\end{equation}
де коефіцієнт $\chi$ називають магнітною сприйнятливістю магнетика.


Використовуючи ці співвідношення і вводячи величини діелектричної проникності середовища
\[
	\epsilon = 1 + 4\pi\alpha,
\]
та магнітної проникності середовища
\[
	\mu = 1 + 4\pi\chi,
\]
маємо співвідношення між векторами $\Efield$ і $\Dfield$ та між $\Bfield$ і $\Hfield$:
\begin{align}
	\Dfield & = \epsilon\Efield, \\
	\Bfield & = \mu\Hfield.
\end{align}


\section{Умови на границі розділу двох середовищ}

\begin{align}
	\left[ \vect{n} \times (\Efield_{1} - \Efield_{2}) \right] & = 0, \label{essEnBc}                                      \\
	\left( \Dfield_{1} - \Dfield_{2}\right) \cdot \vect{n}     & = 4\pi\sigma_\text{вільн}, \label{essEtauBc}              \\
	\left( \Bfield_{1} - \Bfield_{2}\right) \cdot \vect{n}     & = 0,    \label{essBnBc}                                   \\
	\left[ \vect{n} \times (\Hfield_{1} - \Hfield_{2}) \right] & = \frac{4\pi}{c}\vect{i}_\text{вільн}, \label{essBtauBc},
\end{align}

\noindent%
де $ \vect{n}$~-- нормаль до поверхні розділу діелектриків (напрямлена від $2$-го до $1$-го середовища), \\
\hspace*{3ex}$\sigma_\text{вільн}$~-- поверхнева густина вільних зарядів на границі розділу,\\
\hspace*{3ex}$\vect{i}_\text{вільн}$~-- поверхнева густина вільного струму (наприклад, струму провідності в металах).

%\renewcommand{\theequation}{\thechapter.\arabic{equation}}
\multiinclude{\ChaptersTwo}[]

%========================================================================================================
%
%									       Вставка відповідей
%
%========================================================================================================
\bookmarksetup{startatroot}
\answers
\multiinclude{\ChaptersOne}[-Answers]
\multiinclude{\ChaptersTwo}[-Answers]
%========================================================================================================
%
%									             Додатки
%
%========================================================================================================
\appendix
% !TeX program = lualatex
% !TeX encoding = utf8
% !TeX spellcheck = uk_UA
% !TeX root =../FTProblems.tex

\section{Основні формули векторного аналізу}\label{Vanaliz}

\subsection{Тривимірний символ Леві-Чівіти}\label{Levi-Chiv}

Тривимірний символ Леві-Чівіти визначений співвідношеннями
\begin{equation}\label{eq:Levi-Chiv}
\epsilon_{ijk} =  - \epsilon_{jik} =  - \epsilon_{ikj}.
\end{equation}

Деякі співвідношення з цим символом:
\begin{align}
    \epsilon_{ijk}\epsilon_{kpq} &= \delta_{ip}\delta_{jq} - \delta_{iq}\delta_{jp}, \\
    \epsilon_{iqk}\epsilon_{pqk} &= 2\delta _{ip}.
\end{align}

Ротор та векторний добуток у декартових координатах $\{x,y,z\} = \{x_1, x_2, x_3\}$:
\begin{align}
\left[\rot\vect{A} \right]_i &= \epsilon_{ijk}{\partial_j}{A_k}, \\
\left[ \vect{A} \times \vect{B} \right]_i &= \epsilon_{ijk}{A_j}{B_k},
\end{align}
всі індекси пробігають значення $1$, $2$, $3$.

\subsection{Диференціальні операції в різних системах координат}

\subsubsection{Декартова система координат}

\begin{align}\label{cartesian}
	\mathrm{grad}\,\psi \equiv \vect{\nabla}\psi & = \frac{\partial \psi}{\partial x} \vect{e}_x + \frac{\partial \psi}{\partial y} \vect{e}_y +
	\frac{\partial \psi}{\partial z} \vect{e}_z \\
	\mathrm{div}\,\left(\mathrm{grad}\,\psi\right) \equiv \Laplasian\psi    & = \frac{\partial^2 \psi}{\partial x^2} + \frac{\partial^2 \psi}{\partial y^2} + \frac{\partial^2 \psi }{\partial z^2}               \\
	\mathrm{div}\,\vect{A} \equiv \divg\vect{A}     & = \frac{\partial A_x}{\partial x}  + \frac{\partial A_y}{\partial y} + \frac{\partial A_z}{\partial z}                              \\
	\mathrm{rot}\,\vect{A} \equiv  \rot\vect{A}      & = \left( \frac{\partial A_z}{\partial y}  - \frac{\partial A_y}{\partial z}\right)  \vect{e}_x +
	\left( \frac{\partial A_x}{\partial z}  - \frac{\partial A_z}{\partial x}\right)  \vect{e}_y +
	\left( \frac{\partial A_y}{\partial x}  - \frac{\partial A_x}{\partial y}\right)  \vect{e}_z
\end{align}

\subsubsection{Циліндрична система координат}

\begin{align}\label{cylindric}
		\vect{\nabla}\psi & = \frac{\partial \psi}{\partial \rho} \vect{e}_{\rho} + \frac{1}{\rho}\frac{\partial \psi}{\partial \phi} \vect{e}_{\phi} + \frac{\partial \psi}{\partial z} \vect{e}_{z}                                                                                       \\
	\Laplasian\psi    & = \frac{1}{r} \frac{\partial }{\partial r} \left( r \frac{\partial \psi}{\partial r} \right) + \frac{1}{r^2} \frac{\partial^2 \psi}{\partial \phi^2} + \frac{\partial^2 \psi}{\partial z^2}                                                                   \\
	\divg\vect{A}     & = \frac{1}{\rho}\frac{\partial \left(\rho A_{\rho }\right)}{\partial \rho }+\frac{1}{\rho }\frac{\partial A_{\phi } }{\partial \phi }+\frac{\partial A_{z}}{\partial z}                                                                                    \\
	\rot\vect{A}      & =\left({\frac {1}{\rho }}{\frac {\partial A_{z}}{\partial \phi }}-{\frac {\partial A_{\phi }}{\partial z}}\right)\vect{e}_{\rho}+\left({\frac {\partial A_{\rho }}{\partial z}}-{\frac {\partial A_{z}}{\partial \rho }}\right)\vect{e}_{\phi} + \nonumber \\
	                  & +
	{\frac {1}{\rho }}\left({\frac {\partial (\rho A_{\phi })}{\partial \rho }}-{\frac {\partial A_{\rho }}{\partial \phi }}\right)\vect{e}_{z}
\end{align}
Орти циліндричної системи координат зв'язані з декартовими ортами як:
\begin{align}\label{}
    \vect{e}_r &= \cos\phi\, \vect{e}_x + \sin\phi\, \vect{e}_y, \\
    \vect{e}_{\phi} &= -\sin\phi\, \vect{e}_x + \cos\phi\, \vect{e}_y, \\
    \vect{e}_{z} &= \vect{e}_z.
\end{align}

\subsubsection{Сферична система координат}

\begin{align}\label{spheric}
    \vect{\nabla}\psi & = \frac{\partial \psi}{\partial r} \vect{e}_{r} + \frac{1}{r}\frac{\partial \psi}{\partial \theta} \vect{e}_{\theta} + \frac{1}{r\sin\theta}\frac{\partial \psi}{\partial \phi} \vect{e}_{\phi}                                                                                                            \\
	\Laplasian\psi    & = \frac{1}{r^2} \frac{\partial}{\partial r} \left( r^2 \frac{\partial \psi}{\partial r} \right) + \frac{1}{r^2 \sin \theta} \frac{\partial \psi}{\partial \theta} \left( \sin \theta \frac{\partial \psi}{\partial \theta} \right) + \frac{1}{r^2\sin^2 \theta} \frac{\partial^2 \psi}{\partial \phi^2} \\
	\divg\vect{A}     & =\frac{1}{r^2}\frac{\partial \left(r^2A_r\right)}{\partial r}
	+
	\frac{1}{r\sin \theta }\frac{\partial}{\partial \theta }\left(A_{\theta }\sin \theta \right)
	+
	\frac{1}{r\sin\theta}\frac{\partial A_{\phi}}{\partial \phi }                                                                                                                                                                                                                                                            \\
	\rot\vect{A}      & = \frac{1}{r\sin \theta }\left(\frac{\partial}{\partial \theta }\left(A_{\phi }\sin \theta \right)-\frac{\partial A_{\theta }}{\partial \phi }\right)\vect{e}_{r}
	+
	\frac{1}{r}\left(\frac{1}{\sin \theta }\frac{\partial A_{r}}{\partial \phi }-\frac{\partial}{\partial r}\left(rA_{\phi }\right)\right)\vect{e}_{\theta}
	+ \nonumber                                                                                                                                                                                                                                                                                                                    \\
	                  & + \frac{1}{r}\left(\frac{\partial}{\partial r}\left(rA_{\theta }\right)-\frac{\partial A_{r}}{\partial \theta }\right)\vect{e}_{\phi}
\end{align}
Орти сферичної системи координат зв'язані з декартовими ортами як:
\begin{align}\label{}
    \vect{e}_r &= \sin\theta\cos\phi\, \vect{e}_x + \sin\theta\sin\phi\, \vect{e}_y + \cos\theta\, \vect{e}_z, \\
    \vect{e}_{\theta} &= \cos\theta\cos\phi\, \vect{e}_x + \cos\theta\sin\phi\, \vect{e}_y - \sin\theta\, \vect{e}_z, \\
    \vect{e}_{\phi} &= -\sin\phi\, \vect{e}_x + \cos\phi\, \vect{e}_y. \\
\end{align}

\subsection{Другі похідні}

\begin{align}
	\mathrm{rot}\,\mathrm{grad}\,\phi    & = \rot(\vect{\nabla}\phi)  = 0                                             \\
	\mathrm{div}\,\mathrm{rot}\,\vect{A} & = \divg(\rot\vect{A})  = 0                                                 \\
	\mathrm{rot}\,\mathrm{rot}\,\vect{A} & = \rot(\rot\vect{A})  = \vect{\nabla}(\divg\vect{A}) - \Laplasian \vect{A}
\end{align}


\subsection{Похідні від добутків}

\begin{align}
	\mathrm{grad}\,(\phi \psi)             & = \psi\,\mathrm{grad}\,\phi +\phi\, \mathrm{grad}\,\psi                                                                                                                           \\
	\mathrm{div}\,(\phi \vect{A})          & = \phi\,\mathrm{div}\,\vect{A} + \vect{A}\,\mathrm{grad}\,\phi                                                                                                                    \\
	\mathrm{rot}\,(\phi \vect{A})          & = \phi\,\mathrm{rot}\,\vect{A} + \mathrm{grad}\,\phi \times \vect{A}                                                                                                              \\
	\mathrm{grad}\,(\vect{A}\cdot\vect{B}) & = \vect{B}\times\mathrm{rot}\,\vect{A} + \vect{A}\times\mathrm{rot}\,\vect{B} + \left( \vect{B}\vect{\nabla}\right)\vect{A} + \left( \vect{A}\vect{\nabla}\right)\vect{B}         \\
	\mathrm{div}\,(\vect{A}\times\vect{B}) & = \vect{B}\cdot\mathrm{rot}\,\vect{A} - \vect{A}\cdot\mathrm{rot}\,\vect{B}                                                                                                       \\
	\mathrm{rot}\,(\vect{A}\times\vect{B}) & = \left( \vect{B}\vect{\nabla}\right)\vect{A} - \left( \vect{A}\vect{\nabla}\right)\vect{B} + 	\vect{A}\,\mathrm{div}\,\vect{B} - \vect{B}\,\mathrm{div}\,\vect{A} \label{rotvect} \\
	\frac12\mathrm{grad}\,A^2              & =  \left( \vect{A}\vect{\nabla}\right)\vect{A} + \vect{A}\times\mathrm{rot}\,\vect{A}
\end{align}


\subsection{Інтегральні характеристики та теореми}

Теорема Остроградського-Гаусса:
\begin{equation}\label{OGTheorem}
	\oiint\limits_{\partial \Omega} \vect{A}\cdot d\vect{S} = \iiint\limits_{\Omega} \divg\vect{A} dV,
\end{equation}
де $\Omega$~--- об'єм, $\partial\Omega$ --- його межа.

Теорема Стокса:
\begin{equation}\label{Stoksheorem}
	\oint\limits_{\partial S} \vect{A}\cdot d\vect{l} = \iint\limits_S \rot\vect{A} \cdot d\vect{S},
\end{equation}
де $S$~--- поверхня, що спирається на контур $\partial S$.

Теорема Гріна:
\begin{equation}\label{Grin}
	\iiint\limits_{\Omega}(\phi \nabla ^{2}\psi -\psi \nabla ^{2}\phi )\ dV=\iint \limits_{\partial \Omega}(\phi \vect{\nabla} \psi -\psi \vect{\nabla}
	\phi )\cdot d\vect{S}.
\end{equation}


\section{Поліноми Лежандра}\label{Polinoms}

Поліноми Лежандра застосовуються у теорії потенціалу при розкладанні виразу:% в околі точки $\vect{r}$:
\begin{equation*}\label{}
    \frac{1}{|\vect{r} - \vect{r}_0|} = \frac{1}{\sqrt{r^2 - 2rr_0 \cos\chi + r_0^2 }}  =  \sum\limits_{l = 0}^{\infty} \frac{r^l_<}{r^{l+1}_>} P_l(\cos\chi),
\end{equation*}
де $r_>$, $r_<$~--- більша і менша із величин $|\vect{r}|$ та $\vect{r}_0$, відповідно, $\cos\chi$~--- кут між векторами $\vect{r}$ та $|\vect{r}_0|$.

\medskip%
\textbf{Деякі поліноми Лежандра}

\begin{align*}
P_{0}(\cos\chi)  = 1, &\quad P_{1}(\cos\chi)  = \cos\chi, \\
P_{2}(\cos\chi)  = \frac {1}{2}(3\cos^2\chi-1), &\quad P_{3}(\cos\chi)  = \frac {1}{2}(5\cos^2\chi-3\cos\chi).
\end{align*}


\section{Сферичні гармоніки}\label{Spherical_Harmonics}

Сферичні функції, що залежать від полярних кутів визначаються формулою:
\begin{equation*}
    Y_{lm}(\theta,\phi) = (-1)^{(m+|m|)/2} \sqrt{\frac{2l+1}{4\pi}\frac{(l-m)!}{(l+m)!}}P_l^{|m|}(\cos \theta) e^{im\phi},
\end{equation*}
де $l = 0,1,2, \ldots$, $m$ пробігає значення від $-l$ до $l$, а $P_l^{|m|}(x)$~--- приєднані функції Лежандра.

Вони утворюють повну ортонормовану систему функцій:
\[
    \int\limits_{\theta=0}^{\pi} \int\limits_{\phi=0}^{2\pi} Y^*_{l',m'}(\theta, \phi) Y_{l,m}(\theta, \phi) \sin\theta d\theta d\phi  = \delta_{l,l'} \delta_{m,m'}.
\]

Деякі сферичні гармоніки:
\begin{align*}
	Y_{0,0}(\theta,\phi)     & =\sqrt{1\over 4\pi},                                            \\
	Y_{1,\pm 1}(\theta,\phi) & =\sqrt{3\over 8\pi} \, \sin\theta \, e^{\pm i\phi},             \\
	Y_{1,0}(\theta,\phi)     & =\sqrt{3\over 4\pi}\, \cos\theta ,                              \\
	Y_{2,0}(\theta,\phi)     & =\sqrt{5\over 16\pi}\, (3\cos^{2}\theta-1),                     \\
	Y_{2,\pm 1}(\theta,\phi) & =\sqrt{15\over 8\pi}\, \sin\theta\, \cos\theta\, e^{\pm i\phi}, \\
	Y_{2,\pm 2}(\theta,\phi) & =\sqrt{15\over 32\pi} \, \sin^{2}\theta \, e^{\pm2i\phi}.
\end{align*}

\section{Циліндричні функції}\setcounter{equation}{0}

Рівняння, що виникають в задачах з циліндричною симетрією, мають вигляд:
	\begin{equation}\label{eq:Bessel_eq}
		\frac{d^2y}{dx^2} + \frac1x\frac{dy}{dx} + \left(1 - \frac{m^2}{x^2} \right) y = 0,
	\end{equation}
	розв'язок яких можна представити за допомогою функцій Бесселя $J_m(x)$ та Неймана $N_m$ у вигляді лінійної комбінації $y(x) = A J_m(x) + B N_m(x)$ або у вигляді лінійної комбінації $y(x) = A H^{(1)}_m(x) + B H^{(2)}_m(x)$, де функції $H_m^{(1,2)} =J_m \pm i N_m$~--- називаються функціями Ганкеля 1-го та 2-го роду, відповідно. Доцільність введення функцій Ганкеля обумовлена тим, що вони мають прості асимптотичні розкладання при $|x| \gg 1$ і зручні для задач, пов'язаних з поширенням хвиль.

Для $m = 0,1,2,\ldots$ функції Неймана нескінченні в точці $x = 0$, тобто $\lim\limits_{x\to0}N_m(x) = -\infty$.

Функції Бесселя можна представити за допомогою ряду (в околі точки $x = 0$ для цілих, або невід'ємних $m$):
\begin{equation}\label{eq:J}
    J_m(x) = \sum_{n=0}^\infty \frac{(-1)^n}{n! \Gamma(n+m+1)} {\left({\frac{x}{2}}\right)}^{2n+m},
\end{equation}
де $ \Gamma$~--- \href{https://en.wikipedia.org/wiki/Gamma_function}{гамма-функція}. Для $m \in \mathbb{Z}$ має місце рівність $J_{-m}(x)  = (-1)^m J_m(x)$.

\begin{center}
  \begin{tikzpicture}
    \begin{axis}[axis lines = middle,
			axis line style={-stealth},
			minor grid style = {line width=.1pt,draw=gray!10},
            width=\textwidth, height=0.5*\textwidth, xlabel=$x$,
            xtick=\empty,
%            ytick={-0.5,0.5,1},
            legend style={draw=none},
    ]
    \addplot+[id=parable,domain=-20:20, samples=500, mark=none, width=2pt, color=red, thick]
    gnuplot{besj0(x)};% node[pin=95:{$J_0(x)$}]{};
    \addplot+[id=parable,domain=-20:20, samples=500, mark=none, width=2pt, color=blue, thick]
    gnuplot{besj1(x)};% node[pin=130:{$J_1(x)$}]{};
    \addplot+[id=parable2,domain=-20:20, samples=500, mark=none, width=2pt, color=green!50!black, thick]
    gnuplot{2*1/x*besj1(x)-besj0(x)};% node[pin=-140:{$J_2(x)$}]{};
    \legend{$J_0(x)$,$J_1(x)$,$J_2(x)$}
   \end{axis}
  \end{tikzpicture}

    {Графіки функцій Бесселя $J_m$ для $m = 0,1,2$.}
\end{center}

\begin{center}
  \begin{tikzpicture}
    \begin{axis}[axis lines = middle,
			axis line style={-stealth},
			minor grid style = {line width=.1pt,draw=gray!10},
            width=\textwidth, height=0.5*\textwidth, xlabel=$x$,
            xtick=\empty,
%            ytick={-0.5,0.5,1},
            legend style={at={(current axis.south east)}, anchor = south east, draw=none},
    ]
    \addplot+[id=parable,domain=0:40, restrict y to domain=-1.5:1, samples=500, mark=none, width=2pt, color=red, thick]
    gnuplot{besy0(x)};% node[pin=95:{$J_0(x)$}]{};
    \addplot+[id=parable,domain=0:40, restrict y to domain=-1.5:1, samples=1000, mark=none, width=2pt, color=blue, thick]
    gnuplot{besy1(x)};% node[pin=130:{$J_1(x)$}]{};
    \addplot+[id=parable2,domain=0:40, restrict y to domain=-1.5:1, samples=1000, mark=none, width=2pt, color=green!50!black, thick]
    gnuplot{2*1/x*besy1(x)-besy0(x)};% node[pin=-140:{$J_2(x)$}]{};
    \legend{$N_0(x)$,$N_1(x)$,$N_2(x)$}
   \end{axis}
  \end{tikzpicture}

    {Графіки функцій Неймана $N_m$ для $m = 0,1,2$.}
\end{center}

Функції Неймана визначаються через функції Бесселя як:
\begin{equation}\label{eq:NG}
    N_m(x) = \frac{J_m(x)\cos m\pi - J_{-m}(x)}{\sin m\pi}.
\end{equation}


Деякі рекурентні співвідношення:
\begin{equation}
    J_{m + 1}(x) + J_{m - 1}(x) = \frac{2m}{x}J_m(x).
\end{equation}

Деякі диференціальні та інтегральні співвідношення для нецілих $m$ (для цілих $m$ ці функції можна визначити за допомогою граничного переходу).:
\begin{align}
    \frac{d}{dx} J_0(x) = - J_1(x), \\
    \frac{d}{dx} \left( x^{-m}J_m(x)\right)  = - x^{-m}J_{m+1}(x), \\
     \int\limits_0^{x} x^{\prime m+1} J_m(x') dx' &= x^{m+1}J_{m+1}. \label{eq:recInt}
\end{align}

Інтеграли від добутків:
\begin{equation}
    \int\limits_0^x  J_m(k_1x')J_m(k_2x') x' dx' = \frac{x\left( k_2J_m(k_1x)J'_m(k_2x) - k_1J_m(k_2x)J'_m(k_1x)\right)}{k_1^2-k_2^{2}}   \label{eq:JJ0*}.
\end{equation}

В задачах, зазвичай, часто необхідно знайти наближений вигляд циліндричних функцій при малих та великих значеннях аргументу $x$:

при $|x| \ll 1$ з~\eqref{eq:J}

\begin{equation}
    J_0(x) \approx 1 - \frac{x^2}{4}, \quad
    J_m \approx \frac{x^m}{2^m m!}, \ m \ge 1,\, m \in \mathbb{N};
\end{equation}

при $|x| \gg 1$

\begin{align}
    J_m &\approx \sqrt{\frac{2}{\pi x}}  \cos\left( x - m\frac{\pi}{2} - \frac{\pi}{4}\right),
      \label{eq:Jxgg1}\\
    N_m &\approx \sqrt{\frac{2}{\pi x}}  \sin\left( x - m\frac{\pi}{2} - \frac{\pi}{4}\right),
  \label{eq:Yxgg1}\\
    H_m^{(1,2)} &\approx \sqrt{\frac{2}{\pi x}}  e^{\pm i \left( x - m\frac{\pi}{2} - \frac{\pi}{4}\right) }. \label{eq:Hxgg1}
\end{align}

Співвідношення Якобі-Ангера (розкладання за функціями Бесселя):
\begin{equation}\label{eq:JacobiAnger}
    e^{ix\cos\theta} = \sum\limits_{m= - \infty}^{\infty} i^mJ_m(x)e^{im\theta}, \quad
    e^{ix\sin\theta} = \sum\limits_{m= - \infty}^{\infty} J_m(x)e^{im\theta}. \\
\end{equation}

\section{Дельта-функція Дірака}\setcounter{equation}{0}

Дельта-функція Дірака (або $\delta$-функція) є узагальненою функцією і була введена фізиком Полем Діраком для моделювання густини ідеалізованої точкової маси або точкового заряду.

На <<фізичному рівні строгості>> можна визначити $\delta$-функцію формальним співвідношенням:
\begin{equation}
    \int\limits_{-\infty}^{+\infty}f(x)\delta(x-x_0)\,dx=f(x_0),
\end{equation}
У випадку інтегрування по скінченному об'єму $V$:
\begin{equation}\label{eq:delta3D}
    \int\limits_{V}f(\vect{r})\delta(\vect{r} - \vect{r}_0) dV=f(\vect{r}_0),
\end{equation}
де точка $\vect{r}_0$ знаходиться всередині об'єму $V$.

%$\delta$-функцію однієї дійсної змінної можна визначити як функцію, що задовольняє наступним умовам:
%\begin{equation}\label{eq:delta}
%\delta(x - x_0)=\left\{\begin{matrix}
%   +\infty, & x=x_0, \\
%   0, & x \neq x_0; \\
%\end{matrix}\right.
%\end{equation}
%\begin{equation}\label{eq:deltaProp0}
%    \int\limits_{-\infty}^{+\infty}\delta(x-x_0)dx=1.
%\end{equation}
%Тобто ця функція не дорівнює нулю тільки в точці $x = x_0$, де вона перетворюється в нескінченність таким чином, щоб її інтеграл в будь-якому околі точки $x = x_0$ дорівнює $1$.
%
%Аналітичне представлення $\delta$-функції:
%\begin{equation}\label{eq:deltaanalit}
%    \delta(x) = \frac{1}{2\pi}\int\limits_{-\infty}^{\infty} e^{ikx}dk.
%\end{equation}
\bigskip\noindent%
\textbf{Властивості дельта-функції}
\bigskip

\begin{enumerate}[label=\alph*)]
\item Дельта-функція парна $\delta(-x) = \delta(x)$,
\item $x\delta(x) = 0$,
\item $\delta(ax) = \frac{1}{|a|}\delta(x)$,
\item  $x\delta^\prime(x)=-\delta(x)$,
\item $\delta(f(x))=\sum\limits_k\frac{\delta(x-x_k)}{|f'(x_k)|}$, де $x_k$~--- нулі функції $f(x)$,

\end{enumerate}

%\subsection{Розкладання дельта-функції в ряд Фур'є}
%
%$\delta$-Функцію можна представити у вигляді ряду Фур'є на деякому проміжку $x \in (-l,l)$:
%\begin{equation}\label{eq:deltacos}
%    \delta(x) = \frac{1}{2l} +  \frac1l \sum\limits_{m=1}^{\infty}\cos\frac{m\pi}{l}x,
%\end{equation}
%або в комплексній формі
%\begin{equation}\label{eq:deltae}
%    \delta(x) =  \frac{1}{2l} \sum\limits_{m=-\infty}^{\infty}e^{i\frac{2m\pi}{l}x}.
%\end{equation}
%У випадку, якщо ми маємо періодичну функцію у вигляді $\delta$-<<частоколу>>~--- $\sum\limits_{k=-\infty}^{\infty}\delta(x-ka)$~--- відстань між піками якого дорівнює $a=2l$, то для нього розкладання також матиме вигляд~\eqref{eq:deltacos} (або \eqref{eq:deltae}):
%\begin{equation}\label{eq:deltalll}
%    \sum\limits_{k=-\infty}^{\infty}\delta(x-ka)  = \frac{1}{a} +  \frac2a \sum\limits_{m=1}^{\infty}\cos\frac{2m\pi}{a}x
%\end{equation}
% і буде справедливе на всій числовій прямій $x \in (-\infty,\infty)$.
%
%\subsection{Тривимірна дельта-функція}
%
%Тривимірна $\delta$-функція визначається співвідношеннями:
%\begin{equation}\label{eq:delta3D}
%    \delta(\vect{r} - \vect{r}_0) = \delta(x-x_0)\delta(y-y_0)\delta(z-z_0),
%\end{equation}
%\begin{equation}\label{eq:delta3D}
%    \int\limits_{V}f(\vect{r})\delta(\vect{r} - \vect{r}_0) dV=f(\vect{r}_0),
%\end{equation}
%де точка $\vect{r}_0$ знаходиться в середині об'єму $V$.
%
%Аналітичне представлення тривимірної $\delta$-функції:
%\begin{equation}\label{eq:deltaanalit3В}
%    \delta(\vect{r}) = \frac{1}{(2\pi)^3}\int\limits_{-\infty}^{\infty} e^{i\vect{k}\cdot\vect{r}}d\vect{k},
%\end{equation}
%де $d\vect{k}=dk_xdk_ydk_z$.



\input{Additions/PhysConstants}

%%========================================================================================================
%
%									       Вставка бібліографії
%
%=========================================================================================================
\clearpage\pagestyle{bibliography}

\nocite{%
	ZhdanovFT,
	ZhdanovESS,
    ZhdanovRelativity,
	TerletskyElectroDyn,
	Jackson,
	GriffithsElectro,
	PennerUgarov,
	Tamm,
	GreinerElectrodynamics,
	LL2,
	LL8,
	Vlasov,
	% ==========================
	BatyginElectroRus,
	Grechko,
	Kramm,
	Zhyrnov,
	AlekseevElectro,
	VekshteinElectro,
	Ginldenburg,
    ZangwillElectro,
}



\printbibheading[heading=bibintoc]

\defbibheading{subheading}[\bibname]{%
	\section*{#1\label{BibPage}}
}

\printbibliography[category=Textbooks, heading=subheading, title={Підручники та посібники}]
\addtocategory{Textbooks}{%
	ZhdanovFT,
	ZhdanovESS,
    ZhdanovRelativity,
	Jackson,
	GriffithsElectro,
	PennerUgarov,
	Tamm,
	GreinerElectrodynamics,
	LL2,
	TerletskyElectroDyn,
	Vlasov,
	LL8,
    ZangwillElectro,
}


\printbibliography[category=Problems, heading=subheading, title={Задачники}]
\addtocategory{Problems}{%
	BatyginElectroRus,
	Grechko,
	Kramm,
	Zhyrnov,
	AlekseevElectro,
	VekshteinElectro,
	Ginldenburg,
}

\makelastpage
\end{document}
