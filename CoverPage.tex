% !TeX root =./FTProblems.tex

%========================================================================================================
%
%									      Палітурка
%
%========================================================================================================

\newcommand{\CoverPage}{
	\begin{alwayssingle}
		\begin{center}
			\begin{flushright}\bfseries\sffamily
				\MakeUppercase{Міністерство освіти і науки України}\\
				КПІ ім. Ігоря Сікорського\\
			\end{flushright}
			\begin{tcolorbox}[titlepagestyle,
					toprule=0.10cm,
					bottomrule=0.10cm,
					overlay={%
						\node (picture) at ([xshift=4cm]frame.west) {\includegraphics[width=3.5cm]{logo_IPT}};
					}
			]%
			\begin{flushright}
				\large\bfseries\color{white}Фізико-технічний інститут
			\end{flushright}
			\end{tcolorbox}
			\vspace*{-2em}
		 	\begin{Large}\color{themecolordark!90!black}
			\begin{alignat*}{3}
				\vect{\nabla}&\times&&\Efield &&= -\dfrac{1}{c}\dfrac{\partial\Bfield}{\partial t} \\
				\vect{\nabla}&\,\cdot&&\Bfield &&= 0 \\
				\vect{\nabla}&\,\cdot&&\Dfield &&= 4\pi\rho \\
				\vect{\nabla}&\times&&\Hfield &&= \dfrac{4\pi}{c} \vect{j}+\dfrac{1}{c}\dfrac{\partial\Dfield}{\partial t}
			\end{alignat*}
			\end{Large}
			\vspace*{2em}
			\begin{tcolorbox}[
				titlepagestyle,
				toprule=0.15cm,
				bottomrule=0.15cm,
				top=1.3cm,
				bottom=0.7cm,
				overlay={%
				\node[%
							fill=white,
							rounded corners = 15pt,
							draw=themecolorlight,
							line width=0.15cm,
							inner sep=0pt,
							text width=17cm,
							minimum height=2cm,
							align=center,
							%anchor=east,
							font=\sffamily\bfseries\large
						] (title) at (frame.north) {В.~І. Жданов, %
													С.~М. Пономаренко,
													В.~Б. Долгошей%
					};
				}
			]
			\centering
			\Huge\sffamily\bfseries\textcolor{white}{\realtitle}\\
			\huge\sffamily\bfseries\textcolor{white}{\subtitle}
			\end{tcolorbox}
			\vfill
		 	\begin{Large}\color{themecolordark!90!black}
			\begin{gather*}
				\frac{\partial F^{\mu\nu}}{\partial x^\nu} =                                                                                -\frac{4\pi}{c}j^\mu \\
				\frac{\partial F_{\beta\gamma}}{\partial x^\alpha} + \frac{\partial F_{\gamma\alpha}}{\partial x^\beta} +\frac{\partial
				F_{\alpha\beta}}{\partial x^\gamma}  = 0.
			\end{gather*}
			\end{Large}
			\vfill
			\begin{tcolorbox}[titlepagestyle,
					toprule=0.10cm,
					bottomrule=0.10cm]
				\begin{center}\color{white}\bfseries\normalsize
					\MakeUppercase{Київ~\the\year} \\
%					КПІ ім. Ігоря Сікорського \\
%					\the\year
				\end{center}
			\end{tcolorbox}
		\end{center}
		\clearpage
	\end{alwayssingle}
\setcounter{page}{1}
}


%========================================================================================================
%
%									      Титульна сторінка
%
%========================================================================================================

\renewcommand\maketitle{
	\begin{alwayssingle}
		\begin{center}
				\MakeUppercase{Міністерство освіти і науки України}

				\bigskip
				\MakeUppercase{Національний технічний університет України}\\
				<<КИЇВСЬКИЙ ПОЛІТЕХНІЧНИЙ ІНСТИТУТ \\ імені ІГОРЯ СІКОРСЬКОГО>>
				\vspace*{100pt}

				{\large В.~І.~Жданов,
						С.~М.~Пономаренко,
						В.~Б.~Долгошей
						}
				\vspace*{50pt}

				{\Huge\sffamily\bfseries\realtitle}\\[1em]
				{\huge\sffamily\bfseries\subtitle}

			\vspace*{50pt}
			\begin{center}\itshape
			Рекомендовано Методичною радою КПІ ім. Ігоря Сікорського як навчальний посібник для
			здобувачів ступеня бакалавра за спеціальностями E6 <<Прикладна фізика та наноматеріали>>,
			F1 <<Прикладна математика>>
			\end{center}
            \vfill
                \textcolor{malina}{Версія від~\today, час компіляції \currenttime.}
			\vfill
			\begin{center}
				\MakeUppercase{Київ} \\
				КПІ ім. Ігоря Сікорського \\
				\the\year
			\end{center}
		\end{center}
		\clearpage
	\end{alwayssingle}
}


%========================================================================================================
%
%									      Друга сторінка
%
%========================================================================================================
\newcommand\makeinfopage{
	\begin{alwayssingle}
		\noindent%
		\begin{minipage}[t]{0.5\textwidth}
				\begin{flushleft}
					УДК  537\\
					Ж 42
				\end{flushleft}
		\end{minipage}


		\bigskip\noindent%
        \begin{minipage}[t]{0.2\linewidth}
            	\begin{flushleft}
                    Рецензенти:
                \end{flushleft}
        \end{minipage}\hfill
        \begin{minipage}[t]{0.78\linewidth}
                \href{http://www.nas.gov.ua/UA/PersonalSite/Pages/default.aspx?PersonID=0000006576}{В.~О.~Кочелап}, д.ф.-м.н., професор,
                член-кореспондент НАН України, завідувач відділу теоретичної фізики Інституту фізики напівпровідників ім. В.~Є. Лашкарьова
%                \\[1ex]
%                \href{http://apd.ipt.kpi.ua/blog/author/19}{Я.~Д.~Кривенко-Еметов}, к.ф.-м.н., доцент кафедри прикладної фізики, Фізико-технічного
%інституту КПІ ім. Ігоря Сікорського
        \end{minipage}

		\bigskip\noindent%
        \begin{minipage}[t]{0.2\linewidth}
            	\begin{flushleft}
                    Відповідальний редактор:
                \end{flushleft}
        \end{minipage}\hfill
        \begin{minipage}[t]{0.78\linewidth}
                \href{http://is.ipt.kpi.ua/is/smirnovsa/}{С.~А.~Смирнов}, к.ф.-м.н., доцент, голова
                методичної ради НН ФТІ КПІ ім. Ігоря Сікорського
        \end{minipage}

		\begin{center}\itshape\small
				Гриф надано Методичною радою КПІ ім. Ігоря Сікорського (протокол №~10/2020~від 18.06.2020~р.) за поданням Вченої ради Фізико-технічного інституту (протокол №~5/2020 від 25.05.2020 р.)
		\end{center}
		\begin{center}
			\ifelectronic Електронне мережне навчальне видання \fi
			%\par {Версія від~\href{http://www.istpravda.com.ua/dates}{\today}} \par\else \par  \fi
		\end{center}
		\begin{center}
			\href{https://apd.ipt.kpi.ua/kafedra/stuff/zhdanov}{\itshape Жданов Валерій Іванович},
			д.ф.-м.н., професор \\
			\href{https://apd.ipt.kpi.ua/kafedra/stuff/ponomarenko}{\itshape Пономаренко Сергій
			Миколайович}, к.ф.-м.н., доцент \\
			\href{https://apd.ipt.kpi.ua/kafedra/stuff/dolgoshey}{\itshape Долгошей Володимир
			Борисович}, к.ф.-м.н., доцент
		\end{center}
%			\vspace*{1em}%
		\begin{center}\bfseries
			\LARGE\sffamily\realtitle \\
			\Large\sffamily\subtitle
		\end{center}
        \noindent%
        \begin{minipage}[t]{\textwidth}\small
                \realtitle: \subtitle\ [Електронний ресурс] : навч. посіб. для студ. спеціальностей
                E6 <<Прикладна фізика та наноматеріали>> та F1 <<Прикладна математика>> /  В.~І.
                Жданов, С.~М. Пономаренко, В.~Б. Долгошей ; КПІ ім. Ігоря Сікорського.~--- Електронні
                текстові дані
            (1 файл: 570~кБ). – Київ : КПІ ім. Ігоря Сікорського, \the\year. --- \the\numexpr\getpagerefnumber{LastPages}-1\relax~с.
        \end{minipage}

%		\noindent%
%		\begin{flushleft}
%			\begin{tabular}{lp{0.9\textwidth}}
%				     & В.~І. Жданов, С.~М. Пономаренко, В.~Б. Долгошей                                                                                                                                                                        \\
%				Ж 42 & \hspace*{3ex} \realtitle : \subtitle{} [Електронний ресурс] : навчальний посібник / %
%				В.~І. Жданов, %
%				С.~М. Пономаренко%,
%				В.~Б. Долгошей %
%				--- К.:~КПІ ім. Ігоря Сікорського, \the\year. --~\the\numexpr\getpagerefnumber{LastPage}-1\relax~с. -- Бібліогр.: с.~\pageref{BibPage}\relax. \ifelectronic\relax\else-- 80~прим.\fi
%			\end{tabular}
%		\end{flushleft}
		\vfill

		Наведено \TotalValue{totalproblems} задач з курсу класичної електродинаміки. До більшості задач подано відповіді. Задачі різноманітні як за змістом, так і за складністю. Поряд із задачами, які ілюструють основні поняття і закони електродинаміки, до збірника включена певна кількість більш складних задач, які допомагають суттєвішому вивченню предмета.

		Для студентів фізико-технічного інституту КПІ ім. Ігоря Сікорського, які навчаються за спеціальностями 105~<<Прикладна фізика та наноматеріали>> та 113 <<Прикладна математика>>.

		\vfill

%		\begin{flushleft}\small
%			Ілюстративний матеріал підручника підготовлений за допомогою пакету \href{http://pgf.sourceforge.net}{TikZ/Pgf}. Верстка тексту проведена в видавничій системі \LaTeXe{} (компілятор Lua\LaTeX) на базі системи комп'ютерної верстки \TeX{} (Збірка  \href{https://www.tug.org/texlive/}{\TeX Live~\the\year}) з використанням оболонки \href{https://www.texstudio.org}{\TeX Studio}.
%		\end{flushleft}
	\hfill
	\begin{minipage}[t]{0.65\linewidth}\small
        \textcopyright{} В.~І. Жданов, С.~М. Пономаренко, В.~Б. Долгошей, \the\year\,р. \\
        \textcopyright{}  КПІ ім. Ігоря Сікорського (ФТІ), \the\year~р.
    \end{minipage}
		\newpage%
	\end{alwayssingle}
}

%%%%%%%%%%%%%%%%%%%%%%%%%%%%%%%%%%%%%%%%%%%%%%%%%%%%%%%%%%%%%%%%%%%%%%%%%%%%
%%                                                                        %%
%%                              Last Page                                 %%
%%                                                                        %%
%%%%%%%%%%%%%%%%%%%%%%%%%%%%%%%%%%%%%%%%%%%%%%%%%%%%%%%%%%%%%%%%%%%%%%%%%%%%
\newcommand{\makelastpage}{%
\clearpage%
\thispagestyle{empty}%
\vspace*{0.4\textheight}
\begin{center}
		\textbf{Жданов} Валерій Іванович \\
	    \textbf{Пономаренко} Сергій Миколайович \\
	    \textbf{Долгошей} Володимир Борисович
\end{center}

\begin{center}\bfseries
    \Large\sffamily\realtitle \\
    \large\sffamily\subtitle
\end{center}

\begin{center}\itshape
    Комп'ютерне верстання в системі \LaTeXe{} С.\ М. Пономаренко
\end{center}

\vspace*{1em}
\begin{center}\small
Національний технічний університет України \\
<<Київський політехнічний інститут імені Ігоря Сікорського>> \\
Свідоцтво про державну реєстрацію: серія ДК № 5354 від 25.05.2017 р.\\
просп. Перемоги, 37, м. Київ, 03056
\end{center}

%\vfill
%
%\begin{center}
%    Видавництво <<Політехніка>> \kpishort \\
%    вул. Політехнічна, 14, корп. 15 \\
%    м. Київ, 03056 \\
%    Тел. (044) 204-81-78
%\end{center}
}


