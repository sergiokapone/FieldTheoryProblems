% !TeX program = lualatex
% !TeX encoding = utf8
% !TeX spellcheck = uk_UA
% !TeX root =../FTProblems.tex

%=========================================================
\Opensolutionfile{answer}[\currfilebase/\currfilebase-Answers]
\Writetofile{answer}{\protect\section*{\nameref*{\currfilebase}}}
\chapter{Співвідношення векторного аналізу в декартових координатах }\label{\currfilebase}
%=========================================================

\begin{Theory}
В декартових координатах використовуємо позначення:
\[
\{x, y, z\} = \{x_1, x_2, x_3\} =  \vect{r}.
\]
Латинські індекси \(i, j, k\) пробігатимуть допустимі значення 1,2,3. Якщо індекси у виразі повторюються, це означає суму по цих індексах (правило Ейнштейна).

Сукупність усіх частинних похідних від скалярного поля \(\psi( \vect{r}) = \psi(x_1, x_2, x_3)\) утворює градієнт:
\[
\nabla F =
\begin{bmatrix}
\frac{\partial F}{\partial x_1}, & \frac{\partial F}{\partial x_2}, & \frac{\partial F}{\partial x_3}
\end{bmatrix}
=
\begin{bmatrix}
\frac{\partial F}{\partial x}, & \frac{\partial F}{\partial y}, & \frac{\partial F}{\partial z}
\end{bmatrix}
= \Grad{F},
\]
або покомпонентно:
\[
(\nabla F)_i = \frac{\partial F}{\partial x_i} = \partial_i F.
\]
Тут і надалі \((X)_i = X_i\) означає \(i\)-ту компоненту вектора \(X\).

Означимо дивергенцію векторного поля \( \vect{A}( \vect{r})\):
\[
\Div\vect{A} = \partial_j A_j = \nabla \cdot  \vect{A};
\]
та оператор Лапласа:
\[
\Delta F = \partial_j \partial_j F = \frac{\partial^2 F}{\partial x_1^2} + \frac{\partial^2 F}{\partial x_2^2} + \frac{\partial^2 F}{\partial x_3^2}.
\]
Тут \(F = F( \vect{r}) = F(x_1, x_2, x_3)\) -- скалярна функція; але це може бути і компонента векторного поля в декартових координатах.

Введемо абсолютно антисиметричний символ \(\epsilon_{ijk}\): \(\epsilon_{123} = 1\) (символ Леві-Чівіти):
\[
\epsilon_{ijk} = -\epsilon_{jik} = -\epsilon_{ikj}.
\]

Користуючись властивостями визначника, легко перевірити, що:

\[
\varepsilon_{ijk} =
\begin{vmatrix}
\delta_{1i} & \delta_{2i} & \delta_{3i} \\
\delta_{1j} & \delta_{2j} & \delta_{3j} \\
\delta_{1k} & \delta_{2k} & \delta_{3k}
\end{vmatrix},
\]
де $\delta_{ij}$ -- символ Кронекера.

За допомогою цього символу можна записати $i$-ту компоненту векторного добутку:
\[
[ \vect{A} \times  \vect{B}]_i = \epsilon_{ijk} A_j B_k
\]
($A_i$, $B_i$ -- компоненти векторів $ \vect{A}$ та $ \vect{B}$), а також \textbf{ротор векторного поля}:
\[
(\boldsymbol{\nabla} \times \vect{A})_i = \varepsilon_{ijk} \partial_j A_k.
\]

Формула Остроградського--Гаусса (див. «Додатки», \eqref{OGTheorem}):
\begin{equation}\label{eq:OGF}
    \iiint\limits_\Omega (\vect{\nabla} \cdot \vect{A})  d\Omega = \oiint\limits_{\partial \Omega}  \vect{A} \cdot d \vect{S};
\end{equation}
$\Omega$ -- об'єм, $\partial \Omega$ -- його межа.
\end{Theory}


%=========================================================
\begin{problem}%
Виходячи з поданих означень, перевірити співвідношення
	\begin{enumerate}[label=\alph*)]
		\item $\divg (F\vect{A}) = \partial (FA_i) = A_i \partial F + F \partial A_i = \vect{A} \cdot \nabla F + F\ \divg \vect{A}$;
		\item $\rot(F\vect{A})_i = \varepsilon_{ijk} \partial_j (FA_k) = \varepsilon_{ijk} (\nabla F)_j A_k + F \varepsilon_{ijk} \partial_j A_k$, або $\rot(F\vect{A}) = [\nabla F \times \vect{A}] + F \rot\vect{A}$.
        \item Дивергенція векторного добутку $\divg[\vect{A} \times \vect{B}] = \vect{B} \cdot \rot \vect{A} - \vect{A} \cdot \rot\vect{B}$.
	\end{enumerate}
\end{problem}

%=========================================================
\begin{problem}%
Отримати формулу для \textbf{згортки двох символів} Леві-Чивіти
\begin{equation}\label{eq:LCh}
\epsilon_{ijk}\epsilon_{irs} = \delta_{jr}\delta_{ks} - \delta_{js}\delta_{kr}.
\end{equation}

Вказівка: записати
\[
\varepsilon_{ijk}\varepsilon_{irs} =
\begin{vmatrix}
\delta_{ii} & \delta_{ir} & \delta_{is} \\
\delta_{ji} & \delta_{jr} & \delta_{js} \\
\delta_{ki} & \delta_{kr} & \delta_{ks}
\end{vmatrix},
\]
розкрити визначник за правилом трикутників та взяти суму по $k=r$.
\end{problem}


%=========================================================
\begin{problem}%
Отримати за допомогою \eqref{eq:LCh} формулу <<BAC-CAB>>:
\[
 [\vect{A} \times [ \vect{B} \times  \vect{C}]] =  \vect{B}( \vect{A} \cdot  \vect{C}) -  \vect{C}( \vect{A} \cdot  \vect{B})
\]
\end{problem}



%=========================================================
\begin{problem}
За допомогою \eqref{eq:LCh} вивести формулу $\vect{\nabla} \times (\vect{\nabla} \times \vect{A}) = \vect{\nabla}(\vect{\nabla} \cdot \vect{A}) - \Delta \vect{A}$, де $(\Delta \vect{A})_i = \partial_j \partial_j A_i$ (для компонент векторного поля в декартових координатах).
\end{problem}

%=========================================================
\begin{problem}
Наслідки формули Остроградського–Гаусса \eqref{eq:OGF}:
\begin{enumerate}[label=\alph*)]
\item $\iint\limits_{\Omega} \vect{\nabla} F \, dV = \oint\limits_{\partial\Omega} F \, d\vect{S}$.
\item $\iiint\limits_{\Omega} \rot \vect{A} \, dV = \iint\limits_{\partial\Omega} [d\vect{S} \times \vect{A}]$.
\item Довести формулу Гріна \eqref{Grin}) з «Додатків».
\end{enumerate}
\end{problem}

%=========================================================
\begin{problem}%
За допомогою формули Стокса (Додатки, \eqref{Stoksheorem}) перетворити такі інтеграли по замкнутому контуру $C$, що обмежує кусок поверхні $S$:
\begin{enumerate}[label=\alph*)]
\item $\oint\limits_{\partial S} u \, df = \iint\limits_{S} [\nabla u \times \nabla f] \cdot d\vect{S}$;
\item $\oint\limits_{\partial S} u \, d\vect{l} = \iint\limits_{S} d\vect{S} \times \nabla u$.
\end{enumerate}
$u$, $f$ -- Скалярні функції.

Зауваження: орієнтація поверхні $S$ має відповідати напрямку обходу по контуру $\partial S$
\end{problem}


\Closesolutionfile{answer}

