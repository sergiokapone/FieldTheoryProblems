% !TeX program = lualatex
% !TeX encoding = utf8
% !TeX spellcheck = uk_UA


\documentclass[]{ProblemBook}



\begin{document}

%=========================================================
\begin{problem}
    Металевий циліндр радіусу $a$ внесли у зовнішнє однорідне магнітне поле  $B_z = B_0e^{-i\omega t}$, яке паралельне до його осі. Система знаходиться у вакуумі.
\begin{enumerate*}[label=\alph*)]
    \item Знайдіть густину струму Фуко всередині циліндра нехтуючи ефектами самоіндукції.
    \item Знайдіть першу поправку до магнітного поля та густини струму Фуко в циліндрі врахувавши ефекти самоіндукції.
\end{enumerate*}
 Порівняйте результати з точним розв'язком. 
\begin{solution}

\noindent\hrulefill

Використовуючи закон Фарадея $\oint\limits_L \Efield\cdot d\vect{r}  = - \frac1c \iint\limits_S \frac{\partial\Bfield}{\partial t}\cdot d\vect{S}$
\[
    2\pi r E_{\phi}^{(0)} = -\frac1{c} (-i\omega\mu)B_0e^{-i\omega t} \pi r^2,
\]
маємо
\[
    E_{\phi}^{(0)} =  B_0e^{-i\omega t} \frac{i\mu\omega}{2c}r.
\]
Використовуючи закон Ома, знайдемо густину струмів Фуко:
\[
    j_{\phi}^{(0)} = B_0e^{-i\omega t} \frac{i\lambda\mu \omega}{2c}r.
\]
Введемо коефіцієнт $k^2 = \frac{4i\pi\mu\lambda\omega}{c^2}$:
\[
    j_{\phi}^{(0)}(r) = B_0e^{-i\omega t} \frac{k^2c}{8\pi}r.
\]
\end{solution}

Поправки

\bigskip

Знайдемо першу поправку до магнітного поля, зумовлену виникненням струму Фуко. Ці струми Фуко утворюють вкладені один в одний <<соленоїди>>, тому магнітне поле цих струмів можна знайти як
\[
    H_z^{(1)} = \int\limits_r^{a} \frac{4\pi}{c} j_{\phi}^{(0)}(r')dr' = B_0e^{-i\omega t} \frac{k^2}{4}(a^2 - r^2). 
\]
З урахуванням цієї поправки, повне магнітне поле буде в циліндрі:
\[
    H_z =  B_0e^{-i\omega t}\left[ 1 + \frac{k^2}{4}(a^2 - r^2) \right]. 
\]

Знайдемо першу поправку до електричного поля:

\[
    2\pi r E_\phi^{(1)} = -\frac1c \mu \dot{H}_z^{(1)} \pi r^2, 
\]
звідки
\[
    E_\phi^{(1)} =  \frac{\mu i \omega}{c} \frac{k^2}{4}B_0e^{-i\omega t} r(a^2 - r^2),
\]
а перша поправка до густини струму
\[
    j_\phi^{(1)} = \lambda E_\phi^{(1)} = \frac{\lambda\mu i \omega}{c} \frac{k^2}{4}B_0e^{-i\omega t} (a^2 - r^2) = \frac{k^4c}{16\pi} B_0e^{-i\omega t} r(a^2 - r^2),
\]
отже густина струму Фуко 
\[
    j_{\phi} = j_\phi^{(0)} + j_\phi^{(1)} = \frac{k^2c}{8\pi}B_0e^{-i\omega t}r\left[ 1 + \frac{k^2}{4}\left( a^2 - r^2\right) \right]. 
\]
З останнього виразу видно, що впливом самоіндукції можна нехтувати у випадку, якщо $ka \ll 1$, тобто при низьких частотах.

Порівняємо з точним розв'язком:

Магнітне поле в середині циліндра:
\[
    H_z = \frac{J_0(kr)}{J_0(ka)}B_0e^{-i\omega t}.
\]
Оскільки $J_0(kr) \approx 1 - \frac{k^2r^2}{4}$, $J_0(ka) \approx 1 - \frac{k^2a^2}{4}$, а
\begin{multline*}
    H_z \approx \frac{1 - \frac{k^2r^2}{4}}{1 - \frac{k^2a^2}{4}}B_0e^{-i\omega t} \approx \left( 1 - \frac{k^2r^2}{4} \right)\left( 1 + \frac{k^2a^2}{4} \right) B_0e^{-i\omega t} \approx\\\approx B_0e^{-i\omega t}\left[ 1 + \frac{k^2}{4}(a^2 - r^2)\right] .
\end{multline*}

Точний вираз для густини струму
\[
    j_\phi = \frac{kc}{4\pi}\frac{J_1(kr)}{J_0(ka)} B_0e^{-i\omega t}.
\]
Наближений вираз для $J_1(kr) \approx \frac{kr}{2} - \frac{k^3r^3}{8}$:
\begin{multline*}
   j_\phi \approx \frac{kc}{4\pi} \left( \frac{kr}{2} - \frac{k^3r^3}{8} \right) \left( 1 + \frac{k^2a^2}{4} \right) = \\ =
   \frac{kc}{4\pi} B_0e^{-i\omega t} \left( \frac{kr}{2}  + \frac{k^3a^2r}{8} - \frac{k^3r^3}{8} \right)   =  \frac{k^2c}{8\pi} B_0e^{-i\omega t} r\left[ 1 + \frac{k^2}{4}\left( a^2 - r^2\right) \right]
\end{multline*}

\begin{center}
\begin{tikzpicture}
\begin{axis}%
[
			axis lines = middle,
			axis line style={-stealth},
			% === Підпис координатних осей ===
            xtick = \empty,
            ytick = \empty,
			xlabel={$r$},
			ylabel={$j$},
		    % === Налаштування розміру графіка ===
   			width=1\linewidth,
   			height=1\linewidth,
   			% === Розширення границь осей ===
   			enlargelimits={abs=0.2},
]

\addplot[blue, domain=0:1, samples=500] {sqrt(1/x)*e^(-(1-x)/(1e-1))};
\draw[dashed] (axis cs:1,0) node[below] {$a$} -- (axis cs:1,1);
\draw[dashed] (axis cs:0,1) node[left] {$a$} -- (axis cs:1,1);
\end{axis}
\end{tikzpicture}
\end{center}
\end{problem}


\begin{center}
  \begin{tikzpicture}
    \begin{axis}[axis lines = middle,
			axis line style={-stealth},
			minor grid style = {line width=.1pt,draw=gray!10},
            width=\textwidth, height=0.5*\textwidth, xlabel=$x$,
            xtick=\empty,
%            ytick={-0.5,0.5,1},
            legend style={draw=none},
    ]
    \addplot+[id=parable,domain=-20:20, samples=500, mark=none, width=2pt, color=red, thick]
    gnuplot{besj0(x)};% node[pin=95:{$J_0(x)$}]{};
    \addplot+[id=parable,domain=-20:20, samples=500, mark=none, width=2pt, color=blue, thick]
    gnuplot{besj1(x)};% node[pin=130:{$J_1(x)$}]{};
    \addplot+[id=parable2,domain=-20:20, samples=500, mark=none, width=2pt, color=green!50!black, thick]
    gnuplot{2*1/x*besj1(x)-besj0(x)};% node[pin=-140:{$J_2(x)$}]{};
    \legend{$J_0(x)$,$J_1(x)$,$J_2(x)$}
   \end{axis}
  \end{tikzpicture}

    {Графіки функцій Бесселя $J_m$ для $m = 0,1,2$.}
\end{center}

\begin{center}
  \begin{tikzpicture}
    \begin{axis}[axis lines = middle,
			axis line style={-stealth},
			minor grid style = {line width=.1pt,draw=gray!10},
            width=\textwidth, height=0.5*\textwidth, xlabel=$x$,
            xtick=\empty,
%            ytick={-0.5,0.5,1},
            legend style={at={(current axis.south east)}, anchor = south east, draw=none},
    ]
    \addplot+[id=parable,domain=0:20, restrict y to domain=-1.5:1, samples=500, mark=none, width=2pt, color=red, thick]
    gnuplot{besy0(x)};% node[pin=95:{$J_0(x)$}]{};
    \addplot+[id=parable,domain=0:20, restrict y to domain=-1.5:1, samples=1000, mark=none, width=2pt, color=blue, thick]
    gnuplot{besy1(x)};% node[pin=130:{$J_1(x)$}]{};
    \addplot+[id=parable2,domain=0:20, restrict y to domain=-1.5:1, samples=1000, mark=none, width=2pt, color=green!50!black, thick]
    gnuplot{2*1/x*besy1(x)-besy0(x)};% node[pin=-140:{$J_2(x)$}]{};
    \legend{$N_0(x)$,$N_1(x)$,$N_2(x)$}
   \end{axis}
  \end{tikzpicture}

    {Графіки функцій Неймана $N_m$ для $m = 0,1,2$.}
\end{center}

\end{document}
