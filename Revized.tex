% !TeX program = lualatex
% !TeX encoding = utf8
% !TeX spellcheck = uk_UA


\documentclass[]{ProblemBook}


\begin{document}

%=========================================================
\begin{problem}[Клітка Фарадея]\label{prb:Faraday_cage}
Маємо два паралельних масива нескінченно довгих тонких заряджених провідників, напрямлених паралельно осі $z$, які формують нескінченні <<сітки>>, які простягаються до нескінченності по осі $-\infty < x < \infty$. Кожен провідник несе рівномірно розподілений заряд густиною $\lambda$. Дві сітки знаходяться на відстані $d$, як показано на рисунку. Відстань між сусідніми провідниками однієї сітки $a$. Знайдіть потенціал електричного поля у просторі між сітками.
\begin{solution}
	Знайдемо, для зручності, потенціал нижньої <<сітки>> ($z = 0$). Маємо двовимірну задачу.
	Геометрія задачі дає умови симетрії $\pot(-x,-y) = \pot(x,y)$ та періодичності $\pot(x+a,z) = \pot (x,z)$.

	Розв'язок рівняння Лапласа шукаємо у  вигляді:
	\[
		\pot(x,y) = B|y| + \sum\limits_{n = 1}^{\infty} C_n\cos(nx)e^{-n|y|}.
	\]
	Оскільки на далеких відстанях $|z| \gg a$ поле <<сітки>> має виглядати як поле нескінченної зарядженої площини, то
	\[
		B = -\frac{2\pi\lambda}{a},
	\]
	а умова періодичності дає $n = \frac{2\pi mx}{a}$, де $m = 1,2, \ldots$. Отже,
	\[
		\pot(x,y) = -\frac{2\pi\lambda}{a}|y| + \sum\limits_{m = 1}^{\infty} C_m\cos(2\pi mx/a)e^{-2\pi m|y|/a}.
	\]
	Для знаходження коефіцієнтів $C_m$ використаємо граничну умову:
	\[
		\left. \frac{\partial\pot}{\partial y}\right|_{y = 0^-} - \left. \frac{\partial\pot}{\partial y}\right|_{y = 0^+} = 4\pi\sigma(x),
	\]
	де $\sigma(x) = \lambda\sum\limits_{p = -\infty}^{\infty} \delta(x - pa)$.
	Розкладання в ряд Фур'є періодичної функції:
	\[
		\sum\limits_{p = -\infty}^{\infty} \delta(x - pa) =  \frac{1}{a} + \frac{2}{a}\sum\limits_{m=1}^{\infty}\cos(2\pi mx/a),
	\]
	звідки $C_m = \frac{2\lambda}{m}$.
	Остаточно маємо потенціал нижньої <<сітки>>:
	\[
		\pot(x,y) = -\frac{2\pi\lambda}{a}|y| + 2\lambda\sum\limits_{m = 1}^{\infty} \frac1m \cos(2\pi mx/a)e^{-2\pi m|y|/a}.
	\]

	Для верхньої  <<сітки>> потенціал знаходиться аналогічно, і результат відрізняється з точністю до заміни $y \to y - d$.

	В області $0 < z <d$, якщо $z \gg a$ та $d - z \gg a$ (далеко від <<сітки>>), сумарний потенціал стає майже постійним

	\[
		\pot(x,y) \approx \frac{4\pi\lambda d}{a},
	\]

	а електричне поле, відповідно, нульовим. Саме цим пояснюється дія клітки Фарадея.

\end{solution}
\end{problem}
\end{document}
