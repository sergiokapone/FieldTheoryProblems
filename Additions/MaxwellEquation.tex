% !TeX program = lualatex
% !TeX encoding = utf8
% !TeX spellcheck = uk_UA
% !TeX root =../FTProblems.tex

\newpage
\chapter{Система \mbox{мікроскопічних} рівнянь Максвелла}

%\epigraph{\Annabelle  Неможливо позбутися відчуття, що ці математичні формули існують незалежно від нас і володіють власним розумом, що вони мудріші за нас, мудріше навіть тих, хто їх відкрив, і що ми дістаємо з них більше, ніж спочатку було закладено\ldots}{Heinrich Hertz}
%\epigraph{\Annabelle Уж не Боги ли начертали эти знаки?}{Л.~Больцман}

%\renewcommand{\theequation}{\thepart.\arabic{equation}}
\section{Рівняння Максвелла}

Система рівнянь Максвелла сформульована в 60-х роках XIX століття на основі узагальнення емпіричних
законів і розвитку ідей про електромагнітні явища. Ці рівняння складають основу теорії
електромагнітного поля; вони дозволяють знаходити електричні та магнітні поля, створювані зарядами і
струмами, та описувати поширення електромагнітних хвиль.

Інтегральна форма мікроскопічних рівнянь:
\begin{align}
	\oiint\limits_{\partial\Omega} \Efield\cdot d\vect{S} & = 4\pi\iiint\limits_{\Omega}\rho dV   \label{Int
	I},                                                                                                         \\
	\oiint\limits_{\partial\Omega} \Bfield\cdot d\vect{S} & = 0   \label{Int
	II},                                                                                                                                   \\
	\oint\limits_{\partial S} \Efield\cdot d\vect{r}  & = - \frac1c \iint\limits_S
	\frac{\partial\Bfield}{\partial t}\cdot d\vect{S}  \label{Int
	III},                                                          \\
	\oint\limits_{\partial S} \Bfield\cdot d\vect{r}  & =\dfrac{4\pi}{c} \iint\limits_S \vect{j}\cdot d\vect{S} +\frac{1}{c} \iint\limits_S
	\frac{\partial\Efield}{\partial t}\cdot d\vect{S}  \label{Int IV},
\end{align}
де $\Omega$ --- довільний нерухомий об'єм, $\partial\Omega$ --- його межа; $S$ --- довільна нерухома орієнтована поверхня, $\partial S$ --- замкнений
контур, що її обмежує.

Диференціальна форма  мікроскопічних рівнянь:
\begin{flalign}
	\divg\Efield &= 4\pi\rho \label{Diff I},\\[0.8em]
	\divg\Bfield &= 0 \label{Diff II},\\
	\rot\Efield &= -\dfrac{1}{c}\dfrac{\partial\Bfield}{\partial t} \label{Diff III},\\
	\rot\Bfield &= \dfrac{4\pi}{c} \vect{j}+\dfrac{1}{c}\dfrac{\partial\Efield}{\partial t} \label{Diff IV}.
\end{flalign}


\noindent%
$ \Efield $~-- вектор напруженості електричного поля,\\
$ \Bfield $~-- вектор індукції магнітного поля.


\section{Закони збереження (диференціальна форма)}

\begin{itemize}
	\item Електричного заряду
	      \begin{equation}
		      \frac{\partial \rho}{\partial t} + \divg\vect{j} = 0.
	      \end{equation}
	\item Збереження енергії (є наслідком рівнянь Максвелла)
	      \begin{equation}
		      \frac{\partial w}{\partial t} + \divg\vect{S} = -  \vect{j}\cdot\Efield,
	      \end{equation}
\end{itemize}
де $ w $~-- густина енергії електромагнітного поля:
\begin{equation}
	w = \frac{1}{8\pi} (\Efield^2 + \Bfield^2),
\end{equation}
\noindent%
$ \vect{S} $~-- вектор Пойнтінга (вектор густини потоку енергії):
\begin{equation}
	\vect{S} = \frac{c}{4\pi} \Efield\times\Bfield.
\end{equation}
Величина
\begin{equation}
	p =\vect{j}\cdot\Efield ,
\end{equation}
є потужністю, що виділяється на одиницю об'єму речовини.

\section{Граничні умови для границі розділу двох середовищ}

\begin{align}
	\left[ \vect{n} \times (\Efield_{1} - \Efield_{2}) \right] & = 0, \label{EnBc}                        \\
	\left( \Efield_{1} - \Efield_{2}\right) \cdot \vect{n}     & = 4\pi\sigma, \label{EtauBc}             \\
	\left( \Bfield_{1} - \Bfield_{2}\right) \cdot \vect{n}     & = 0,    \label{BnBc}                     \\
	\left[ \vect{n} \times (\Bfield_{1} - \Bfield_{2}) \right] & = \frac{4\pi}{c}\vect{i} \label{BtauBc},
\end{align}

\noindent%
де $\sigma$~-- поверхнева густина вільних зарядів на межі розділу,\\
$\vect{i}$~-- поверхневий струм провідності, який протікає по межі розділу, \\
$\vect{n}$~--- вектор нормалі з середовища $2$  в середовище $1$.

\section{Потенціали електромагнітного поля}

\begin{align}
	\Efield & = -\frac{1}{c}\frac{\partial \vect{A}}{\partial t} - \vect{\nabla}\phi, \label{E=phi+A} \\
	\Bfield & = \rot\vect{A},
\end{align}
де $\phi$~-- скалярний потенціал, $ \vect{A} $~-- вектор-потенціал.

Потенціали електромагнітного поля не визначені однозначно. Якщо замість потенціалів $\vect{A}$ та $\phi$ вибрати інші $\vect{A}'$ та $\phi'$, які пов'язані з вихідними перетвореннями:
\begin{align}
	\vect{A}' & = \vect{A} +\vect{\nabla}f,                     \\
	\phi'     & = \phi - \frac1c \frac{\partial f}{\partial t},
\end{align}
де  $f(x,y,z,t)$~-- довільна функція, то поля $\Efield$ та $\Bfield$ залишаються незмінними.

Умова
\begin{equation}\label{Lorentz_gauge}
	\vect\nabla\cdot\vect{A} = - \frac{1}{c}\frac{\partial\phi}{\partial t},
\end{equation}
називається калібруванням Лоренца і зазвичай використовується для динамічних задач.
Використовуючи калібрування Лоренца, рівняння для скалярного і векторного потенціалів можна записати у симетричному вигляді:
\begin{align}\label{Maxwell_equations_via_potantials}
	\Delta \vect{\phi} - \frac{1}{c^2} \frac{\partial^2 \phi}{\partial t^2} = -4\pi\rho, \\
	\Delta \vect{A} - \frac{1}{c^2} \frac{\partial^2 \vect{A}}{\partial t^2} = - \frac{4\pi}{c}\vect{j}.
\end{align}

У випадку ізольованої системи зарядів та струмів розв'язками цих рівнянь є \emph{запізнюючі потенціали}:
\begin{equation}\label{Retarded_potentials}
	\phi(\vect{r} , t) = \iiint\limits_{\Omega}\frac{\rho\left( \vect{r}', t - \frac{|\vect{r} - \vect{r}'|}{c}\right) }{|\vect{r} - \vect{r}'|}dV',
	\quad
	\vect{A}(\vect{r} , t) = \frac1c\iiint\limits_{\Omega}\frac{\vect{j}\left( \vect{r}', t - \frac{|\vect{r} - \vect{r}'|}{c}\right) }{|\vect{r} -
	\vect{r}'|}dV'.
\end{equation}

%\renewcommand{\theequation}{\thechapter.\arabic{equation}}
