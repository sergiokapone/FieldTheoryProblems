\protect \section *{\nameref *{EMstaticsInMedia}}
\begin{Solution}{6.{5}}
    $\Efield_2 = \Efield_1 + \left( \frac{\epsilon_1}{\epsilon _2} - 1 \right)\vect{n}\left( \vect{n} \cdot \Efield_1 \right) - \frac{4\pi \sigma }{\epsilon _2}{\vect{n}}$.
\end{Solution}
\begin{Solution}{6.{6}}
	$
		\pot  = \frac{2}{{{\varepsilon _1} + {\varepsilon _2}}}\frac{q}{r},
	$
	$
		\Efield = \frac{2q}{\epsilon_1 + \epsilon_2}\frac{\vect{r}}{r^3},
	$
	$ \Dfield_i = \frac{2\epsilon_i}{\epsilon_1 + \epsilon_2}\frac{q\vect{r}}{r^3} $ ($i = 1,2$).
\end{Solution}
\begin{Solution}{6.{7}}
	Граничні умови на поверхні провідника $\pot = \const$  та межі розділу діелектриків можна задовольнити, використовуючи сферично-симетричний потенціал  $\pot(r) = \frac{C}{r}$, $r > R$ . Константу $C$ знаходимо з теореми Гаусса для індукції. Остаточно маємо потенціал:
	\[
		\pot(r) = \frac{2q}{(\epsilon_1 + \epsilon_2)r}, \quad r > R,
	\]
	та густину зарядів на поверхнях кулі:
	\[
		\sigma_i = \frac{q\epsilon_i}{2\pi(\epsilon_1 + \epsilon_2)R^2}, \quad i = 1,2.
	\]
\end{Solution}
\begin{Solution}{6.{14}}
	Зовні кулі на зовнішнє однорідне поле накладається поле, що створюється поляризаційними зарядами які виникли на кулі. З урахуванням нульової умови на нескінченності для поля зарядів на кулі та умови регулярності в початку координат, шукатимемо потенціал у вигляді:
	\[
		\pot(r) =
		\begin{cases}
			-Cr\cos\theta, \quad r \le R \\
			-E_0 r \cos\theta + \frac{p\cos\theta}{r^2}, \quad r > R.
		\end{cases}
	\]
	Граничні умови мають вигляд:
	\[
		\left. \pot^{(i)} \right|_{r = R} = \left. \pot^{(e)} \right|_{r = R} \left., \quad \epsilon^{(i)}\frac{\partial\pot^{(i)}}{\partial r}\right|_{r = R} =  \left. \epsilon^{(e)}\frac{\partial\pot^{(e)}}{\partial r}\right|_{r = R},
	\]
	і дають змогу визначити константу $C$ та дипольний момент $p$:
	\[
		\begin{cases}
			-CR\cos\theta = -E_0 R \cos\theta + \frac{p\cos\theta}{R^2} \\
			\epsilon^{(i)}(-C\cos\theta) = \epsilon^{(e)} E_0 \cos\theta + \epsilon^{(e)}\frac{2p\cos\theta}{R^3}.
		\end{cases}
	\]
	Звідки:
	\[
		p = \frac{\epsilon^{(i)} - \epsilon^{(e)}}{\epsilon^{(i)} + 2\epsilon^{(e)}}R^3E_0,
	\]
	\[
		C = \frac{3\epsilon^{(e)}}{\epsilon^{(i)} + 2\epsilon^{(e)}}.
	\]
	Потенціал дорівнює:
	\[
		\pot =
		\begin{cases}
			-\frac{3\epsilon^{(e)}}{\epsilon^{(i)} + 2\epsilon^{(e)}}\left( \Efield_0\cdot \vect{r}\right), & r \le R \\
			-\left( \Efield_0\cdot \vect{r}\right) + \frac{\vect{p} \vect{r}}{r^3},                         & r > R.
		\end{cases}
	\]
	Густина зв'язаного заряду:
	\[
		\sigma_   \text{зв'яз} = P_{1n} - P_{2n} = \frac{1}{4\pi} \left( \left.\frac{\partial\pot^{(i)}}{\partial r}\right|_{r = R} -  \left. \frac{\partial\pot^{(e)}}{\partial r}\right|_{r = R} \right)  = \frac{3}{4\pi} \frac{\epsilon_i - \epsilon_e}{\epsilon_i + 2\epsilon_e} E_0\cos\theta.
	\]
\end{Solution}
\begin{Solution}{6.{15}}
	\emph{Вказівка}: Шукаємо розв’язок рівняння Лапласа всередині кулі (в сферичних координатах)  у вигляді $\pot(r,\theta,\phi) = f(r)\cos\theta$. Маємо загальний розв'язок для  $f(r) = A\frac{r}{R} + B\left( \frac{R}{r}\right)^2$.  Константи знаходимо з умови регулярності в нулі та з граничної умови при  $r = R$. Густина зв’язаних зарядів визначається за формулою $\sigma = \vect{P}\cdot\vect{n}$.
\end{Solution}
\begin{Solution}{6.{16}}
	$\pot = \pot_0 \left( \frac{R}{r}\right)^2\cos\theta $.
\end{Solution}
\begin{Solution}{6.{17}}
	\emph{Вказівка}: Шукаємо розв’язок рівняння Лапласа всередині кулі (в сферичних координатах)  у вигляді $\pot(r,\theta,\phi) = f(r)\cos\theta$. Маємо загальний розв'язок для  $f(r) = A\frac{r}{R} + B\left( \frac{R}{r}\right)^2$.  Константи $A$ та $B$ в середині та зовні кулі різні. Знаходимо іх з нульової умови на нескінченності при $r>R$, умови регулярності та з граничної умови при  $r = R$. Густину вільного заряду знаходимо з граничної умови для нормальної компоненти індукції на поверхні кулі.
\end{Solution}
\begin{Solution}{6.{18}}
	\(
		\pot(R,z,\phi) =
		\begin{cases}
			\pot _0\left( \frac{r}{R} \right)^m\cos (m\varphi), r < R \\
			\pot _0\left( \frac{R}{r} \right)^m\cos (m\varphi), r \ge R.
		\end{cases}
	\),
	\(
		\sigma_\text{вільн} = \frac{m(\varepsilon  + 1)}{4\pi }\cos (m\varphi ).
	\)
\end{Solution}
\begin{Solution}{6.{19}}
	Розв'язок слід шукати окремо зовні кулі ($r >R$ ) та всередині ($r<R$). Користуючись повнотою системи сферичних функцій на сфері, шукаємо розв’язок у вигляді:
	\[
		\pot(r,\theta ,\varphi ) = \sum\limits_{n = 0}^\infty  {\sum\limits_{m =  - n}^n {{f_{nm}}(r){Y_{nm}}(\theta ,\varphi )} } .
	\]
	Підставимо це в рівняння Лапласа в сферичних координатах:
	\begin{equation}\label{Laplace_prb:L}
		\Delta \pot  \equiv \frac{1}{{{r^2}}}\frac{\partial }{{\partial r}}\left( {{r^2}\frac{{\partial \pot }}{{\partial r}}} \right) + \frac{1}{{{r^2}}}\hat \Lambda \psi  = 0,
	\end{equation}
	де \(\hat \Lambda {Y_{nm}} \equiv \frac{1}{{\sin \theta }}\frac{\partial }{{\partial \theta }}\left( {\sin \theta \frac{{\partial {Y_{nm}}}}{{\partial \theta }}} \right) + \frac{1}{{{{\sin }^2}\theta }}\frac{{{\partial ^2}{Y_{nm}}}}{{\partial {\varphi ^2}}} =  - n\left( {n + 1} \right){Y_{nm}}\left( {\theta ,\varphi } \right)\) .

	Звідси, з огляду на незалежність сферичних функцій з різними індексами і лінійність рівняння Лапласа по $\pot$  (принцип суперпозиції!), кожну складову потенціалу можна розглядати окремо. Маємо
	\[
		\frac{1}{{{r^2}}}\frac{\partial }{{\partial r}}\left( {{r^2}\frac{{\partial {f_{nm}}}}{{\partial r}}} \right) - \frac{{n\left( {n + 1} \right)}}{{{r^2}}}{f_{nm}}(r) = 0
	\]
	для усіх $n= 0,1,2,\ldots$ , і $m = -n, -(n-1), \ldots, 0, \ldots, n-1, n$  для кожного $n$.
	Загальний розв’язок рівняння~\eqref{Laplace_prb:L} є комбінацією степеневих функцій
	\begin{equation}\label{star_prb:L}
		f_{nm}(r) = A_{nm}{\left( {\frac{r}{R}} \right)^n} + {B_{nm}}{\left( {\frac{R}{r}} \right)^{n + 1}}
	\end{equation}

	Нормування на $R$ введено для зручності, константи $A_{nm}$ та $B_{nm}$~--- різні всередині і зовні кулі.
	Зовні кулі ($r >R$) поле обмеженої системи зарядів має прямувати до нуля на нескінченності, тому $A_{nm} = 0$,

	\begin{equation}\label{dstar_prb:L}
		\pot_\mathrm{out}(r,\theta ,\varphi ) = \sum\limits_{n = 0}^\infty  {\sum\limits_{m =  - n}^n {{B_{nm}}{{\left( {\frac{R}{r}} \right)}^{n + 1}}{Y_{nm}}(\theta ,\varphi )} }.
	\end{equation}

	Аналогічно розглядаємо розв’язок усередині кулі, де, однак, поле має бути регулярним в нулі (\emph{в центрі немає якихось точкових зарядів}); тому в~\eqref{star_prb:L} слід відкинути члени з від'ємними степенями змінної $r < R$ :
	\begin{equation}\label{ddstar_prb:L}
		\pot_\mathrm{in}(r,\theta ,\varphi ) = \sum\limits_{n = 0}^\infty  {\sum\limits_{m =  - n}^n {{{\tilde A}_{nm}}{{\left( {\frac{r}{R}} \right)}^n}{Y_{nm}}(\theta ,\varphi )} }
	\end{equation}
	Маємо дві граничні умови на поверхні кулі ($r = R$). Умову для тангенціальних компонент напруженості електричного поля в даній задачі можна замінити умовою неперервності потенціалу
	\[
		\pot_\mathrm{out}(R,\theta ,\varphi ) = {\pot_\mathrm{in}}(R,\theta ,\varphi ).
	\]
	Це дає  $\tilde{A}_{nm} = B_{mn}$.
	Друга гранична умова використовує нормальні компоненти індукції $D_n =  - \epsilon \vect{n} \cdot \vect{\nabla}\pot  =  - \epsilon \frac{\partial \pot }{\partial r}$ , вона дає
	\[
		- \left. {\frac{{\partial {\pot_\mathrm{out}}}}{{\partial r}}} \right|_{r = R} + \varepsilon {\left. {\frac{{\partial {\pot_\mathrm{in}}}}{{\partial r}}} \right|_{r = R}} = 4\pi \sum\limits_{n = 0}^\infty  {\sum\limits_{m =  - n}^n {{s_{nm}}{Y_{nm}}(\theta ,\varphi )} }.
	\]
	Підстановка~\eqref{dstar_prb:L}, \eqref{ddstar_prb:L} з урахуванням  $\tilde{A}_{nm} = B_{mn}$ визначає ці коефіцієнти:
	\[
		\left( n + 1 + n\varepsilon  \right)B_{nm} = 4\pi s_{nm}.
	\]
	Отже, остаточно маємо:
	\[
		\pot(r,\theta,\phi) =
		\begin{cases}
			4\pi \sum\limits_{n = 0}^\infty  {\sum\limits_{m =  - n}^n {\frac{{{s_{nm}}}}{{n + 1 + n\varepsilon }}{{\left( {\frac{r}{R}} \right)}^n}{Y_{nm}}(\theta ,\varphi )} } , \quad r < R \\
			4\pi \sum\limits_{n = 0}^\infty  {\sum\limits_{m =  - n}^n {\frac{{{s_{nm}}}}{{n + 1 + n\varepsilon }}{{\left( {\frac{R}{r}} \right)}^{n + 1}}{Y_{nm}}(\theta ,\varphi )} } , \quad r \ge R.
		\end{cases}
	\]

\end{Solution}
\begin{Solution}{6.{22}}
    $\Hfield_2 = \Hfield_1 + \left( \frac{\mu_1}{\mu_2} - 1 \right)\vect{n}\left( \vect{n} \cdot \Hfield_1 \right) + \frac{4\pi}{c}\left[\vect{n}\times\vect{i}\right] $.
\end{Solution}
\begin{Solution}{6.{24}}
	$
\vect{A}(r) =
\begin{cases}
-\mu a (r - R) \vect{e}_z, & r \leqslant R, \\
-\mu a R \ln \left( \dfrac{r}{R} \right) \vect{e}_z, & r > R.
\end{cases}
	$
\end{Solution}
\begin{Solution}{6.{26}}
	$F = \frac{\epsilon_2 - \epsilon_1}{(\epsilon_2 + \epsilon_1)\epsilon_1} \frac{q}{4d^2}$.
\end{Solution}
\begin{Solution}{6.{27}}
	В діелектрику виникає зображення диполя, з моментом $\vect{p}' = -\vect{p}\frac{\epsilon - 1}{\epsilon + 1}$.

	Сумарний момент сил, що діє на диполь
    \[
        \left[\vect{p}\times\Efield\right] + \left[\vect{p}\times\left( \frac{3(\vect{p}'\vect{r})\vect{r}}{r^5} - \frac{\vect{p}'}{r^3} \right) \right] = 0
    \]
Звідси для положень рівноваги маємо $\alpha=0$ та $\alpha = \pi$ . При $\xi = \left|\frac{E}{\frac{\epsilon - 1}{\epsilon + 1} \frac38\frac{p}{d^3}}\right|<1$  також маємо $\alpha = \pi \pm \arccos\xi$.
Положення рівноваги $\alpha = 0$  є стійким. Положення рівноваги $\alpha = \pi$ є нестійким при при $\xi >1$ , але стає стійким при $\xi < 1$. Положення, що відповідають $\alpha = \pi \pm \arccos\xi$  при $\xi <1$, є нестійкими.
\end{Solution}
\begin{Solution}{6.{28}}
	$\phi = \sqrt{8mg/(\epsilon-1)}$.
\end{Solution}
\begin{Solution}{6.{29}}
	$\sigma_\text{зв'яз} = - \frac{\epsilon - 1}{\epsilon + 1} \frac{qd}{2\pi(x^2  +h^2)^{3/2}}$, де $x$~-- координата вздовж межі розділу, $q_\text{зв'яз} = - \frac{\epsilon - 1}{\epsilon + 1}q$.
\end{Solution}
\begin{Solution}{6.{30}}
	$F_1 = \frac{\epsilon_1 - \epsilon_2}{(\epsilon_1 + \epsilon_2)\epsilon_1} \frac{q_1^2}{4d^2} + \frac{2}{\epsilon_1 + \epsilon_2} \frac{q_1q_2}{4d^2}$,
	$F_2 = \frac{\epsilon_2 - \epsilon_1}{(\epsilon_1 + \epsilon_2)\epsilon_2} \frac{q_2^2}{4d^2} + \frac{2}{\epsilon_1 + \epsilon_2} \frac{q_1q_2}{4d^2}$.
\end{Solution}
