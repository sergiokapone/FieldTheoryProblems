% !TeX program = lualatex
% !TeX encoding = utf8
% !TeX spellcheck = uk_UA


\documentclass[]{ProblemBook}

\def\Dolgoshey{D\raisebox{-0.5ex}{o}\raisebox{0.5ex}{l}}

\begin{document}

%=========================================================
\begin{problem}
    Чому птахи літають ключами $c_{sdfg}$?
	\begin{equation}\label{key}
		\left[\Efield\times\Bfield\right]^2 \LaTeXe
	\end{equation}
\begin{solution}
	Із-за аеродинаміки повітря.
		$\overline{S_1S_2}$
\end{solution}
\end{problem}

\begin{center}
\textbf{Рівняння Пуассона і Лапласа. Розділення змінних в декартових, сферичній і циліндричній системах координат.}
\end{center}

\begin{problem}
    Площина $z$ = 0 заряджена з поверхневою густиною зарядів $σ(x,y) = σ_{0}sin(αx)\cdot \sin(βy)$, де $σ_{0}$ , $α$ і  $β$ – постійні. Знайти потенціал цієї системи зарядів?

	\begin{solution}
	Відповідь: Рішення очікується.
		$\overline{S_1S_2}$
\end{solution}
\end{problem}

\begin{problem}
    Металічна труба, безкінечно уздовж осі \textit{oz} , має прямокутний переріз аxb. Одна із сторін має потенціал $φ_{0}$, а інша – потенціал 0. Знайти розподіл потенціалу всередині труби.
	
\begin{solution}
	Відповідь: Очікується.
		$\overline{S_1S_2}$
\end{solution}
\end{problem}

\begin{problem}
    Знайти потенціал між двома безкінечними площинами, якщо потенціал площини \textit{х} = 0 рівний $Аsin(βy)$, а потенціал площини \textit{x} = d рівний $φ0 + Bsin(βy)$. 
	
\begin{solution}
	Відповідь Очікується.
		$\overline{S_1S_2}$
\end{solution}
\end{problem}


\begin{problem}
   Незаряджена металічна сфера радіуса \textit{а} розміщена у однорідному електричному полі напруженістю E$_{0}$ . Визначити результуюче поле і густину поверхневих зарядів на сфері.
	
\begin{solution}
	Відповідь:
		\[\ \textbf{E} = \left\{ {\begin{array}{*{20}{c}}
{{{\ \textbf{E}}_0} + \frac{{3\left( {\ \textbf{d} \cdot \ \textbf{R}} \right)\ \textbf{R}}}{{{R^5}}} - \frac{{\ \textbf{d}}}{{{R^3}}}}\\
{0\;\;,\;\;\;\;\;\;\;\;\;\;\;\;\;\;\;\;\;\;\;\;\;\;\;\;\;\;\;R < a}
\end{array}} \right.\;,\;\;\;\;\;R > a\]

\[\sigma \left( \theta  \right) = \left. {\frac{1}{{4\pi }}{E_R}} \right|\begin{array}{*{20}{c}}
{}\\
{R = a}
\end{array} = \frac{3}{{4\pi }}{E_0}\cos \left( \theta  \right)\]
Тут \textbf{d} - дипольний момент, набутий сферою в однорідному електричному полі $E_{0}$.
\end{solution}
\end{problem}


\begin{problem}
    По якому закону повинна бути розподілена густина заряду \textit{ρ}(r) всередині циліндра радіуса \textit{R}, щоб напруженість електричного поля \textit{Е} всередині циліндра була постійною за величиною і рівною \textbf{Е}$_{0}$? Який буде розподіл потенціалу при цьому?

	\begin{solution}
	Відповідь: 
$\rho  = \frac{{{E_0}}}{{4\pi r}}$

$\begin{array}{*{20}{c}}
{{\varphi _1} =  - {E_0}r\;,\;\;\;\;r \le R}\\
{{\varphi _2} =  - {E_0}R\left( {\ln \frac{r}{R} + 1} \right)\;,\;\;\;\;r > R}
\end{array}$
\end{solution}
\end{problem}


\begin{problem}
    Безкінечний порожнистий циліндр радіуса $R$ заряджений з поверхневою густиною заряду $\sigma = \sigma_{0}\cos(2\alpha)$, де $\alpha$ --- полярний кут циліндричної системи координат з віссю $OZ$, напрямленою уздовж осі циліндра. Знайти розподіл потенціалу $\phi$ та напруженість $R$ електричного поля всередині та ззовні циліндра. 
\begin{solution}
\begin{align*}
    \phi_1 &= \pi\sigma_0 R \left( \frac{r}{R} \right)^2\cos 2\alpha,  \quad r \le R  \\
    \phi_2  &= \pi\sigma_0R\left(\frac{R}{r}  \right)^2\cos 2\alpha, \quad r \ge R
    \end{align*}

\begin{align*}
    E_\text{in}  &= 2\pi\sigma_0 R  \frac{\vect{r}}{R} ,  \quad r \le R  \\
    E_\text{out} &= 2\pi\sigma_0 R \left( \frac{R}{r} \right)^3,  \quad r \le R 
\end{align*}

\end{solution}
\end{problem}
	
\textbf{\begin{center}
\begin{center}
Метод зображень. Площина, сфера.
\end{center}
\end{center}}

\begin{problem}
Точковий заряд \textit{q} знаходиться у вакуумі на відстані \textit{а} від плоскої границі безкінечно протяжного провідника. Знайти потенціал φ, напруженість електричного поля \textit{Е}, розподіл поверхневої густини заряду σ та повний індукований заряд \textit{Q} на металі, а також силу \textit{F}, що діє на заряд.    

	\begin{solution}
	Відповідь: 
\[\varphi  = q\left( {\frac{1}{r} - \frac{1}{{r'}}} \right),\;\;\;\;\;{\rm{}}\;\;\;z \ge 0\]

\[\begin{array}{*{20}{c}}
{\ \textbf{E} = \frac{{q \cdot \ \textbf{r}}}{{{r^3}}} - \frac{{q \cdot \  \textbf{r'}}}{{{{\left( {r'} \right)}^3}}},\;\;\;\;\;{\rm{}}\;\;\;z > 0}\\
{\ \textbf{E} = 0,\;\;\;\;\;{\rm{}}\;\;\;z < 0}
\end{array}\]
Тут \textit{r} - відстань від заряду до точки спостереження; \textit{r'} - відстань від зображення заряду до тієї ж точки спостереження.
\[\ \textbf{F} =  - \frac{{{q^2}}}{{4{a^2}}}\frac{{\ \textbf{z}}}{z}\]
\[\sigma \left( \theta  \right) = \left. {\frac{1}{{4\pi }}{E_z}} \right|\begin{array}{*{20}{c}}
{}\\
{z = 0}
\end{array} =  - \frac{q}{{2\pi {a^2}}}{\cos ^3}\left( \theta  \right)\]

\[Q =  - \int\limits_0^{\frac{\pi }{2}} {\frac{q}{{2\pi {a^2}}}} {\cos ^3}\left( \theta  \right) \cdot 2\pi {a^2}\frac{{\sin \left( \theta  \right)}}{{{{\cos }^3}\left( \theta  \right)}}d\theta  =  - q\]

\end{solution}
\end{problem}

\begin{problem}
    Заряд $q$ знаходиться на відстані $l$ від провідної ізольованої сфери радіуса $a < l$ і зарядом $Q$. Знайти силу взаємодії заряду із сферою. При якому значенні заряду на сфері ця сила буде рівна нулю?
	\begin{solution}
\[\vect{F} = \left[ \frac{Qq}{l^2} - \frac{q^2a^3( 2l^2 - a^2)}{l^3{( l^2 - a^3 )}^2} \right]\frac{\vect{l}}{l}\]

Cфера відштовхує заряд $q$, якщо $Q > q\frac{{{a^3}\left( {2{l^2} - {a^2}} \right)}}{{l{{\left( {{l^2} - {a^2}} \right)}^2}}}$

Cфера притягує заряд $q$, якщо $Q < q\frac{{{a^3}\left( {2{l^2} - {a^2}} \right)}}{{l{{\left( {{l^2} - {a^2}} \right)}^2}}}$

Сила, що діє з боку заряду на сферу, рівна нулю, якщо $Q = q\frac{{{a^3}\left( {2{l^2} - {a^2}} \right)}}{{l{{\left( {{l^2} - {a^2}} \right)}^2}}}$

\end{solution}
\end{problem}

\begin{center}
\textbf{Електростатика в середовищі}
\end{center}

\begin{problem}
 Знайти силу, що діє на малий заряд $q$, розміщений в безкінечно вузькій щілині в діелектрику з проникністю $ε$, якщо діелектрик знаходиться у зовнішньому електричному полі $\Efield$ так, що вісь щілини утворює кут $α$ з напрямком зовнішнього поля.   
	\begin{solution}
	\[F = qE\varepsilon \sqrt {{{\sin }^2}\alpha  + \frac{{{{\cos }^2}\alpha }}{{{\varepsilon ^2}}}} \] 
\end{solution}
\end{problem}

\begin{problem}
    Точковий заряд $q$ розміщений на плоскій границі розділу двох однорідних безкінечних діелектриків з проникностями $ε_{1}$ і $ε_{2}$. Знайти напруженість і індукцію електричного поля, а також його потенціал. 
	\begin{solution}
\[\varphi  = \frac{2}{{{\varepsilon _1} + {\varepsilon _2}}}\frac{q}{R}\];

\[\Efield = \frac{{2q}}{{{\varepsilon _1} + {\varepsilon _2}}}\frac{{\vect{R}}}{{{R^3}}}\];

\begin{center}
${\Dfield_1} = \frac{{2{\varepsilon _1}}}{{{\varepsilon _1} + {\varepsilon _2}}}\frac{{q\vect{R}}}{{{R^3}}}$  при $z$<0;

\end{center}
\begin{center}
${\Dfield_2} = \frac{{2{\varepsilon _2}}}{{{\varepsilon _1} + {\varepsilon _2}}}\frac{{q\vect{R}}}{{{R^3}}}$  при $z$>0;
\end{center}
\end{solution}
\end{problem}

\begin{problem}
    Центр провідної сфери радіуса $R$, що заряджена зарядом $q$, знаходиться на плоскій границі розділу двох безкінечно однорідних діелектриків з проникностями $ε_{1}$ та $ε_{2}$ відповідно. Знайти потенціал електричного поля, а також розподіл вільних і зв’язаних зарядів на поверхні сфери.
	\begin{solution}
	\[{\varphi _{1,2}} = \frac{{2q}}{{{\varepsilon _1} + {\varepsilon _2}}}\frac{1}{r}\]
    \[\sigma_{\text{вільн}}^{1,2} = \frac{{q{\varepsilon _{12}}}}{{2\pi {R^2}({\varepsilon _1} + {\varepsilon _2})}}\]
    \[\sigma_{\text{зв'яз}}^{1,2} = \frac{{q(1 - {\varepsilon _{12}})}}{{2\pi {R^2}({\varepsilon _1} + {\varepsilon _2})}}\]
\end{solution}
\end{problem}

\begin{problem}
  Від прямої, на якій знаходиться точковий заряд q, розходяться віялоподібно три півплощини, що утворюють три двогранні кути $α_{1}$, $α_{2}$, $α_{3}$, причому $α_{1} + α_{2} + α_{3} = 2π$. Простір всередині кожного із кутів заповнений однорідним діелектриком з проникністю відповідно $ε_{1}$, $ε_{2}$, $ε_{3}$. Визначити потенціал, напруженість і індукцію електричного поля у трьох областях.
	\begin{solution}
	\[{\varphi _i} = \frac{2\pi q}{{{\varepsilon _1}{\alpha _1} + {\varepsilon _2}{\alpha _2} + {\varepsilon _3}{\alpha _3}}}\frac{1}{r}\]
 \[\Efield_{i} =\frac{2\pi q}{{{\varepsilon _1}{\alpha _1} + {\varepsilon _2}{\alpha _2} + {\varepsilon _3}{\alpha _3}}}\frac{\vect{r}}{r^{3}}\]
\[\vect{D_{i}}=\Efield_{i}\]
тут $i=1,2,3.$
\end{solution}
\end{problem}

\begin{problem}
  Однорідна сфера радіуса $a$ з діелектричною проникністю $ε_{1}$ занурена в однорідний необмежений діелектрик $ε_{2}$. На великій відстані від сфери в діелектрику знаходиться однорідне електричне поле $\Efield_{0}$. Знайти потенціал $\varphi$ і напруженість електричного поля $\Efield$ в усьому просторі, а також розподіл зв’язаних зарядів $\sigma_{\text{зв'яз}}$ на сфері і її набутий дипольний момент $\vect{d}$.
	\begin{solution}
\[\vect{d}=\frac{\varepsilon_{1}-\varepsilon_{2}}{\varepsilon_{1}+2\varepsilon_{2}}a^3{\Efield_{0}}\]

\[\varphi_{1}=-\frac{3\varepsilon_{2}}{{\varepsilon _1} + 2{\varepsilon _2}}{{E_{0}}z} \;\;\;\;\;\text{при} \;\;\;\;\;R \le a\]

\[\varphi_{2}=-{E_{0}}z+\frac{(\vect{d}\cdot\vect{R})}{R^{3}}=-{E_{0}R}\cos\theta+a^3\frac{\varepsilon_{1}-\varepsilon_{2}}{\varepsilon_{1}+2\varepsilon_{2}}\cdot\frac{E\cos\theta}{R^2}\;\;\;\;\;\text{при} \;\;\;\;\;R \ge a\]

\[\Efield_{1}=-\frac{3\varepsilon_{2}}{\varepsilon_{1}+2\varepsilon_{2}}\Efield_{0}\;\;\;\;\;\text{при} \;\;\;\;\;R < a\]

\[\Efield_{2}=\Efield_{0}-\frac{\vect{d}}{R^3}+\frac{3(\vect{d}\cdot\vect{R})\vect{R}}{R^5}\;\;\;\;\;\text{при} \;\;\;\;\;R > a\]
\[\sigma_{\text{зв'яз}}=\frac{3}{4\pi}\cdot\frac{\varepsilon_{1}-\varepsilon_{2}}{\varepsilon_{1}+2\varepsilon_{2}}\cdot\Efield_{0}\cos\theta\]
\end{solution}
\end{problem}

Метод зображень на границі "діелектрик-діелекрик"

\begin{problem}
    Точковий заряд $q$ знаходиться на відстані $h$ від плоскої границі поділу двох безкінечно протяжних однорідних діелектриків з проникностями $\varepsilon_{1}$ і $\varepsilon_{2}$ (заряд знаходиться в діелектрику з $\varepsilon_{1}$). Знайти потенціал електричного поля.
	\begin{solution}

\[\varphi_{1}=\frac{q}{\varepsilon_{1}r_{1}}+\frac{q}{\varepsilon_{1}}\cdot\frac{\varepsilon_{1}-\varepsilon_{2}}{\varepsilon_{1}+\varepsilon_{2}}\cdot\frac{1}{r_{2}}\]

\[\varphi_{2}=\frac{1}{\varepsilon_{2}}\cdot\frac{2\varepsilon_{2}}{\varepsilon_{1}+\varepsilon_{2}}\cdot\frac{q}{r_{1}}\]
\end{solution}
\end{problem}

\begin{problem}
   Знайти поверхневу густину $\sigma_{\text{зв’яз}}$ зв’язаних зарядів, індукованих на плоскій границі поділу двох однорідних діелектриків $\varepsilon_{1}$ і $\varepsilon_{2}$, точковим зарядом $q$, що знаходиться на відстані $a$ над цією границею (заряд в діелектрику з діелектричною проникністю $\varepsilon_{1}$ ).  Яким буде результат, якщо $\varepsilon_{2}$ → ∞ ?
	\begin{solution}
	\[\sigma_{\text{зв'яз}}=\frac{q}{\varepsilon_{1}}\cdot\frac{\varepsilon_{1}-\varepsilon_{2}}{\varepsilon_{1}+\varepsilon_{2}}\cdot\frac{h}{2\pi(h^2+R^2)^{3/2}}\]
при $\varepsilon_{2}$ → ∞ :
\[\sigma_{\text{зв'яз}}=-\frac{qh}{2\pi\varepsilon_{1}(h^2+R^2)^{3/2}}\]
\end{solution}
\end{problem}

\begin{problem}
  Напівпростори заповнені діелектриком: верхній з проникністю $\varepsilon_{1}$, нижній - $\varepsilon_{2}$. На осі, перпендикулярній до площини поділу, розміщені три заряди $q_{1}$, $q_{2}$ і $q_{3}$. В початку координат розміщений заряд $q_{2}$, а $q_{1}$ і $q_{3}$ – симетрично на відстані $a$ від заряду $q_{2}$. Знайти силу, що діє на заряд $q_{1}$.   
	\begin{solution}
	\[\vect{F}=\left[\frac{2q_{2}q_{1}}{a^2(\varepsilon_{2}+\varepsilon_{1})}+\frac{q_{3}q_{1}}{2a^2(\varepsilon_{2}+\varepsilon_{1})}-\frac{q_{1}^2(\varepsilon_{2}-\varepsilon_{1})}{4a^2\varepsilon_{1}(\varepsilon_{2}+\varepsilon_{1})}\right]\frac{\vect{z}}{z}.\]
\end{solution}
\end{problem}

Ємність

\begin{problem}
Оцінити ємність: а) металічної пластини розмірами $h$ \ll $a$  $\ll$  $l$ ;  б) циліндра з радіусом $a$ і довжиною $l$, причому $a$ $\ll$  $l$.     
	\begin{solution}
\[\text{а) для пластини:}\]
\[C=\frac{l}{1+2\pi+2ln\frac{l}{a}}\sim\frac{l}{2\left(\pi+ln\frac{l}{a}\right)}\]

\[\text{б) для циліндра:}\]
\[C=\frac{l}{1+2ln\frac{l}{a}}\sim\frac{l}{2ln\frac{l}{a}}\]
\end{solution}
\end{problem}

\begin{problem}
    Плоский конденсатор заповнений діелектриком, проникність якого змінюється за законом: $\varepsilon=\varepsilon_{0}(x+a)/a$, де $a$ – відстань між обкладками, вісь $ox$ направлена перпендикулярно до обкладок, площа яких $S$. Нехтуючи крайовими ефектами, знайти ємність такого конденсатора і розподіл у ньому зв’язаних зарядів, якщо до обкладок прикладена різниця потенціалів $U$.	
	\begin{solution}
	\[C=\frac{\varepsilon_{0}S}{4\pi{a}\cdot\ln2}\] 

\[\left. \sigma_{\text{зв'яз}}  \right|_{x = 0}=-\frac{CU}{S}\left(1-\frac{1}{\varepsilon_{0}}\right)\]
\[\left. \sigma_{\text{зв'яз}}  \right|_{x = a}=\frac{CU}{S}\left(1-\frac{1}{2\varepsilon_{0}}\right)\]

\end{solution}
\end{problem}


\begin{problem}
   Всередині сферичного конденсатора з радіусами обкладок $a$ і $b$ діелектрична проникність змінюється по закону: 

\[\varepsilon (r) = \left\{ {\begin{array}{*{20}{c}}
{\varepsilon _1 = const\;\;\;\;a \le r < c}\\
{\varepsilon _2 = const\;\;\;\;c \le r \le b}
\end{array}} \right.\]

де $a$ < $b$ < $c$. Знайти ємність конденсатора, розподіл зарядів $\sigma_{\text{зв’яз}}$ і повний зв’язаний заряд в діелектрику.

	\begin{solution}
    \[C = {\left[ {\frac{1}{\varepsilon _1}\left( {\frac{1}{a} - \frac{1}{c}} \right) + \frac{1}{\varepsilon _2}\left( {\frac{1}{c} - \frac{1}{b}} \right)} \right]^{ - 1}}\]

    \[\sigma_{\text{зв’яз}}(a) =\frac{E_{1n}-D_{1n}}{4\pi} =-\frac{Q_{1}}{4\pi a^2}\left( 1 - \frac{1}{\varepsilon _1} \right)\]

\[\sigma_{\text{зв’яз}}(c)=\frac{E_{c+}-E_{c-}}{4\pi}=\frac{Q_1}{4\pi c^2}\left(\frac{1}{\varepsilon_2}-\frac{1}{\varepsilon_1}\right)\]

\[\sigma_{\text{зв’яз}}(b) =\frac{Q_1}{4\pi b^2}\left(1-\frac{1}{\varepsilon_2}\right)\]

де $Q_{1}$ – заряд внутрішньої обкладки.
\end{solution}
\end{problem}

\begin{problem}
    Простір між обкладками сферичного конденсатора частково заповнено діелектриком, розміщеним всередині тілесного кута $\Omega$ з вершиною в центрі обкладок. Радіуси обкладок – $a$ і $b$, проникність діелектрика – $\varepsilon$. Знайти ємність конденсатора.
	\begin{solution}
\[C=\frac{ab}{b-a}\left\lbrace \frac{S_{\text{сфери}}-S_{\text{кут}}}{S_{\text{сфери}}}+\varepsilon\frac{S_{\text{кут}}}{S_{\text{сфери}}}\right\rbrace =\frac{ab}{b-a}\left\{1-\frac{\Omega}{4\pi}+\varepsilon\frac{\Omega}{4\pi}\right\}\]
\end{solution}
\end{problem}

\begin{problem}
    Знайти взаємну ємність двох сфер радіуса а, якщо відстань між їх центрами рівна $b \gg 2a$
	\begin{solution}
\[C\simeq\frac{a}{2}\left(1+\frac{a}{b}\right)\]
\end{solution}
\end{problem}

\begin{problem}
   Всередині плоского конденсатора, зарядженого до напруги $U$, на відстані $h$ від однієї із пластин знаходиться маленька металічна куля радіуса $r$. Нехтуючи викривленням поля конденсатора, знайти заряд, що з’явився на кулі, якщо з’єднати його з пластиною з нульовим потенціалом. Відстань між пластинами $d$. 
	\begin{solution}
\[Q =  - \frac{Uhr}{d}\]
\end{solution}
\end{problem}

Енергія поля. Тиск. Сили

\begin{problem}
  Знайти енергію електростатичного поля зарядженого рівномірно по об’єму кулі через густину енергії і через густину заряду і потенціал. Заряд кулі $Q$, радіус $R$.
	\begin{solution}
\[W = \frac{3}{5}\frac{Q^2}{R}\]
\end{solution}
\end{problem}

\begin{problem}
   Диполь з моментом $\vect{p_1}$ знаходиться в початку координат, а інший диполь з моментом $\vect{p_2}$ - в точці з радіус-вектором $\vect{r}$. Знайти енергію взаємодії цих диполів і діючу між ними силу. При якій орієнтації диполів ця сила максимальна?
	\begin{solution}
\[U(\vect{r})=\frac{p_2p_1}{r^3}\left(sin\theta_1 sin\theta_2 cos\phi-2cos\theta_1 cos\theta_2\right)\]
де $\theta_1 = \widehat {\vect{r}\vect{p_1}}$ , $\theta_2 = \widehat {\vect{r}\vect{p_2}}$ , $\phi$ --- кут між площинами, побудованими на векторах  $\vect{p_1},\vect{r}$ і $\vect{p_2},\vect{r}$ відповідно.
\[F_{max}(r)=\frac{6p_1p_2}{r^4}\]
\end{solution}
\end{problem}

\begin{problem}
   Електричний диполь з моментом $\vect{p}$ знаходиться в однорідному діелектрику поблизу плоскої границі безкінечно протяжного провідника. Знайти потенціальну енергію взаємодії диполя з індукованими зарядами, силу і обертовий момент, прикладений до диполя. Відстань $a$, діелектрична проникність $\epsilon$. 
	\begin{solution}

\[U =-\frac{p^2}{8a^3\varepsilon}\left(1 + \cos ^2\theta\right)\]

\[W =-\frac{p^2}{16a^3\varepsilon}\left(1 + \cos^2\theta \right)\]

де
\[\theta = \left({\widehat {\vect{p},\vect{e_z}}}\right)\]

\[{F_z} = -\frac{3W}{a}\]

\[{N_\theta } =  -\frac{p^2\sin2\theta}{16\varepsilon a^3}\]

\end{solution}
\end{problem}

\begin{problem}
   Електричний диполь $\vect{p_0}$ знаходиться в однорідному діелектрику на відстані $r$ від центра заземленої провідної сфери радіуса $R$. Знайти енергію взаємодії диполя із сферою, силу і обертовий момент, прикладені до диполя. Розглянути випадок $r \to R (r > R)$.
	\begin{solution}
\[W=-\frac{{p^2_0}R\left(R^2+r^2\cos^2\theta\right)}{2\varepsilon\left(r^2-R^2\right)^3}\]
\[F_r=-\frac{p^2_0Rr}{\varepsilon}\frac{\left(R^2+2r^2\right)cos^2\theta+3R^2}{\left(r^2-R^2\right)^4}\]
\[N_{\theta}=-\frac{p^2_0Rr^2sin 2\theta}{2\varepsilon\left(r^2-R^2\right)^3}\]

де
\[\theta = \left({\widehat {\vect{p},\vect{r}}}\right)\]
\end{solution}
\end{problem}

\begin{problem}
 Плоский конденсатор (підключений до батареї з ЕРС $E$) з вертикально розміщеними пластинами опущений в рідкий діелектрик з діелектричною проникністю $\epsilon$. Густина рідини $\rho$, відстань між пластинами $d$. На яку висоту підніметься рідина всередині конденсатора?
	\begin{solution}
\[h = \frac{\left({\varepsilon-1}\right)E^2}{8\pi \rho gd^2}\]
\end{solution}
\end{problem}

\begin{problem}
 Знайти переріз розсіяння на малі кути електронів (заряд – $e$, маса – $m$, швидкість на безмежності – $υ_0$), що пролітають з великим прицільним параметром $\rho$ повз сферу радіуса $a$, якщо: а) сфера провідна і заземлена; б) сфера провідна і ізольована; в) сфера діелектрична з проникністю $\varepsilon$; г) сфера діелектрична з поляризованістю $\sim E^2$.  
	\begin{solution}
	
а) провідна заземлена сфера: $\frac{d\sigma}{d\Omega} \approx \frac{\pi ae^2}{4mυ_0^2} \cdot \frac{1}{\theta ^3}$

б) провідна ізольована сфера: $\frac{d\sigma}{d\Omega} \approx \frac{\pi a^2}{8} \cdot \sqrt {\frac{3e^2}{a\pi mυ_0^2}}\cdot\theta ^{-5/2}$

в) діелектрична сфера із проникністю $\varepsilon$: $\frac{d\sigma }{d\Omega} = \frac{\pi a^2}{8} \cdot \left( \frac{3}{4}\frac{e^2}{am\upsilon _0^2} \cdot \frac{\varepsilon  - 1}{\varepsilon  + 2} \right)^{1/2} \cdot \theta ^{-5/2}$
\end{solution}
\end{problem}

{\bf {Протікання струму. Закон збереження заряду:}}
Протікання струму. Закон збереження заряду:
\begin{problem}
 В безмежному середовищі з провідністю $\sigma$ і проникністю $\varepsilon$  розміщений заряд $q$. Знайти час релаксації $\tau$, тобто час, протягом якого заряд в цій точці зменшиться в $e$ разів. 
	\begin{solution}
	\[\tau  = \frac{\varepsilon }{4\pi \sigma }\]
\end{solution}
\end{problem}

\begin{problem}
 В неоднорідному провідному середовищі з провідністю $\sigma\left(\vect{r}\right)$ і діелектричною проникністю $\varepsilon\left(\vect{r}\right)$   підтримується розподіл струмів $\vect{j}\left(\vect{r}\right)$. Знайти об’ємний розподіл зарядів $\rho\left(\vect{r}\right)$ у цьому середовищі.
	\begin{solution}
	\[\rho \left( \vect {r} \right) = \frac{\left( {\vect {j} \cdot grad\sigma } \right)}{4\pi \sigma^2}\]
\end{solution}
\end{problem}

\begin{problem}
 Знайти закон заломлення ліній току (вектора густини струму) на плоскій поверхні поділу двох середовищ із провідностями $\sigma_{1}$ і $\sigma_{2}$.
	\begin{solution}
\[\frac{tg\alpha_1}{tg\alpha _2} = \frac{\sigma _1}{\sigma _2}\]	
\end{solution}
\end{problem}

\begin{problem}
 Простір між безкінечно довгими коаксіальними ідеально провідними циліндрами радіусів $a$ і $b$ заповнений середовищем із провідністю $\sigma(r)=\alpha\cdot r^n$. Знайти розподіл потенціалів у просторі між циліндрами і опір на одиницю довжини. Потенціали циліндрів: $U(a) = 0$, $U(b) = U_0$. 
	\begin{solution}
\[\varphi (r) = {U_0}\frac{\left( \frac{b}{r}\right)^n - 1}{\left( {\frac{b}{a}} \right)^n - 1}\]
\[R=\frac{\left(\frac{b}{a}\right)^n-1}{2\pi \alpha b^n}\]	
\end{solution}
\end{problem}

\begin{problem}
 Знайти стаціонарне поле $E$ в плоскому конденсаторі з напругою $U$, діелектрик якого складається із двох шарів товщини $l_{1}$, $l_{2}$ з діелектричними постійними $\varepsilon_1$, $\varepsilon_2$ і провідностями $\sigma_1$, $\sigma_2$. Визначити вільний і зв’язаний заряди на межі поділу середовищ.
	\begin{solution}
\[E_{1,2} = \frac{\sigma _{2,1}U}{l_1\sigma_2 + l_2\sigma_1}\]
\[\sigma_{\text{вільн}} = \frac{U(\sigma_1\varepsilon_2 - \sigma_2\varepsilon_1)}{4\pi ({l_1}{\sigma _2} + {l_2}{\sigma _1})}\]
\end{solution}
\end{problem}

{\bf \textcolor[rgb]{0.6,0.0,0.1}{Нові задачі:}}

{\bf {Протікання струму. Закон збереження заряду:}}

\begin{problem}
 В безкінечному середовищі з провідністю $\sigma$, де проходив струм густиною $j_0$, виникла сферична порожнина радіуса $R$ (всередині порожнини $\sigma = 0$). Знайти результуючий розподіл струмів $\vect{j(r)}$.
	\begin{solution}

\[\vect{j}(r) = \left\{ {\begin{array}{*{20}{c}}
{\vect{j}_0}\left(1+\frac{R^3}{2r}\right)-\frac{3R^3\left(\vect{j}_0\cdot\vect{r}\right)\vect{r}}{2r^5}\;\;\;\;\;\text{при} \;\;\;\;\; r > R\\
{0} \;\;\;\;\;\text{при} \;\;\;\;\;r < R
\end{array}} \right.\]	
\end{solution}
\end{problem}

\begin{problem}
 Суцільний безкінечно довгий циліндр радіуса $a$ і провідністю $\sigma_1$ знаходиться в однорідному провіднику з провідіністю $\sigma_2$. Всередині циліндра діє стороннє однорідне поле напруженістю $\vect{E}_{\text{стор}}$, направлене перпендикулярно осі циліндра. Знайти розподіл струму в усьому просторі.
	\begin{solution}
\[\left\{ {\begin{array}{*{20}{c}}
{\vect{j}_1 =\frac{\sigma_2}{\sigma_1+\sigma_2}\vect{E}_{\text{стор}}}\\{\vect{j}_2 =-\frac{\sigma_1\sigma_2 a^2}{\sigma_1+\sigma_2}\left( {\frac{\vect{E}_\text{стор}}{r^2}-2\frac{\left(\vect{E}_\text{стор}\cdot\vect{r}\right)\vect{r}}{r^4}} \right)}
\end{array}} \right.\]
\end{solution}
\end{problem}

\begin{problem}
 Знайти вольт-амперну характеристику плоского діода (площа електродів --- $S$, відстань між ними --- $d$), катод якого необмежено випускає електрони з нульовою початковою швидкістю (закон "3/2"). Вважати, що електричне поле біля катода відсутнє (зовнішнє поле самого діода компенсується полем об’ємного заряду, що утворився між електродами). 
	\begin{solution}
	\[j=\frac{\sqrt2}{9\pi}\sqrt{\frac{e}{m}}\frac{S}{d^2}\cdot U^{3/2}\]
\end{solution}
\end{problem}

\begin{problem}
 Узагальнити із попередньої задачі закон "3/2" на область ультрарелятивістських енергій
	\begin{solution}
\[j=\frac{cU}{2\pi d^2}\]	
\end{solution}
\end{problem}

\begin{problem}
 Знайти вольт-амперну характеристику циліндричного діода із нульовим радіусом катода та радіусом анода, рівним $a$.
	\begin{solution}
\[j=\frac{2\sqrt2}{9}\cdot\frac{l}{a}\sqrt{{\frac{e}{m}}}U^{3/2}\]	
\end{solution}
\end{problem}

{\bf {Магнетики:}}

\begin{problem}
    Знайти величину магнітного поля на осі рівномірно зарядженого диска із зарядом $q$ і радіусом $a$, що обертається навколо осі з кутовою швидкістю $\omega$ на відстані $h$ від диска. 
	\begin{solution}
\[H_z \left(h\right) = \frac{2\omega q}{ca^2}\left\{{\frac{2h^2+a^2}{\sqrt{h^2+a^2}}-2h}\right\}\] 
\end{solution}
\end{problem}

\begin{problem}
Визначити магнітне поле, що створюється двома паралельними площинами, уздовж яких течуть струми з однаковими поверхневими густинами $i = const$. Розглянути випадки: а) струми течуть в протилежних напрямках ;  б) струми однаково напрямлені.    
	\begin{solution}
a)	$H=\frac{4\pi i}{c}$ --- між площинами ; $H=0$ --- поза їх межами
б) $H=0$ --- між площинами ;  $H=\frac{4\pi i}{c}$ --- поза їх межами.
Поле $\Hfield$ напрямлене уздовж площин і перпендикулярно струму. 
\end{solution}
\end{problem}

{\bf \textcolor[rgb]{0.6,0.0,0.1}{Нові задачі:}}

{\bf {Магнетостатика, магнетики:}}

\begin{problem}
 Знайти векторний потенціал: 1) однорідного поля в координатах:
а) декартових ; б) циліндричних ; в) сферичних
2) поля прямого струму;  3) поля колового витка із струмом на великих відстанях від витка.   
	\begin{solution}

1. a) Декартова система координат: 
\[A_x = -\frac{1}{2}B_z\cdot y ,\;\;\; A_y=-\frac{1}{2}B_z\cdot x ,\;\;\; A_z = 0\]
б) Циліндрична система координат:
\[A_{\alpha} = -\frac{1}{2}B_{\alpha}\cdot r\]
в) Сферична система координат:
\[A_R = A_\theta = 0,\;\;\;  A_\alpha = \frac{1}{2}BR\sin\theta\]

2. Поле прямого струму:
\[A_z = -\frac{2I}{c}\ln r + const\]

3. Поле колового витки із струмом:
\[\vect{A}\left(\vect{R}\right)=\frac{\left[\vect{m} \times \vect{R}\right]}{R^3}\]
$\;\;\;\;\;$ тут $\;\;\vect{m}=\frac{\pi R^2_0 I}{c}\vect{n}_z$ --- магнітний момент кільця радіусом $R_0$ із струмом $I$.
\end{solution}
\end{problem}

\begin{problem}
 Дві рівномірно заряджені кульки із зарядами $q_1$, $q_2$ і радіусами $a_1$, $a_2$ обертаються без поступального руху з кутовими швидкостями $\omega_1$, $\omega_2$ так, що вектори $\vect{\omega}_1$ , $\vect{\omega}_2$ перпендикулярні відрізку $l$, що з’єднує центри кульок ($l \gg a_1,a_2$). Оцінити силу взаємодії кульок. 
	\begin{solution}
\[\vect{F}=\frac{3}{25}\frac{q_1 q_2}{c^2}\omega_1 \omega_2 \frac{\vect{l}}{l^5}+\frac{q_1 q_2\vect{l}}{l^3}\]

\end{solution}
\end{problem}

\begin{problem}
 Знайти магнітний момент однорідно зарядженої кулі (сфери), що обертається навколо одного із своїх діаметрів з кутовою швидкістю $\omega$. Заряд кулі – $q$, радіус – $a$.   
	\begin{solution}

Магнітний момент зарядженої сфери:
\[m = \frac{q a^2\omega}{3c}\]	
Магнітний момент зарядженої кулі:
\[m = \frac{q a^2\omega }{5c}\] 
\end{solution}
\end{problem}

\begin{problem}
Знайти магнітне поле в тонкій плоскій щілині, якщо поле в середовищі ($\mu$) можна вважати однорідним.
	\begin{solution}
\[\Hfield=\frac{1}{\mu}\vect{B} +\left(1-\frac{1}{\mu}\right)\left(\vect{B}\cdot \vect{n}\right)\vect{n}\]	 
\end{solution}
\end{problem}
\begin{problem}
В однорідне магнітне поле $H_0$ вноситься куля радіуса $a$ з магнітною проникністю $\mu_1$. Визначити результуюче поле, індукований магнітний момент кулі $\vect{m}$ , намагніченість $\vect{M}$ і густину струмів $\vect{j}_\text{мол}$, які еквівалентні намагніченості, набутій кулею. Магнітна проникність оточуючого середовища $\mu_2$.    
	\begin{solution}
\[\begin{array}{l}
{\Hfield}_1 = \frac{{3\mu _2}}{\mu_1 + 2\mu_2}{\Hfield}_0\\
{\Hfield}_2 = {\Hfield}_0 - \frac{\vect{m}}{r^3} + \frac{3\left({\vect{m} \cdot \vect{r}} \right)\vect{r}}{r^5}\\
\vect{m} = \frac{\mu_1 - \mu_2}{\mu_1 + 2\mu_2}{a^3}{\Hfield}_0 \\
\vect{M} = \frac{\vect{m}}{\frac{4}{3}\pi a^3}
\end{array}\]
\end{solution}
\end{problem}
\begin{problem}
Розв’язати попередню задачу і знайти індукований магнітний момент $\vect{m}$ та напруженість магнітного поля $\Hfield$, якщо в магнітне поле вноситься рівномірно намагнічена сфера радіуса $a$ (ідеалізований феромагнетик), вектор намагніченості якого $\vect{M}_0$.    
	\begin{solution}

\[\begin{array}{l}
{\Hfield}_1 = \frac{{3\mu_2}}{\mu_1 + 2\mu_2}{\Hfield}_0- \frac{4\pi}{\mu_1 + 2\mu_2}{\vect{M}_0},\;\;\;\;\;\;r \le a\\
{\Hfield}_2 = {\Hfield}_0 - \frac{\vect{m}}{r^3} + \frac{3\left({\vect{m} \cdot \vect{r}} \right)\vect{r}}{r^5},\;\;\;\;\;\;r > a\\
 \vect{m} = \frac{\mu_1 - \mu_2}{\mu_1 + 2\mu_2}{a^3}\Hfield_0 +\frac{4\pi a^3}{\mu_1 + 2\mu _2}{\vect{M}_0}
\end{array}\]	 

\end{solution}
\end{problem}
\begin{problem}
Знайти поле постійного кулеподібного магніту з намагніченістю $\vect{M}$ і магнітною проникністю $\mu$.    
	\begin{solution}

Поле всередині:  ${\Hfield}_{in} = \frac{8\pi}{\mu + 2}\vect{M}$

Поле ззовні: ${\Hfield}_{out} =  - \frac{\vect{m}}{r^3} + \frac{3\left( {\vect{m}\cdot\vect{r}} \right)\vect{r}}{r^5}$
	 
\end{solution}
\end{problem}

\begin{problem}
Знайти максимальне магнітне поле кулеподібного постійного магніту радіуса $R$, прийнявши в даному випадку залежність $B\left(H\right)= 4\pi B_0 \left(1+\frac{H}{H_0}\right)$  , де поле насичення $B_0$, а коерцетивна сила $H_0$.
	\begin{solution}
\[B_{\max} = \frac{B_0}{1 + \frac{B_0}{2H_0}}\]	
\end{solution}
\end{problem}

\begin{problem}
Всередині металічної сфери радіуса R по діаметру проходить тонка провідна дротинка радіуса $r_0 \gg R$. По ній йде струм І, який далі розтікається по сфері і знову сходиться до дротинки. Знайти: а) магнітне поле всередині і ззовні сфери; б) оцінити індуктивність системи. 
	\begin{solution}

Всередині сфери: $H_\alpha = \frac{2I}{cr}$

Ззовні сфери: $H =0$

Індуктивність системи: $L \approx 4R\ln \frac{R}{r_0}$	
\end{solution}
\end{problem}


\begin{problem}
    $\vect{F}$
$\Efield$
$\Bfield$
$\Dfield$
$\Hfield$
$\vect{a}$
наприклад, живіт з роками змінює свою округлість від $R = 0$ до $R = \infty$, тобто в межах $R = 0 \ldots \infty$.
	\begin{solution}
	Відповідь: 
\end{solution}
\end{problem}

\begin{problem}
 
	\begin{solution}
	
\end{solution}
\end{problem}

\begin{problem}
    
	\begin{solution}
	Відповідь: 
\end{solution}
\end{problem}

\begin{problem}
    
	\begin{solution}
	Відповідь: 
\end{solution}
\end{problem}



\begin{center}
\begin{tikzpicture}
	\draw (0,0) -- +(3,0);
\end{tikzpicture}

\end{center}
\end{document}
