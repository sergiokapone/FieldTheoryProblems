\documentclass{article}
\usepackage{amsmath}
\usepackage{amssymb}
\usepackage{physics}

	\usepackage{fontspec}
	\setsansfont{CMU Sans Serif}%{Arial}
	\setmainfont{CMU Serif}%{Times New Roman}
	\setmonofont{CMU Typewriter Text}%{Consolas}
	\defaultfontfeatures{Ligatures={TeX}}
	\usepackage[math-style=TeX]{unicode-math}

\usepackage[%
	a4paper,%
	footskip=1cm,%
	headsep=0.3cm,%
	top=2cm, %поле сверху
	bottom=2cm, %поле снизу
	left=2cm, %поле ліворуч
	right=2cm, %поле праворуч
    ]{geometry}

\begin{document}


У нескінченому прямому циліндричному провіднику радіуса \( R \) тече струм із густиною:
\[
\vb{j}(r) = \frac{a}{r} \vb{e}_z,
\]
де \( a = \text{const} \), \( r \) — відстань від осі циліндра. Магнітна проникність: \( \mu =
\text{const} \) всередині провідника та \( \mu_0 \) (вакуум) зовні. Потрібно знайти векторний
потенціал \( \vb{A}(r) \) з умовою калібрування \( \nabla\cdot \vb{A} = 0 \).

\section{Розв'язок}
Векторний потенціал \( \vb{A} \) має лише \( z \)-компоненту: \( \vb{A} = A_z(r) \vb{e}_z \). Рівняння
для \( A_z(r) \):
\[
\nabla^2 A_z = -\mu \frac{a}{r}.
\]

\subsection{Внутрішня область (\( r \leq R \))}
У циліндричних координатах:
\[
\frac{1}{r} \dv{r} \left( r \dv{A_z}{r} \right) = -\mu \frac{a}{r}.
\]
Інтегруємо:
\[
\dv{r} \left( r \dv{A_z}{r} \right) = -\mu a.
\]
Перше інтегрування:
\[
r \dv{A_z}{r} = -\mu a r + C_1.
\]
Поділимо на \( r \):
\[
\dv{A_z}{r} = -\mu a + \frac{C_1}{r}.
\]
Друге інтегрування:
\[
A_z(r) = -\mu a r + C_1 \ln r + C_2.
\]
Оскільки \( A_z \) має бути скінченним при \( r = 0 \), то \( C_1 = 0 \). Тому:
\[
A_z^{\text{in}}(r) = -\mu a r + C_2.
\]

\subsection{Зовнішня область (\( r \geq R \))}
Тут \( \vb{j} = 0 \), тому:
\[
\frac{1}{r} \dv{r} \left( r \dv{A_z}{r} \right) = 0.
\]
Інтегруємо:
\[
r \dv{A_z}{r} = C_3.
\]
Поділимо на \( r \):
\[
\dv{A_z}{r} = \frac{C_3}{r}.
\]
Інтегруємо ще раз:
\[
A_z^{\text{out}}(r) = C_3 \ln r + C_4.
\]

\subsection{Умови зшивання при \( r = R \)}
\begin{enumerate}
    \item Неперервність \( A_z \):
    \[
    -\mu a R + C_2 = C_3 \ln R + C_4.
    \]
    \item Неперервність \( \dv{A_z}{r} \):
    \[
    -\mu a = \frac{C_3}{R}.
    \]
\end{enumerate}
З другого рівняння:
\[
C_3 = -\mu a R.
\]
Підставляємо \( C_3 \) у перше рівняння:
\[
-\mu a R + C_2 = -\mu a R \ln R + C_4.
\]
Виберемо \( C_4 = 0 \) (калібрувальна свобода), тоді:
\[
C_2 = \mu a R (1 - \ln R).
\]

\subsection{Фінальні вирази}
\begin{itemize}
    \item Всередині провідника (\( r \leq R \)):
    \[
    A_z^{\text{in}}(r) = \mu a (R - r) - \mu a R \ln R.
    \]
    \item Зовні провідника (\( r \geq R \)):
    \[
    A_z^{\text{out}}(r) = -\mu a R \ln r.
    \]
\end{itemize}
Можна опустити загальну константу \( -\mu a R \ln R \), оскільки вона не впливає на магнітне поле \(
\vb{B} = \curl \vb{A} \). Тому остаточно:
\[
\boxed{
\vb{A}(r) =
\begin{cases}
\mu a (R - r) \vu{e}_z, & r \leq R, \\
-\mu a R \ln \left( \dfrac{r}{R} \right) \vu{e}_z, & r \geq R.
\end{cases}
}
\]

\subsection*{Перевірка}
\begin{itemize}
    \item Умова калібрування \( \nabla\cdot \vb{A} = \pdv{A_z}{z} = 0 \) виконується.
    \item На межі \( r = R \):
    \[
    A_z^{\text{вн}}(R) = 0, \quad A_z^{\text{зов}}(R) = -\mu a R \ln R.
    \]
    Щоб була неперервність, потрібно додати константу \( \mu a R \ln R \) до \( A_z^{\text{in}} \), що
    вже зроблено.
    \item Похідна на межі:
    \[
    \dv{A_z^{\text{вн}}}{r} \bigg|_{r=R} = -\mu a, \quad \dv{A_z^{\text{out}}}{r} \bigg|_{r=R} = -\mu
    a.
    \]
    Умова зшивання похідної виконується.
\end{itemize}

\end{document}