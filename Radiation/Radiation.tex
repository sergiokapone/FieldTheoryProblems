% !TeX program = lualatex
% !TeX encoding = utf8
% !TeX spellcheck = uk_UA
% !TeX root =../FTProblems.tex

%=========================================================
\Opensolutionfile{answer}[\currfilebase/\currfilebase-Answers]
\Writetofile{answer}{\protect\section*{\nameref*{\currfilebase}}}
\chapter{Випромінювання та \mbox{поширення} електромагнітних хвиль}\label{\currfilebase}
%=========================================================
%\epigraph{\Annabelle  It's of no use whatsoever. This is just an experiment that proves Maestro Maxwell was right --- we just have these mysterious electromagnetic waves that we cannot see with the naked eye. But they are there.%
%}{Heinrich Hertz}

\begin{Theory}%[Потенціали Ліенара-Віхерта]
	Потенціали полів, які створюються точковим зарядом $q$, що рухається по заданій траєкторії $\vect{r}_q = \vect{r}_q(t_q)$ дається виразами:
	\begin{equation}\label{Lienard-Wikhert}
		\begin{aligned}
			\pot(t,\vect{r}) =  \frac{q}{\left( R_q - \frac{\vect{v}_q\cdot \vect{R}_q}{c}\right) }, \quad
			\vect{A}(t,\vect{r}) =\frac{q\vect{v}_q/c}{\left( R_q - \frac{\vect{v}_q\cdot \vect{R}_q}{c} \right)},
		\end{aligned}
	\end{equation}
	де
    \begin{equation*}
        \vect{R}_q =  \vect{r} - \vect{r}_q(t_q), \quad R_q = |\vect{R}_q|, \quad  \vect{v}_q = \dot{r}_q(t_q),
    \end{equation*}
    а $t_q$ є розв'язком рівняння
	\begin{equation}\label{Lienard-Wikhert-time}
		t_q = t - \frac{|\vect{r} - \vect{r}_q(t_q)|}{c}.
	\end{equation}

	\begin{center}
		\begin{tikzpicture}[scale=1]
			\coordinate (O) at (-4,-3);
			\draw[-latex] (O) -- ([yshift=3cm]O) node[left] {$z$};
			\draw[-latex] (O) -- ([xshift=3cm]O)  node[below] {$y$};
			\draw[-latex] (O) -- ([shift={(225:3)}]O)  node[left] {$x$};
			\draw[red, thick] (-2,2) .. controls (-1,0) and (1,0) .. (2,-2)  coordinate[pos=0.2] (A) coordinate[pos=0.8] (B);
			\fill[black] (A) circle (1pt);
			\draw[-latex] (A) -- +(-45:1.5) node[below] {$\vect{v}_q$};
			\fill[black] (B) circle (1pt);
			\draw[-latex] (A) -- node[above] {$\vect{R}_q(t_q)$} (5,1.5) coordinate[pos=0.1] (C1) coordinate (C);
			%	\draw[-latex] (A) -- (C1) ;
			\draw[-latex] ([xshift=-2cm, yshift=1cm]A) node[left, text width=10em] {\itshape\small Положення частинки в момент $t_q = t - R_q/c$} to [bend right] (A);
			\draw[-latex] ([xshift=0cm, yshift=-1cm]B) node[below, text width=10em] {\itshape\small Положення частинки в момент $t$} to [bend right] (B);
			\fill[black] (C) circle (1pt);
			\draw[-latex] (O) -- node[above=5pt] {$\vect{r}_q$} (A);
			\draw[-latex] (O) -- node[pos=0.7, above] {$\vect{r}$} (C);
%			\node[below right, text width=5em] at (C) ;
%			\draw[-latex, dashed] (B) -- node[below=5pt] {$\vect{R}_q(t)$}  (C);
		\end{tikzpicture}
		\captionof{figure}{До розрахунку запізнюючих потенціалів}
		\label{pic:retardion}
	\end{center}

	Поля випромінювання електричного диполя, розташованого в початку координат $\vect{r} = 0$, у хвильовій зоні:
	\begin{align}\label{eq:dipole_radiation}
		\Bfield(t, \vect{r}) & = \frac{\left[ \ddot{\vect{p}}\left(t - \frac{r}{c}\right)\times \vect{n}\right] }{c^2 r}                                                                       \\
		\Efield(t, \vect{r}) & = \left[ \Bfield \times\vect{n}\right] = \frac{\left[ \left[ \ddot{\vect{p}}\left(t - \frac{r}{c}\right)\times \vect{n}\right]\times \vect{n}\right]  }{c^2 r},
	\end{align}
	де $\vect{n} = \frac{\vect{r}}{r}$~--- одиничний вектор в напрямку точки спостереження.

	Діаграма спрямованості дипольного випромінювання:
	\begin{equation}\label{eq:diagram_direction}
		\frac{dN}{do} = \frac{\ddot{\vect{p}}^2}{4\pi c^3}\sin^2\theta.
	\end{equation}
	Повна потужність випромінювання  в усіх напрямках:
	\begin{equation}\label{eq:power}
		N = \frac{2}{3c^3}\ddot{\vect{p}}^2.
	\end{equation}

%	Тензор поляризації (наближено монохроматичного) світла:
%	\begin{equation}\label{eq:polarization_tensor}
%		\rho_{\alpha\beta} = \frac{\left\langle E_{0\alpha} E_{0\beta}^* \right\rangle }{\left\langle \Efield_0\Efield^*_0\right\rangle },
%	\end{equation}
%	де $\Efield_0$~--- амплітуда хвилі, індекси пробігають по два значення $\alpha,\beta = 1,2$, що відповідає осям $y$ та $z$, а вісь $x$ напрямлена
%вздовж напрямку розповсюдження хвилі.
%
%	Ступінню поляризації називається величина $P$, яка визначається як:
%	\begin{equation}\label{eq:polarization_lewel}
%		|\rho_{ij}| = \frac14(1 - P^2),
%	\end{equation}
%	де $|\rho_{ik}|$ визначник тензора поляризації~\eqref{eq:polarization_tensor}.
При розгляді поляризації випромінювання треба зважати на векторну природу польових функції, зокрема, напруженості електричного поля. Зауважимо, що для
електромагнітної хвилі, що поширюється у фіксованому напрямку, досить розглядати лише електричне поле; воно визначає також і вектор напруженості
магнітного поля.

Розглянемо випадкове поле випромінювання, що поширюється в напрямку осі $x_3$, з напруженістю електричного поля  $\Efield = (E_1,E_2,0) \equiv
(E_x,E_y,0)$, причому будемо вважати, що поле є суперпозицією монохроматичних хвиль $\sim e^{-\omega t}$  з близькими частотами. Напрямок вектора
$\Efield$, що є ортогональним напрямку поширення хвилі, визначає поляризацію. Для випадкового поля $\Efield$ введемо тензор поляризації (за~\cite{LL2}):
	\begin{equation}\label{eq:polarization_tensor}
		\rho_{ij} = \frac{\left\langle E_{i} E_{j}^* \right\rangle }{\left\langle |E_1|^2\right\rangle + \left\langle |E_2|^2\right\rangle },\quad i,j =
		1,2.
	\end{equation}
Оскільки область значень $\det||\rho|| = $ є відрізком $[0,1/4]$, можна записати:
	\begin{equation}\label{eq:polarization_lewel}
		\det||\rho|| = \frac14(1 - P^2),\quad 0\le P \le 1,
	\end{equation}
де параметр $P$ називають \emph{ступінню поляризації}; він приймає значення від $0$ (неполяризоване світло) до $1$.
\end{Theory}




\section{Співвідношення невизначеностей}

%=========================================================
\begin{problem}
Електричний сигнал має форму
\[
	\phi(t) = \frac{1}{\sqrt{4\pi\sigma}} \exp\left( -\frac{t^2}{2\sigma^2}\right).
\]
Розрахувати (прямим обчисленням) величину $\left\langle \Delta t^2 \right\rangle \left\langle \left(\Delta \omega  - \left\langle \omega  \right\rangle\right)^2 \right\rangle$  у співвідношенні невизначеностей.
\begin{solution}
	$1/4$ ( реалізується знак рівності у співвідношенні невизначеностей)
\end{solution}
\end{problem}


\begin{problem}
Коли в співвідношенні невизначеностей для хвильових пакетів
\[
	\left\langle \Delta t^2 \right\rangle\left\langle \Delta \omega^2 \right\rangle \ge \frac14
\]
має місце знак рівності? Відповідь обґрунтувати й знайти відповідну форму сигналу.
\end{problem}


\section{Монохроматичні хвилі. Інтерференція та поляризація хвиль}

%=========================================================
%\begin{problem}
%    В монохроматичній плоскій електромагнітній хвилі, що поширюється у вакуумі, електричне поле в деякій точці та в деякій момент дорівнює $E_0$. Чому дорівнює вектор Пойнтінга густини потоку електромагнітної енергії?
%\end{problem}

%=========================================================
\begin{problem}
Гармонічні коливання двох електричних диполів відбуваються з однаковою частотою $\omega$, але зсунуті по фазі на $\frac{\pi}{2}$. Амплітуди дипольних моментів дорівнюють $p_0$ і утворюють кут $\frac{\pi}{2}$ один з одним. Відстань між диполями мала порівняно з довжиною хвилі. Знайти середню за часом потужність випромінювання системи.
\end{problem}

%=========================================================
\begin{problem}
Гармонічні коливання двох електричних диполів синфазні і відбуваються з однаковою частотою $\omega$. Амплітуди дипольних моментів дорівнюють $p_0$ і утворюють кут $\frac{\pi}{2}$ один з одним. Відстань між диполями мала порівняно з довжиною хвилі. Знайти середню за часом потужність випромінювання системи.
\end{problem}

%=========================================================
\begin{problem}
Два електричних диполя розташовані на прямій на відстані $m\frac{\lambda}{2}$, $m \in \mathbb{N}$. Їх дипольні моменти напрямлені перпендикулярні до цієї прямої, вони синфазні, дорівнюють одне одному та осцилюють з частотою $\omega  = \frac{2\pi c}{\lambda}$ та амплітудою $p_0$. Знайти кутовий розподіл випромінювання та повну інтенсивність.
\end{problem}


%=========================================================
\begin{problem}
Два електричних диполя розташовані на прямій на відстані $m\frac\lambda2$ ($m = 1,2, \ldots$) один від одного. Їх дипольні моменти напрямлені вздовж
цієї прямої, вони синфазні, дорівнюють одне одному та осцилюють з частотою $\omega = \frac{2\pi c}{\lambda}$ та амплітудою $p_0$. Знайти кутовий
розподіл випромінювання (діаграму спрямованості) та повну інтенсивність.
\end{problem}


%=========================================================
\begin{problem}%
    Електромагнітна хвиля поширюється в напрямку осі $OZ$; напруженість  електричного поля у початку координат  $\Efield = f(t)\left(a\vect{e}_x +
    b\vect{e}_y \right) $;  $a$, $b$ --- комплексні числа,  $f(t)$ --- випадкова функція з нульовим середнім,  $\left\langle |f(t)|^2\right\rangle =
    E_0^2$,  $\vect{e}_x$, $\vect{e}_y$ --- орти осей декартових координат. Обчислити тензор поляризації та знайти ступінь поляризації.
\begin{solution}
$P = 1$.
\end{solution}
\end{problem}


%=========================================================
\begin{problem}%
Електромагнітне випромінювання є сумою двох електромагнітних  плоских хвиль, згенерованих незалежними джерелами. Хвилі поширюються в напрямку осі $OZ$,
напруженості електричних полів у початку координат відповідно $\Efield_1 = f_1(t)\left(\vect{e}_x + i
    \vect{e}_y \right) $  та $\Efield_2 = f_2(t) \vect{e}_x $, де  $f_1(t)$, $f_2(t)$  --- випадкові функції з нульовими середніми,  $\left\langle
    |f_1(t)|^2\right\rangle = \left\langle |f_2(t)|^2\right\rangle =
    E_0^2$, $\vect{e}_x$, $\vect{e}_y$ --- орти осей декартових координат. Знайти ступінь поляризації.
\begin{solution}
$P = \frac1{\sqrt{3}}$.
\end{solution}
\end{problem}


%=========================================================
\begin{problem}%
   Електромагнітне випромінювання є сумою двох електромагнітних плоских хвиль, згенерованих незалежними джерелами. Хвилі поширюються в напрямку осі
   $OZ$,
   напруженості електричних полів у початку координат відповідно $\Efield_1 = f_1(t)\vect{e}_x $  та $\Efield_2 = f_2(t) \vect{e}_y $, де  $f_1(t)$,
   $f_2(t)$  --- випадкові функції з нульовими середніми,  $\left\langle
    |f_1(t)|^2\right\rangle = \left\langle |f_2(t)|^2\right\rangle =
    E_0^2$, $\vect{e}_x$, $\vect{e}_y$ --- орти осей декартових координат. Знайти ступінь поляризації.
\begin{solution}
$P = 0$.
\end{solution}
\end{problem}

%=========================================================
\begin{problem}
Два електричних диполя розташовані на осі аплікат з ортом $\vect{e}_z$  на відстані $d \ll \lambda$ . Їх дипольні моменти дорівнюють:
\begin{align*}
	\vect{p}_1(t) & = p_0\vect{e}_z\exp(-i\omega t),                                           \\
	\vect{p}_2(t) & = p_0\vect{e}_x\exp\left[ - i\left(\omega t + \frac{\pi}{2}\right)\right].
\end{align*}
Знайти тензор поляризації випромінювання у напрямку, перпендикулярному дo $\vect{e}_z$ та $\vect{e}_x$.
%\begin{solution}
%	Електричні поля випромінювання диполів у напрямку, перпендикулярному $\vect{e}_z$:
%	\begin{align*}
%		\Efield_1 & = E_0\vect{e}e^{-i\omega t}, \\
%		\Efield_2 & = \Efield_1e^{-i\frac\pi2},
%	\end{align*}
%	де
%	\[
%		E_0 = \frac{\omega^2 p_0}{cr^2}.
%	\]
%
%	Тому, сумарне поле $\Efield = \Efield_1 + \Efield_2$ буде дорівнювати:
%	\[
%		\Efield =\sqrt2 E_0\vect{e}_ze^{-i(\omega t + \pi/4)}.
%	\]
%    Ненульова компонента тензора поляризації $\rho_{zz} = 1$.
%\end{solution}
\end{problem}

%=========================================================
%\begin{problem}
%Два електричних диполя розташовані на осі аплікат з ортом $\vect{e}_z$  на відстані $d \ll \lambda$ . Їх дипольні моменти дорівнюють:
%\begin{align*}
%	\vect{p}_1(t) = p_0\vect{e}_z\exp(-i\omega t), \\
%	\vect{p}_2(t) = p_0\vect{e}_x\exp(-i\omega t ).
%\end{align*}
%Знайти тензор поляризації випромінювання у напрямку, перпендикулярному дo $\vect{e}_z$ та $\vect{e}_x$.
%%\begin{solution}
%%    $\{\rho\} =
%%\begin{pmatrix}
%%0 & 1 \\
%%1 & 0
%%\end{pmatrix}$.
%%\end{solution}
%\end{problem}

%=========================================================
\begin{problem}
Два електричних диполя розташовані на осі аплікат з ортом $\vect{e}_z$  на відстані $d \ll \lambda$ . Їх дипольні моменти дорівнюють:
\begin{align*}
	\vect{p}_1(t) = p_0\vect{e}_z\exp(-i\omega t)\cos\xi, \\
	\vect{p}_2(t) = p_0\vect{e}_z\exp(-i\omega t )\sin\xi.
\end{align*}
Знайти кутовий розподіл випромінювання та повну інтенсивність. Випадкова величина  $\xi$ рівномірно розподілена на проміжку $0 \le \xi \le 2\pi$.
\end{problem}

%=========================================================
\begin{problem}
Два електричних диполя розташовані на осі аплікат з ортом $\vect{e}_z$  на відстані $d \ll \lambda$ . Їх дипольні моменти дорівнюють:
\begin{align*}
	\vect{p}_1(t) & = p_0\vect{e}_z\exp(-i\omega t)\cos\xi,        \\
	\vect{p}_2(t) & = p_0\vect{e}_z\exp(-i\omega t +\pi/2)\sin\xi.
\end{align*}
Знайти ступінь поляризації та тензор поляризації випромінювання у напрямку, перпендикулярному до $\vect{e}_z$. Випадкова величина $\xi$ рівномірно розподілена на проміжку $0 \le \xi \le 2\pi$.
\end{problem}



\section{Випромінювання та розсіювання електромагнічних хвиль}


%=========================================================
\begin{problem}
Отримайте потенціали Ліенара-Віхерта~\eqref{Lienard-Wikhert} із загальних формул для запізнюючих потенціалів~\eqref{Retarded_potentials}.
\end{problem}


%=========================================================
\begin{problem}
Доведіть, що у випадку заряду, який рухається зі швидкістю $\vect{v} = \const$ запізнюючі потенціали~\eqref{Lienard-Wikhert} дають значення, які співпадають з <<миттєвими>> потенціалами:
\[
	\pot =  \frac{q}{R_t\sqrt{1-\frac{v^2}{c^2}\sin^2\theta}}, \quad
	\vect{A} =\frac{q\vect{v}}{R_t\sqrt{1-\frac{v^2}{c^2}\sin^2\theta}},
\]
де $R_t = |\vect{r} - \vect{r}_q(t)|$~--- відстань від заряду до точки спостереження в момент спостереження (див. рис.~\ref{pic:retardion}), а $\theta$~--- кут між $\vect{R}_t$ та $\vect{v}$.
\begin{solution}
	При $\vect{v} = \const$  вираз
	\[
		\vect{R}_q - \frac{\vect{v}}{c} R_q = \vect{R}_q -\vect{v}(t - t_q) = \vect{R}_t,
	\]
	тобто є вектором проведеним від заряду до точки спостереження в момент самого спостереження.
	Знаменник потенціалів Ліенара-Віхерта можна записати як:
	\[
		R_q - \frac{\vect{v}}{c} R_q = R_t\sqrt{1-\frac{v^2}{c^2}\sin^2\theta}.
	\]
\end{solution}
\end{problem}


%=========================================================
\begin{problem}
Обчислити електричне і магнітне поля рухомого заряду використовуючи потенціали Ліенара-Віхерта~\eqref{Lienard-Wikhert}.
%\begin{solution}
%	Для знаходження електричного поля з виразів для потенціалів Ліенара-Віхерта~\eqref{Lienard-Wikhert}:
%	\[
%		\phi =  \frac{q}{\left( R - \frac{\vect{v}\cdot \vect{R}}{c}\right) }, \quad
%		\vect{A} =\frac{q\vect{v}}{\left( R - \frac{\vect{v}\cdot \vect{R}}{c} \right)},
%	\]
%	використаємо формулу~\eqref{E=phi+A}:
%	\[
%		\Efield = -\frac{1}{c}\frac{\partial \vect{A}}{\partial t} - \vect{\nabla}\phi.
%	\]
%
%	Оскільки $R = c(t - t')$, а $t' = t'(t,x,y,z)$ є функцією часу і координат точки спостереження $(x,y,z,t)$, то $R$ є неявною функцією координат і часу точки спостереження. Отже,  знайдемо похідні для $R$, які нам знадобляться для знаходження полів:
%	\begin{align*}
%		\frac{\partial R}{\partial t'}                                             & = c\left( \frac{dt}{dt'} - 1 \right),                                        \\
%		\frac{\partial R^2}{\partial t'}                                           & = 2R\frac{\partial R}{\partial t'} ,                                         \\
%		\frac{\partial \vect{R}}{\partial t'}  = \frac{\partial (\vect{r} - \vect{r}')}{\partial t'} &= - \vect{v},\\
%		\frac{\partial R^2}{\partial t'} = \frac{\partial \vect{R}^2}{\partial t'} & = 2\vect{R}\frac{\partial \vect{R}}{\partial t'}  = -2\vect{R}\cdot\vect{v} ,\\
%		\frac{\partial t'}{\partial t}                                             & = \frac{1}{1- \frac{1}{\dfrac{\vect{v}\cdot\vect{R}}{cR}}}\\
%		\vect{\nabla}t' &= -\frac1c \vect{\nabla}R(t') = -\frac1c\left( \frac{\partial \vect{R}}{\partial t'} \vect{\nabla}t' + \frac{\vect{R}}{R}\right)  \\
%		\vect{\nabla}t' &= - \frac{\vect{R}}{c\left( R - \dfrac{\vect{v}\cdot\vect{R}}{c}\right) }.
%	\end{align*}
%
%	Знайдемо похідну вектор-потенціалу $\vect{A}$ по часу спостереження $t$.
%	\begin{multline*}
%		\frac{\partial\vect{A}}{\partial t} =q\frac{\partial \vect{v}(t')}{\partial t}\left(\frac{1}{ R - \frac{\vect{v}\cdot \vect{R}}{c}}\right)+q\vect{v}(t')\frac{\partial}{\partial t} \left(\frac{1}{{ R - \frac{\vect{v}\cdot \vect{R}}{c} }}\right) = \\
%		=q\frac{\partial \vect{v}(t')}{\partial t'}\left(\frac{1}{ R - \frac{\vect{v}\cdot \vect{R}}{c}}\right)\frac{\partial t'}{\partial t} + q\vect{v}(t')\frac{\partial}{\partial t'} \left(\frac{1}{{ R - \frac{\vect{v}\cdot \vect{R}}{c} }}\right)\frac{\partial t'}{\partial t},
%	\end{multline*}
%
%\end{solution}
\end{problem}

%=========================================================
\begin{problem}
Доведіть, що радіально осцилююча однорідно заряджена сфера не випромінює електромагнітні хвилі.
\end{problem}

%=========================================================
\begin{problem}
Довести, що система, яка складається з $n$ заряджених частинок, у яких однаковий питомий заряд, за відсутності зовнішніх полів не буде мати дипольного випромінювання. Частинки взаємодію за законом Кулона. Релятивістськими поправками і реакцією випромінювання знехтувати.
\begin{solution}
	Дипольний момент системи
	\[
		\vect{p} = \frac{q_i}{m_i} \sum\limits_{i = 1}^n m_i \vect{r}_i = \frac{q_i}{m_i} M\vect{R}_\mathrm{CM},
	\]
	де $\vect{R}_\mathrm{CM}$~--- радіус-вектор центра мас системи. Нехтуючи втратами імпульсу на випромінювання, у першому наближенні за відсутності зовнішніх полів $\ddot{\vect{R}}_\mathrm{CM} = 0$, отже інтенсивність випромінювання дорівнює нулю.
\end{solution}
\end{problem}

%=========================================================
\begin{problem}
Модуль вектора індукції магнітного поля дипольного випромінювання, що  створюється системою струмів на великій відстані $r$, має вигляд:
\[
	B(t,r,\theta,\phi) = r^{-n}f\left(t - \frac{r}{c},\theta, \phi\right),
\]
де $f$~--- довільна функція вказаних аргументів. Чому дорівнює $n$?
\end{problem}

%=========================================================
\begin{problem}
Заряд $q$ рухається з малою швидкістю $\vect{v}$ та прискоренням $\dot{\vect{v}}$  в обмеженій області простору. Знайти наближені вирази електромагнітного поля $\Efield$, $\Bfield$ частинки в точках, відстань до яких від частинки набагато більша у порівнянні з розмірами області руху заряду. Визначити також кутовий розподіл випромінювання та повну потужність випромінювання.
\begin{solution}%
	Поля $\Efield(t) = \frac{q\vect{n}\times[ \vect{n} \times \dot{\vect{v}}] }{c^2r}$,
	$\Bfield(t) = \frac{q\dot{\vect{v}}\times\vect{n}}{c^2r}$,
	розподіл випромінювання $\frac{dN}{do} = \frac{q^2}{4\pi c^3}[\dot{\vect{v}}\times\vect{n}]^2$, повна потужність $N = \frac{2q^2}{3c^3}\dot{\vect{v}}^2$. Прискорення $ \dot{\vect{v}} $ є функцією $t - \frac{r}{c}$.
\end{solution}
\end{problem}

%=========================================================
\begin{problem}
Визначити середню повну потужність випромінювання диполя $\vect{p}_e$, що обертається в одній площині з постійною кутовою швидкістю~$\omega$.
\begin{solution}
	$I = \frac{2p^2\omega^4}{3c^3}$.
\end{solution}
\end{problem}

%=========================================================
\begin{problem}
Два електричні диполі, що розташовані на осі $OX$ і орієнтовані уздовж осі $OZ$, осцилюють з однаковою амплітудою $p_0$ і точно в протифазі. Їх $x$-координати відрізняються на $\nfrac\lambda2$. Обчисліть вектор Пойнтінга на великих відстанях від цієї системи.
\begin{solution}
	$\vect{S} = \frac{p_0^2ck^4}{2\pi r^2}\sin^2\theta\sin^2\left(\frac\pi2\cos\phi\sin\theta\right)\ \vect{n}$, де $k=\nfrac{2\pi}{\lambda}$, $r,\theta,\phi$~--- сферичні координати, $\vect{n} = \nfrac{\vect{r}}{r}$.
\end{solution}
\end{problem}

%=========================================================
\begin{problem}% Батыгин, 1978, 778
Дві нерелятивістські частинки з зарядами $q_1$ та $q_2$ і масами $m_1$ та $m_2$ здійснюють коловий рух на відстані $r$ одна від одної. Знайдіть потужність випромінювання. Віддачею випромінювання на рух частинок знехтувати.
\begin{solution}
	Згідно~\eqref{eq:power} і з урахуванням $\vect{p} = q_1\vect{r}_1 + q_2\vect{r}_2 = \mu\left( \frac{q_1}{m_1}  - \frac{q_2}{m_2}\right)\vect{r} $, де $\mu$~--- приведена маса, а $\vect{r} = \vect{r}_2 - \vect{r}_1$, маємо:
	\[
		N(t) = \frac{2\mu^2}{3c^3}\left( \frac{q_1}{m_1}  - \frac{q_2}{m_2}\right)^2\ddot{\vect{r}}^2(t).
	\]
	Згідно рівнянню руху зарядів $\mu\ddot{\vect{r}} = {q_1q_2}\frac{\vect{r}}{r^3}$, отримуємо
	\[
		N(t) = \frac{2q_1^2q_2^2}{3c^3} \left( \frac{q_1}{m_1} - \frac{q_2}{m_2}\right)^2 \frac{1}{r^4}.
	\]
\end{solution}
\end{problem}

%=========================================================
\begin{problem}
Використовуючи рівняння~\eqref{eq:dipole_radiation} знайдіть рівняння силових ліній напруженості електричного та магнітного поля диполя, який знаходиться у початку координат і має момент $\vect{p} = p_0\cos\omega t \vect{e}_z$. Покажіть, що існують точки, де поля дорівнюють нулю в будь-який момент часу.
\end{problem}


%\section{Магнітодипольне та квадрупольне випромінювання}

%=========================================================
\begin{problem}
Сила струму в тонкому круглому витку змінюється за законом $I = I_0 \sin\omega t$. Визначити діаграму спрямованості випромінювання цього витка та повну потужність випромінювання в хвильовій зоні.
\end{problem}

%=========================================================
\begin{problem}
В тонкій нерухомій квадратній рамці зі стороною $a$ протікає струм за законом $I = I_0e^{-b t^2}$. Знайдіть повну енергію довгохвильового випромінювання за час $-\infty < t < \infty$.
\begin{solution}
	$\mathcal{E} = \frac{I_0^2a^4b}{c^5}\sqrt{2\pi b}$.
\end{solution}
\end{problem}


%=========================================================
\begin{problem}
У тонкому прямому проводі розташованому на осі $OZ$  довжиною $L = m\lambda/2$ ($m = 1,2,3, \ldots$), збуджено струм $I(t,z) = I_0 e^{-i\omega t}\sin(kz)$, де $k = \frac{\omega}{c} = \frac{m\pi}{L}$, $z \in [0,L]$. Знайти діаграму напрямленості $\frac{dN}{do}$  як функцію кута $\theta$  в сферичних координатах.
\begin{solution}
	$\left\langle {dN/do} \right\rangle  = \frac{I_0^2}{2\pi c}\frac{\left[\sin(m\pi \sin^2\theta/2) \right]^2}{(\sin\theta)^2}$
\end{solution}
\end{problem}

%=========================================================
\begin{problem}\label{prb:rad_rotated_disk}
Однорідно заряджений тонкий диск радіусом $R$ обертається навколо свого діаметра з постійною кутовою швидкістю $\omega$. Заряд диска дорівнює $q$. Знайти інтенсивність випромінювання такої системи.
\begin{solution}
	$I = \frac{q^2R^4\omega^6}{10c^5} $.
\end{solution}
\end{problem}

%=========================================================
%\begin{problem}\label{prb:rad_rotated_ball}
%Однорідна куля радіусом $R$ обертається навколо свого діаметра з постійною кутовою швидкістю $\omega$. Вісь обертання нахилена під кутом $\theta$ до напрямку зовнішнього постійного однорідного магнітного поля $\Bfield$ (рис.~\ref{rad_rotated_ball}). Заряд і маса кулі $q$ і $m$. Визначити інтенсивність випромінювання такої системи.
%\begin{solution}
%	$I = \frac{q^2\omega^2}{600c} \left( \frac{qBR}{mc^2}\right)^4 \sin^2\theta $.
%\end{solution}
%\end{problem}
%%---------------------------------------------------------
%\begin{center}
%	\begin{tikzpicture}
%		\begin{scope}[rotate=-45]
%			\draw [ball color =red!50] (0,0) circle (1);
%			\draw[dashed] (0,0) -- (0,1);
%			\draw[-latex] (0,1) -- +(0,1) node[above] {$\omega$};
%			\draw (-1,0) arc (-180:0:1 and 0.25) ;
%			\draw [dashed] (1,0) arc (0:180:1 and 0.25);
%		\end{scope}
%		\draw[dashed] (0,0) -- (0,1);
%		\draw[-latex] (0,1) -- +(0,1) node[above] {$\Bfield$};
%		\draw (0,0.5)  arc (90:45:0.5) node[pos=0.7, above] {$\theta$};
%	\end{tikzpicture}
%	\captionof{figure}{До задачі~\ref{prb:rad_rotated_ball}}
%	\label{rad_rotated_ball}
%\end{center}
%%---------------------------------------------------------

%=========================================================
%\begin{problem}
%При розпаді нерухомого ядра радіуса $R$ утворилася
%$\alpha$-частинка зі швидкістю, що дорівнює нулю. Заряд $\alpha$-частинки $q$, а її радіус дуже малий у порівнянні з $R$. В результаті
%кулонівського відштовхування $\alpha$-частинка віддалилась на нескінченність.
%Знайти діаграму спрямованості випромінювання з урахуванням малого доданка порядку $v/c \ll 1$, де $v$`--- швидкість $\alpha$-частинки на нескінченності
%\begin{solution}
%	Магнітодипольне випромінювання відсутнє.  Шукаємо квадрупольне випромінювання.
%
%	Діаграма спрямованості
%	\[
%		\frac{dI}{do} = \frac{q^2}{15\pi R}\left( \frac{v}{c}\right)^3\left( 1+ \frac58\frac{v}{c}\cos\theta\right) \sin^2\theta.
%	\]
%\end{solution}
%\end{problem}


%\section{Радіаційне тертя}

%=========================================================
%\begin{problem}
%Знайдіть силу, з якою заряджена сферично-симетрична частинка діє сама на себе (сила самодії) при прискореному поступальному нерелятивістському русі ($v \ll c$). Запізнення і скорочення Лоренца не враховуйте.
%\begin{solution}
%	Силу, що діє між двома довільними елементами частинки $dq_1$ та $dq_2$ можна записати у вигляді:
%	\[
%		d\vect{F} = \frac{dq_1dq_2}{r^2}\frac{\dot{\vect{v}} - \vect{r}_0(\vect{r}\cdot\dot{v})}{c}.
%	\]
%	де $\vect{r}$~--- радіус-вектор, напрямлений від елемента  $dq_1$ до $dq_2$.
%
%	В останній формулі кулонівська частина не врахована, оскільки є сферично-симетричною, тому не дає вкладу в самодію, як і квазістаціонарне  магнітне  поле, а тому врахована лише частина сили, яка залежить від прискорення.
%
%	Повну силу, що діє на частинку знайдемо проінтегрувавши останній вираз:
%	\[
%		\vect{F} = \int d\vect{F} = -\frac43\frac{\dot{\vect{v}}}{c^2} \int \frac{dq_1dq_2}{r} = -\frac43\frac{E_0\dot{\vect{v}}}{c^2},
%	\]
%	де $E_0 = \int \frac{dq_1dq_2}{r}$~--- енергія покою електромагнітного поля частинки. Ввівши масу спокою $m = \frac{E_0}{c^2}$, отримуємо вираз для сили самодії:
%	\[
%		\vect{F} = -m_0\dot{\vect{v}},
%	\]
%	який співпадає з силою інерції.
%\end{solution}
%\end{problem}


%=========================================================
\begin{problem}
На дві плоскі паралельні металеві сітки подали напругу, завдяки чому між ними діє електричне поле $\Efield$, яке  наближено вважаємо однорідним.  Відстань між сітками $d$, зовні сіток поля немає. Нерелятивістська частинка з масою $m$ та зарядом $q$ влітає в область, де діє електричне поле, перетинаючи обидві сітки.
Кут між вектором $ \Efield $ і напрямком швидкості $ \vect{v}_0 $ частинки при вльоті дорівнював $ 0<\alpha<\pi/2 $. Знайти енергію, що втрачає частинка на дипольне випромінювання під час прольоту між сітками.
\begin{solution}
	$W = \frac{2qEv_0d}{3mc^3}\left( \sqrt{\frac{2qEd}{mv_0^2}+\cos^2\alpha} - \cos\alpha\right) $.
\end{solution}
\end{problem}

%=========================================================
\begin{problem}
У класичній моделі атома Резерфорда електрон з масою $m$ і зарядом $e$ обертається по коловій орбіті навколо нерухомого ядра з зарядом $Z|e|$. Знайти закон зменшення повної енергії електрона, яке обумовлене дипольним випромінюванням. Обчислити час, після закінчення якого електрон впаде на ядро внаслідок втрати енергії на дипольне випромінювання. У початковий момент часу електрон знаходиться на відстані $R$ від ядра.
\begin{solution}
	$I = \frac{2e^2}{3c^3} \left( \frac{Ze^2}{mR^2}\right)^2 $, $t = \frac{m^2c^3R^3}{4Ze^4}$.
\end{solution}
\end{problem}

%=========================================================
%\section{Розсіювання електромагнітних хвиль}
%=========================================================



%=========================================================
\begin{problem}
На нерелятивістський заряд $e$, маси $m$, що розташований у точці $\vect{r}$, падає монохроматична електромагнітна хвиля $\vect{E} = \vect{E}_0e^{-i\omega t}$. Визначити диференціальний та повний переріз розсіювання електромагнітних хвиль. Власна частота коливань заряду $\omega_0$.
\begin{solution}
	При дії хвилі $\vect{E} = \vect{E}_0e^{-i\omega t}$, де $\vect{E}_0 = \const$, що падає, частинка рухається згідно рівняння
	\[
		m \left( \ddot{\vect{r}} + 2\gamma\dot{\vect{r}} + \omega_0^2\vect{r}\right) = e\Efield e^{-i\omega t}.
	\]
	Шукаємо розв’язок у вигляді $\vect{r}  = \vect{r}_0e^{-i\omega t}$. Дипольний момент $\vect{p}  = e\vect{r}_0e^{-i\omega t}$ .

	Отже,
	\[
		\vect{p} = \frac{e^2\vect{E}_0\exp{- i\omega t}}{m\left[ {\omega _0^2 - \omega ^2 - 2i\gamma \omega } \right]}.
	\]

	У хвильовій зоні $\vect{B} = \frac{\ddot{\vect{p}} \times \vect{n}}{rc^2} =  - \omega^2 \frac{\vect{p} \times \vect{n}}{rc^2}$.
	Густина потоку енергії розсіяного випромінювання (середнє за часом) $\Pi  = \frac{c}{8\pi}{\omega ^4}\frac{{\left| {\left[ {{\vect{p}} \times {\vect{n}}} \right]} \right|}^2}{{{r^2}{c^4}}}$.

	Отже, диференціальний переріз розсіювання:
	\[\frac{{d\sigma }}{{do}} = r_e^2\frac{{{\omega ^4}{{(\sin \theta )}^2}}}{{{{({\omega ^2} - \omega _0^2)}^2} + 4{\gamma ^2}{\omega ^2}}},
	\]
	де $r_e = \frac{e^2}{mc^2}$~--- класичний заряд електрона, $\theta  = (\widehat{\vect{n},\Efield_0})$.

	Повний переріз розсіювання:
	\[
		\sigma  = \frac{{8\pi }}{3}r_e^2\frac{{{\omega ^4}}}{{{{({\omega ^2} - \omega _0^2)}^2} + 4{\gamma ^2}{\omega ^2}}}
	\]
\end{solution}
\end{problem}

%=========================================================
\begin{problem}
Запишіть диференціальний і повний перерізи розсіювання лінійно поляризованого і неполяризованого світла осцилятором із
тертям, а також перерізи поглинання.
\end{problem}

%=========================================================
\begin{problem}
На два вільних заряди (зарядом  $q$), що розташовані на осі аплікат $OZ$ на відстані  $b= n\lambda/2$ один від одного, падає монохроматична електромагнітна хвиля з напруженістю електричного поля $\Efield = \Efield_0\cos(\omega t - kx)$. Вектор $\Efield_0$  паралельний осі $OZ$. Амплітуда коливань зарядів під дією поля значно менша за довжину хвилі, швидкість нерелятивістська. Знайти диференціальний переріз розсіювання. Оцінити кут між сусідніми пелюстками  перерізу діаграми розсіювання у площині $XOZ$ при $n \gg 1$  в околі напрямку $\theta = \theta_0$  ( $\theta$ кут між  $\vect{n}$ та $\vect{e}_z$).
\end{problem}


\Closesolutionfile{answer}

