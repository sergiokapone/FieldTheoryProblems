\protect \section *{\nameref *{Radiation}}
\begin{Solution}{3.{1}}
	$1/4$ ( реалізується знак рівності у співвідношенні невизначеностей)
\end{Solution}
\begin{Solution}{3.{7}}
$P = 1$.
\end{Solution}
\begin{Solution}{3.{8}}
$P = \frac1{\sqrt{3}}$.
\end{Solution}
\begin{Solution}{3.{9}}
$P = 0$.
\end{Solution}
\begin{Solution}{3.{14}}
	При $\vect{v} = \const$  вираз
	\[
		\vect{R}_q - \frac{\vect{v}}{c} R_q = \vect{R}_q -\vect{v}(t - t_q) = \vect{R}_t,
	\]
	тобто є вектором проведеним від заряду до точки спостереження в момент самого спостереження.
	Знаменник потенціалів Ліенара-Віхерта можна записати як:
	\[
		R_q - \frac{\vect{v}}{c} R_q = R_t\sqrt{1-\frac{v^2}{c^2}\sin^2\theta}.
	\]
\end{Solution}
\begin{Solution}{3.{17}}
	Дипольний момент системи
	\[
		\vect{p} = \frac{q_i}{m_i} \sum\limits_{i = 1}^n m_i \vect{r}_i = \frac{q_i}{m_i} M\vect{R}_\mathrm{CM},
	\]
	де $\vect{R}_\mathrm{CM}$~--- радіус-вектор центра мас системи. Нехтуючи втратами імпульсу на випромінювання, у першому наближенні за відсутності зовнішніх полів $\ddot{\vect{R}}_\mathrm{CM} = 0$, отже інтенсивність випромінювання дорівнює нулю.
\end{Solution}
\begin{Solution}{3.{19}}
%
	Поля $\Efield(t) = \frac{q\vect{n}\times[ \vect{n} \times \dot{\vect{v}}] }{c^2r}$,
	$\Bfield(t) = \frac{q\dot{\vect{v}}\times\vect{n}}{c^2r}$,
	розподіл випромінювання $\frac{dN}{do} = \frac{q^2}{4\pi c^3}[\dot{\vect{v}}\times\vect{n}]^2$, повна потужність $N = \frac{2q^2}{3c^3}\dot{\vect{v}}^2$. Прискорення $ \dot{\vect{v}} $ є функцією $t - \frac{r}{c}$.
\end{Solution}
\begin{Solution}{3.{20}}
	$I = \frac{2p^2\omega^4}{3c^3}$.
\end{Solution}
\begin{Solution}{3.{21}}
	$\vect{S} = \frac{p_0^2ck^4}{2\pi r^2}\sin^2\theta\sin^2\left(\frac\pi2\cos\phi\sin\theta\right)\ \vect{n}$, де $k=\nfrac{2\pi}{\lambda}$, $r,\theta,\phi$~--- сферичні координати, $\vect{n} = \nfrac{\vect{r}}{r}$.
\end{Solution}
\begin{Solution}{3.{22}}
	Згідно~\eqref{eq:power} і з урахуванням $\vect{p} = q_1\vect{r}_1 + q_2\vect{r}_2 = \mu\left( \frac{q_1}{m_1}  - \frac{q_2}{m_2}\right)\vect{r} $, де $\mu$~--- приведена маса, а $\vect{r} = \vect{r}_2 - \vect{r}_1$, маємо:
	\[
		N(t) = \frac{2\mu^2}{3c^3}\left( \frac{q_1}{m_1}  - \frac{q_2}{m_2}\right)^2\ddot{\vect{r}}^2(t).
	\]
	Згідно рівнянню руху зарядів $\mu\ddot{\vect{r}} = {q_1q_2}\frac{\vect{r}}{r^3}$, отримуємо
	\[
		N(t) = \frac{2q_1^2q_2^2}{3c^3} \left( \frac{q_1}{m_1} - \frac{q_2}{m_2}\right)^2 \frac{1}{r^4}.
	\]
\end{Solution}
\begin{Solution}{3.{25}}
	$\mathcal{E} = \frac{I_0^2a^4b}{c^5}\sqrt{2\pi b}$.
\end{Solution}
\begin{Solution}{3.{26}}
	$\left\langle {dN/do} \right\rangle  = \frac{I_0^2}{2\pi c}\frac{\left[\sin(m\pi \sin^2\theta/2) \right]^2}{(\sin\theta)^2}$
\end{Solution}
\begin{Solution}{3.{27}}
	$I = \frac{q^2R^4\omega^6}{10c^5} $.
\end{Solution}
\begin{Solution}{3.{28}}
	$W = \frac{2qEv_0d}{3mc^3}\left( \sqrt{\frac{2qEd}{mv_0^2}+\cos^2\alpha} - \cos\alpha\right) $.
\end{Solution}
\begin{Solution}{3.{29}}
	$I = \frac{2e^2}{3c^3} \left( \frac{Ze^2}{mR^2}\right)^2 $, $t = \frac{m^2c^3R^3}{4Ze^4}$.
\end{Solution}
\begin{Solution}{3.{30}}
	При дії хвилі $\vect{E} = \vect{E}_0e^{-i\omega t}$, де $\vect{E}_0 = \const$, що падає, частинка рухається згідно рівняння
	\[
		m \left( \ddot{\vect{r}} + 2\gamma\dot{\vect{r}} + \omega_0^2\vect{r}\right) = e\Efield e^{-i\omega t}.
	\]
	Шукаємо розв’язок у вигляді $\vect{r}  = \vect{r}_0e^{-i\omega t}$. Дипольний момент $\vect{p}  = e\vect{r}_0e^{-i\omega t}$ .

	Отже,
	\[
		\vect{p} = \frac{e^2\vect{E}_0\exp{- i\omega t}}{m\left[ {\omega _0^2 - \omega ^2 - 2i\gamma \omega } \right]}.
	\]

	У хвильовій зоні $\vect{B} = \frac{\ddot{\vect{p}} \times \vect{n}}{rc^2} =  - \omega^2 \frac{\vect{p} \times \vect{n}}{rc^2}$.
	Густина потоку енергії розсіяного випромінювання (середнє за часом) $\Pi  = \frac{c}{8\pi}{\omega ^4}\frac{{\left| {\left[ {{\vect{p}} \times {\vect{n}}} \right]} \right|}^2}{{{r^2}{c^4}}}$.

	Отже, диференціальний переріз розсіювання:
	\[\frac{{d\sigma }}{{do}} = r_e^2\frac{{{\omega ^4}{{(\sin \theta )}^2}}}{{{{({\omega ^2} - \omega _0^2)}^2} + 4{\gamma ^2}{\omega ^2}}},
	\]
	де $r_e = \frac{e^2}{mc^2}$~--- класичний заряд електрона, $\theta  = (\widehat{\vect{n},\Efield_0})$.

	Повний переріз розсіювання:
	\[
		\sigma  = \frac{{8\pi }}{3}r_e^2\frac{{{\omega ^4}}}{{{{({\omega ^2} - \omega _0^2)}^2} + 4{\gamma ^2}{\omega ^2}}}
	\]
\end{Solution}
